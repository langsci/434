\SetupAffiliations{output in groups = false,
                   orcid placement = after,
                   separator between two = {\bigskip\\},
                   separator between multiple = {\bigskip\\},
                   separator between final two = {\bigskip\\}
                   }

% ORCIDs in langsci-affiliations 
\definecolor{orcidlogocol}{cmyk}{0,0,0,1}
\RenewDocumentCommand{\LinkToORCIDinAffiliations}{ +m }
  {%
    \,\orcidlink{#1}%
  }

\makeatletter
\let\thetitle\@title
\let\theauthor\@author
\makeatother

% Cite and cross-reference other chapters
\newcommand{\crossrefchaptert}[2][]{\citet*[#1]{chapters/#2}, Chapter~\ref{chap-#2} of this volume} 
\newcommand{\crossrefchapterp}[2][]{(\citealp*[#1]{chapters/#2}, Chapter~\ref{chap-#2} of this volume)}
\newcommand{\crossrefchapteralt}[2][]{\citealt*[#1]{chapters/#2}, Chapter~\ref{chap-#2} of this volume}
\newcommand{\crossrefchapteralp}[2][]{\citealp*[#1]{chapters/#2}, Chapter~\ref{chap-#2} of this volume}

\newcommand{\crossrefcitet}[2][]{\citet*[#1]{chapters/#2}} 
\newcommand{\crossrefcitep}[2][]{\citep*[#1]{chapters/#2}}
\newcommand{\crossrefcitealt}[2][]{\citealt*[#1]{chapters/#2}}
\newcommand{\crossrefcitealp}[2][]{\citealp*[#1]{chapters/#2}}


\newcommand{\sub}[1]{\textsubscript{\scriptsize\textrm{#1}}}
% Müller
\newcommand{\page}{}

\let\citew\citet
\def\underRevision{Revise and resubmit}
\let\textbfemph\emph
\newcommand{\textbfremoved}[1]{#1}
\newcommand{\todostefan}[1]{\todo[color=orange!80]{\footnotesize #1}\xspace}
\newcommand{\todosatz}[1]{\todo[color=red!40]{\footnotesize #1}\xspace}
\newcommand{\inlinetodostefan}[1]{\todo[color=green!40,inline]{\footnotesize #1}\xspace}
\newcommand{\inlinetodoopt}[1]{\todo[color=green!40,inline]{\footnotesize #1}\xspace}
\newcommand{\inlinetodoobl}[1]{\todo[color=red!40,inline]{\footnotesize #1}\xspace}
\newcommand{\itd}[1]{\inlinetodoobl{#1}}
\newcommand{\itdobl}[1]{\inlinetodoobl{#1}}
\newcommand{\itdopt}[1]{\inlinetodoopt{#1}}
\newcommand{\addpages}{\todostefan{add pages}}

%% % taken from https://tex.stackexchange.com/a/95079/18561
\newbox\usefulbox

\makeatletter
\def\getslant #1{\strip@pt\fontdimen1 #1}

\def\skoverline #1{\mathchoice
 {{\setbox\usefulbox=\hbox{$\m@th\displaystyle #1$}%
    \dimen@ \getslant\the\textfont\symletters \ht\usefulbox
    \divide\dimen@ \tw@ 
    \kern\dimen@ 
    \overline{\kern-\dimen@ \box\usefulbox\kern\dimen@ }\kern-\dimen@ }}
 {{\setbox\usefulbox=\hbox{$\m@th\textstyle #1$}%
    \dimen@ \getslant\the\textfont\symletters \ht\usefulbox
    \divide\dimen@ \tw@ 
    \kern\dimen@ 
    \overline{\kern-\dimen@ \box\usefulbox\kern\dimen@ }\kern-\dimen@ }}
 {{\setbox\usefulbox=\hbox{$\m@th\scriptstyle #1$}%
    \dimen@ \getslant\the\scriptfont\symletters \ht\usefulbox
    \divide\dimen@ \tw@ 
    \kern\dimen@ 
    \overline{\kern-\dimen@ \box\usefulbox\kern\dimen@ }\kern-\dimen@ }}
 {{\setbox\usefulbox=\hbox{$\m@th\scriptscriptstyle #1$}%
    \dimen@ \getslant\the\scriptscriptfont\symletters \ht\usefulbox
    \divide\dimen@ \tw@ 
    \kern\dimen@ 
    \overline{\kern-\dimen@ \box\usefulbox\kern\dimen@ }\kern-\dimen@ }}%
 {}}
\makeatother

% 1_intro.tex

% For the block quote:
\definecolor{linequote}{RGB}{224,215,188}
\definecolor{backquote}{RGB}{249,245,233}

\NewDocumentEnvironment{myquote}{ +m }
  {%
    \begin{tblsfilled}{}[black!12]
    #1%
  }
  {\end{tblsfilled}}

% 2_gibson.tex


% Example(s) Environments
% 12pt, No new-lines after example number is printed

\newcounter{examplectr}
\newcounter{fnexamplectr}

% Note: don't use subexamples in footnotes.

% This line is to overcome a bug in cmu-art style: it prints counter
% values to the aux file using \theaux... rather than using \the...
\def\theauxexamplectr{\theexamplectr}

\newcounter{subexamplectr}
\def\theauxsubexamplectr{\thesubexamplectr}
\def\theauxfnexamplectr{\thefnexamplectr}

\renewcommand{\theexamplectr}{\arabic{examplectr}}
% This command causes example numbers to appear without following periods

\renewcommand{\thefnexamplectr}{\roman{fnexamplectr}}
% This command causes example numbers to appear without following periods

\renewcommand{\thesubexamplectr}{\theexamplectr\alph{subexamplectr}}
% This command gives the number of an example and subexample as e.g. 1a, 2b

\newlength{\wdth}
\newcommand{\strike}[1]{\settowidth{\wdth}{#1}\rlap{\rule[.5ex]{\wdth}{1pt}}#1}

\newcommand{\exref}[1]{(\ref{#1})}
% This command puts reference numbers with parentheses
% surrounding them 

% The environment ``examples'' gives a list of examples, one on each line,
% numbered with a lower case alphabetic character
\newenvironment{examples}%
   { \vspace{-\baselineskip}
     \begin{list}%
     \textrm{\alph{subexamplectr}.}%
     {\usecounter{subexamplectr}
     \setlength{\topsep}{-\parskip}
     \setlength{\itemsep}{-2pt}
     \setlength{\leftmargin}{0.5in}
     \setlength{\rightmargin}{0in} } }%
   { \end{list}}

% The environment ``myexample'' outputs an arabic counter ``examplectr''
% surrounded by parentheses.
\newenvironment{myexample}
   { \vspace{20pt}
     \noindent
     \begin{minipage}{\textwidth}    % minipage environment disallows
                 % breaks across pages

     \refstepcounter{examplectr}     % step the counter and cause this
                 % section to be referenced by the
                 % counter ``examplectr''
     (\arabic{examplectr})}%
   { \vspace{20pt}
     \end{minipage}}

\newenvironment{myfnexample}
   { \vspace{2pt}
     \noindent
     \begin{minipage}{\textwidth}    % minipage environment disallows
                 % breaks across pages

     \refstepcounter{fnexamplectr}     % step the counter and cause this
                 % section to be referenced by the
                 % counter ``examplectr''
     (\roman{fnexamplectr})}%
   { \vspace{2pt}c
     \end{minipage}}
    
\newcommand*\circled[1]{\tikz[baseline=(char.base)]{
            \node[shape=circle,draw,inner sep=2pt] (char) {#1};}}

\newcommand{\data}[1]{\textit{#1}}
\newcommand{\nodata}[1]{#1}
\newcommand{\blank}{\rule{1.2em}{0.5pt}}
\newcommand{\pt}[1]{\ensuremath{\mathsf{#1}}}
\newcommand{\ptv}[1]{\ensuremath{\textsf{\textsl{#1}}}}
\newcommand{\sv}[1]{\ensuremath{\mathcal{#1}}}

\newcommand{\sX}{\sv{X}}
\newcommand{\sF}{\sv{F}}
\newcommand{\sG}{\sv{G}}
\newcommand{\greekp}{\upvarphi}
\newcommand{\greekr}{\uprho}
\newcommand{\greeks}{\upsigma}
\newcommand{\MultiLine}[1]{\renewcommand{\arraystretch}{.8}%
\ensuremath{\begin{array}[b]{l} #1 \end{array}}}
\newcommand{\LexEnt}[3]{#1; \ensuremath{#2}; \syncat{#3}}
\newcommand{\BobsO}{\circ}
\newcommand{\HypSpace}{\hskip-.8ex}
\newcommand{\RaiseHeight}{\raisebox{2.2ex}}
\newcommand{\ThreeColHyp}[1]{\RaiseHeight{\Bigg[}\HypSpace#1\HypSpace\RaiseHeight{\Bigg]}}
\newcommand{\ptfont}[1]{\texttt{#1}}
                                % Question
\newcommand{\grey}[1]{\colorbox{mycolor}{#1}}
\definecolor{mycolor}{gray}{0.8}

\newcommand{\gap}{\longrule}
\newcommand{\gp}{\gap}
\newcommand{\vs}{\raisebox{.05em}{\ensuremath{\,\upharpoonright}}}

\newcommand{\E}{ε}

\newcommand{\EBob}[1]{\textsl{#1}}

\newcommand{\B}{\textbf}
\newcommand{\f}{{\color{green}f}}  % Question what does f do? It does not have any output in the
                                % original PDF
%\newcommand{\Lemma}{{\color{pink}Lemma}}
\newcommand{\Lemma}{\ensuremath{\vdots\hskip.5cm\vdots}\noLine}

%\newcommand{\calP}{{\color{pink}calP}} % Sebastian
\newcommand{\calP}{\ensuremath{\mathcal{P}}}


\newcommand{\maru}[1]{\ooalign{\hfil#1\/\hfil\crcr
      \raise.05ex\hbox{\LARGE\mathhexbox20D}}}


%\newcommand{\sem}[2][M\!,g]{\mbox{$[\![ \mathrm{#2} ]\!]^{#1}$}}
\newcommand{\sem}{\ensuremath}

%
\newcommand{\trns}[1]{\textbf{#1}\xspace}
\newcommand{\bs}{{\textbackslash}}
\newcommand{\bsl}{{\bs}}
\newcommand{\fb}[1]{\textsubscript{#1}}
\newcommand{\syncat}[1]{\ensuremath{\mathrm{#1}}}
\newcommand{\term}[1]{\textit{#1}}
\newcommand{\LemmaAlt}{\ensuremath{\vdots\hskip.5cm\vdots}}
