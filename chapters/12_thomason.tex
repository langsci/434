%
% Sally Thomason, version of 7—13—2023
%
\documentclass[output=paper,colorlinks,citecolor=brown]{langscibook}
\IfFileExists{../localcommands.tex}{
   \addbibresource{../localbibliography.bib}
   \usepackage{orcidlink}
\usepackage{tabularx,multicol}
\usepackage{url}
\urlstyle{same}

\usepackage{siunitx}
\sisetup{group-digits = none}

\usepackage{langsci-branding} 
\usepackage{langsci-optional}
\usepackage{langsci-lgr}
\usepackage{langsci-tbls}
\usepackage{langsci-gb4e}

% Müller
\usepackage{tikz-qtree}
\usepackage{hologo}

% 3_pullum.tex
\usepackage{langsci-textipa}

% 8_levine
\usepackage{bm}
\usepackage{umoline}
\usepackage{pifont}
\usepackage{pstricks,pst-node,pst-tree}
\usepackage{ulem}
\usepackage{mathrsfs}
\usepackage{bussproofs}

% 14_kornai
\usepackage[matrix,arrow]{xy}
\usepackage{subcaption}

\usepackage[linguistics, edges]{forest}
\usetikzlibrary{arrows, arrows.meta}

   \SetupAffiliations{output in groups = false,
                   orcid placement = after,
                   separator between two = {\bigskip\\},
                   separator between multiple = {\bigskip\\},
                   separator between final two = {\bigskip\\}
                   }

% ORCIDs in langsci-affiliations 
\definecolor{orcidlogocol}{cmyk}{0,0,0,1}
\RenewDocumentCommand{\LinkToORCIDinAffiliations}{ +m }
  {%
    \,\orcidlink{#1}%
  }

\makeatletter
\let\thetitle\@title
\let\theauthor\@author
\makeatother

% Cite and cross-reference other chapters
\newcommand{\crossrefchaptert}[2][]{\citet*[#1]{chapters/#2}, Chapter~\ref{chap-#2} of this volume} 
\newcommand{\crossrefchapterp}[2][]{(\citealp*[#1]{chapters/#2}, Chapter~\ref{chap-#2} of this volume)}
\newcommand{\crossrefchapteralt}[2][]{\citealt*[#1]{chapters/#2}, Chapter~\ref{chap-#2} of this volume}
\newcommand{\crossrefchapteralp}[2][]{\citealp*[#1]{chapters/#2}, Chapter~\ref{chap-#2} of this volume}

\newcommand{\crossrefcitet}[2][]{\citet*[#1]{chapters/#2}} 
\newcommand{\crossrefcitep}[2][]{\citep*[#1]{chapters/#2}}
\newcommand{\crossrefcitealt}[2][]{\citealt*[#1]{chapters/#2}}
\newcommand{\crossrefcitealp}[2][]{\citealp*[#1]{chapters/#2}}


\newcommand{\sub}[1]{\textsubscript{\scriptsize\textrm{#1}}}
% Müller
\newcommand{\page}{}

\let\citew\citet
\def\underRevision{Revise and resubmit}
\let\textbfemph\emph

%% % taken from https://tex.stackexchange.com/a/95079/18561
\newbox\usefulbox

\makeatletter
\def\getslant #1{\strip@pt\fontdimen1 #1}

\def\skoverline #1{\mathchoice
 {{\setbox\usefulbox=\hbox{$\m@th\displaystyle #1$}%
    \dimen@ \getslant\the\textfont\symletters \ht\usefulbox
    \divide\dimen@ \tw@ 
    \kern\dimen@ 
    \overline{\kern-\dimen@ \box\usefulbox\kern\dimen@ }\kern-\dimen@ }}
 {{\setbox\usefulbox=\hbox{$\m@th\textstyle #1$}%
    \dimen@ \getslant\the\textfont\symletters \ht\usefulbox
    \divide\dimen@ \tw@ 
    \kern\dimen@ 
    \overline{\kern-\dimen@ \box\usefulbox\kern\dimen@ }\kern-\dimen@ }}
 {{\setbox\usefulbox=\hbox{$\m@th\scriptstyle #1$}%
    \dimen@ \getslant\the\scriptfont\symletters \ht\usefulbox
    \divide\dimen@ \tw@ 
    \kern\dimen@ 
    \overline{\kern-\dimen@ \box\usefulbox\kern\dimen@ }\kern-\dimen@ }}
 {{\setbox\usefulbox=\hbox{$\m@th\scriptscriptstyle #1$}%
    \dimen@ \getslant\the\scriptscriptfont\symletters \ht\usefulbox
    \divide\dimen@ \tw@ 
    \kern\dimen@ 
    \overline{\kern-\dimen@ \box\usefulbox\kern\dimen@ }\kern-\dimen@ }}%
 {}}
\makeatother

% 1_intro.tex

% For the block quote:
\definecolor{linequote}{RGB}{224,215,188}
\definecolor{backquote}{RGB}{249,245,233}

\NewDocumentEnvironment{myquote}{ +m }
  {%
    \begin{tblsfilled}{}[black!12]
    #1%
  }
  {\end{tblsfilled}}

% 2_gibson.tex


% Example(s) Environments
% 12pt, No new-lines after example number is printed

\newcounter{examplectr}
\newcounter{fnexamplectr}

% Note: don't use subexamples in footnotes.

% This line is to overcome a bug in cmu-art style: it prints counter
% values to the aux file using \theaux... rather than using \the...
\def\theauxexamplectr{\theexamplectr}

\newcounter{subexamplectr}
\def\theauxsubexamplectr{\thesubexamplectr}
\def\theauxfnexamplectr{\thefnexamplectr}

\renewcommand{\theexamplectr}{\arabic{examplectr}}
% This command causes example numbers to appear without following periods

\renewcommand{\thefnexamplectr}{\roman{fnexamplectr}}
% This command causes example numbers to appear without following periods

\renewcommand{\thesubexamplectr}{\theexamplectr\alph{subexamplectr}}
% This command gives the number of an example and subexample as e.g. 1a, 2b

\newlength{\wdth}
\newcommand{\strike}[1]{\settowidth{\wdth}{#1}\rlap{\rule[.5ex]{\wdth}{1pt}}#1}

\newcommand{\exref}[1]{(\ref{#1})}
% This command puts reference numbers with parentheses
% surrounding them 

% The environment ``examples'' gives a list of examples, one on each line,
% numbered with a lower case alphabetic character
\newenvironment{examples}%
   { \vspace{-\baselineskip}
     \begin{list}%
     \textrm{\alph{subexamplectr}.}%
     {\usecounter{subexamplectr}
     \setlength{\topsep}{-\parskip}
     \setlength{\itemsep}{-2pt}
     \setlength{\leftmargin}{0.5in}
     \setlength{\rightmargin}{0in} } }%
   { \end{list}}

% The environment ``myexample'' outputs an arabic counter ``examplectr''
% surrounded by parentheses.
\newenvironment{myexample}
   { \vspace{20pt}
     \noindent
     \begin{minipage}{\textwidth}    % minipage environment disallows
                 % breaks across pages

     \refstepcounter{examplectr}     % step the counter and cause this
                 % section to be referenced by the
                 % counter ``examplectr''
     (\arabic{examplectr})}%
   { \vspace{20pt}
     \end{minipage}}

\newenvironment{myfnexample}
   { \vspace{2pt}
     \noindent
     \begin{minipage}{\textwidth}    % minipage environment disallows
                 % breaks across pages

     \refstepcounter{fnexamplectr}     % step the counter and cause this
                 % section to be referenced by the
                 % counter ``examplectr''
     (\roman{fnexamplectr})}%
   { \vspace{2pt}c
     \end{minipage}}
    
\newcommand*\circled[1]{\tikz[baseline=(char.base)]{
            \node[shape=circle,draw,inner sep=2pt] (char) {#1};}}

\newcommand{\data}[1]{\textit{#1}}
\newcommand{\nodata}[1]{#1}
\newcommand{\blank}{\rule{1.2em}{0.5pt}}
\newcommand{\pt}[1]{\ensuremath{\mathsf{#1}}}
\newcommand{\ptv}[1]{\ensuremath{\textsf{\textsl{#1}}}}
\newcommand{\sv}[1]{\ensuremath{\mathcal{#1}}}

\newcommand{\sX}{\sv{X}}
\newcommand{\sF}{\sv{F}}
\newcommand{\sG}{\sv{G}}
\newcommand{\greekp}{\upvarphi}
\newcommand{\greekr}{\uprho}
\newcommand{\greeks}{\upsigma}
\newcommand{\MultiLine}[1]{\ensuremath{\begin{array}[b]{@{}l@{}}#1\end{array}}}
\newcommand{\LexEnt}[3]{#1; \ensuremath{#2}; \syncat{#3}}

\newcommand{\LexEntBroken}[3]
  {\Shortstack
      {%
        {#1;} 
        {\ensuremath{#2};} 
        {\syncat{#3}}%
      }%
  }

\newcommand{\grey}[1]{\colorbox{mycolor}{#1}}
\definecolor{mycolor}{gray}{0.8}

\newcommand{\gap}{\longrule}
\newcommand{\gp}{\gap}
\newcommand{\vs}{\raisebox{.05em}{\ensuremath{\,\upharpoonright}}}

\newcommand{\E}{ε}

\newcommand{\EBob}[1]{\textsl{#1}}

\newcommand{\B}{\textbf}
\newcommand{\f}{{\color{green}f}}  % Question what does f do? It does not have any output in the
                                % original PDF
%\newcommand{\Lemma}{{\color{pink}Lemma}}
\newcommand{\Lemma}{\ensuremath{\vdots\hskip.5cm\vdots}\noLine}

%\newcommand{\calP}{{\color{pink}calP}} % Sebastian
\newcommand{\calP}{\ensuremath{\mathcal{P}}}


\newcommand{\maru}[1]{\ooalign{\hfil#1\/\hfil\crcr
      \raise.05ex\hbox{\LARGE\mathhexbox20D}}}


%\newcommand{\sem}[2][M\!,g]{\mbox{$[\![ \mathrm{#2} ]\!]^{#1}$}}
\newcommand{\sem}{\ensuremath}

%
\newcommand{\trns}[1]{\textbf{#1}\xspace}
\newcommand{\bs}{{\textbackslash}}
\newcommand{\bsl}{{\bs}}
\newcommand{\fb}[1]{\textsubscript{#1}}
\newcommand{\syncat}[1]{\ensuremath{\mathrm{#1}}}
\newcommand{\term}[1]{\textit{#1}}
\newcommand{\LemmaAlt}{\ensuremath{\vdots\hskip.5cm\vdots}}
\NewDocumentCommand{\VanLabel}{m}{\MakeUppercase{#1}}

   %% hyphenation points for line breaks
%% Normally, automatic hyphenation in LaTeX is very good
%% If a word is mis-hyphenated, add it to this file
%%
%% add information to TeX file before \begin{document} with:
%% %% hyphenation points for line breaks
%% Normally, automatic hyphenation in LaTeX is very good
%% If a word is mis-hyphenated, add it to this file
%%
%% add information to TeX file before \begin{document} with:
%% %% hyphenation points for line breaks
%% Normally, automatic hyphenation in LaTeX is very good
%% If a word is mis-hyphenated, add it to this file
%%
%% add information to TeX file before \begin{document} with:
%% \include{localhyphenation}
\hyphenation{
    Ber-ti-net-to
    caus-a-tive
    fest-schrift
    Fest-schrift
    Hix-kar-ya-na
    In-do-ne-sian
    mor-pho-phon-o-log-i-cal
    Mo-se-tén
    par-a-digm
    phra-ses
    Que-chua
}

\hyphenation{
    Ber-ti-net-to
    caus-a-tive
    fest-schrift
    Fest-schrift
    Hix-kar-ya-na
    In-do-ne-sian
    mor-pho-phon-o-log-i-cal
    Mo-se-tén
    par-a-digm
    phra-ses
    Que-chua
}

\hyphenation{
    Ber-ti-net-to
    caus-a-tive
    fest-schrift
    Fest-schrift
    Hix-kar-ya-na
    In-do-ne-sian
    mor-pho-phon-o-log-i-cal
    Mo-se-tén
    par-a-digm
    phra-ses
    Que-chua
}

   \boolfalse{bookcompile}
   \togglepaper[23]%%chapternumber
}{}


%\title{Transitivity in S\'eli\v{s}-Ql'isp\'e}
\title{Transitivity in Séliš-Ql’ispé}
\author{Sarah G. Thomason\affiliation{University of Michigan} and
       Daniel Everett\affiliation{Bentley University}}
\abstract{S\'eli\v{s}-Ql'isp\'e, an Interior Salishan language spoken in
 northwestern Montana, has a verbal system that seems at first glance
 to distinguish transitive constructions from intransitive ones in a
 quite straightforward way: transitive verbs have a transitive suffix
 and a characteristic set of subject and object markers, while
 intransitive verbs have no transitive suffix or object markers and
 have a completely different set of subject markers.  In addition,
 the two constructions differ systematically in their marking of
 adjunct (or argument) noun phrases.  Initial appearances are
 deceiving, however.  It turns out that morphologically intransitive
 verbs can take object noun phrases, and that certain transitive
 constructions, most notably monotransitive continuatives, lack part
 of the transitive morphology.  The goal of this paper is to explore
 the morphosyntactic means by which different degrees of transitivity
 are signalled in S\'eli\v{s}-Ql'isp\'e, and to propose an analysis
 that pulls apparently disparate facts together in a unified way.}


\begin{document}
\SetupAffiliations{output in groups = true,
                       separator between two = {~\&~},
                       separator between multiple = {,~},
                       separator between final two = {~\&~}
 }
\maketitle

{\renewcommand{\thefootnote}{}
\footnotetext{Thomason is most
   grateful to elders and members of the S\'eli\v{s}-Ql'isp\'e
   Culture Committee of St. Ignatius, Montana, for permission to
   publish this paper, which is an extensively revised version of
   \citew{S.Thomason&Everett:1993}.  Besides examining written and
   audio materials prepared by the Culture Committee, she has worked
   extensively with many elders: {\dag}Louis Adams, {\dag}Clara
   Bourdon, {\dag}Alice Camel, {\dag}Joe Cullooyah, {\dag}Mike
   Durglo, Sr., {\dag}Joe Eneas, {\dag}Dorothy Felsman,
   {\dag}Margaret Finley, Sophie Haynes, {\dag}Dolly Linsebigler,
   {\dag}Felicite ``Jim'' Sapiel McDonald, {\dag}Agnes Pokerjim Paul,
   {\dag}John Peter Paul, {\dag}Noel Pichette, {\dag}Josephine
   Quequesah, Stephen Small Salmon, Shirley Trahan, {\dag}Eneas
   Vanderburg, {\dag}Joseph Vanderburg, Lucy Vanderburg, {\dag}Janie
   Waubansee, {\dag}Harriet Whitworth, and {\dag}Clarence Woodcock.
   Thomason is also very grateful to S\'eli\v{s}-Ql'isp\'e Culture
   Committee staff who have helped facilitate her meetings with the
   elders over many years: the three Directors, {\dag}Clarence
 Woodcock, {\dag}Tony Incashola, Sr., and Sadie Peone, Office Manager
 {\dag}Gloria Whitworth, and Longhouse caretaker Richard Alexander.
 Both authors acknowledge the many curriculum developers, language
 teachers, and students who, for the past twenty years and counting,
 have been building on the earlier work of the Culture Committee to
 develop a more comprehensive language curriculum and to mount a
 heroic effort to reclaim this critically endangered language.}}
\setcounter{footnote}{0}


% FOR REVISION, CHECK: Shapard, Jeffrey. 1980. Interior Salishan
% (Di)transitive systems.  ICSNL ?15?:; Available  on line via UBC (see Mylib).
% CHECK ALL KROEBER 1991 REFERENCES!  ONE OR MORE STILL NEED TO ==> 1999!
% FIX HEADER NUMBERS!

\section{Introduction}
\label{thomason_section_1}
  Not surprisingly, Salishan languages show both similarities and
  differences in their morphosyntactic patterns relating to
  transitivity.  Many or most of these patterns have been described,
  of course, but as far as we know, comprehensive discussions of the
  patterns we focus on in this paper are still rather rare for a
  Southern Interior Salishan language.\footnote{See e.g.
  \citew{Mattina:1982}, \citew{Kinkade:1981}, \citew{Carlson:1980},
  \citew{Doak:1997} for descriptions of
  the morphology of transitive verbs in Colville-Okanagan,
  Moses-Columbia, Spokane, and Coeur d'Alene, respectively, and
  \citew{Kroeber:1999} for insightful comments on parts of the transitivity
  systems in various Salishan languages.  Relevant analyses of parts
  of transitivity systems are found in e.g. \citew{L.Thomason:1994},
  \citew{Mattina:1994}, \citew{Mattina:2004}, \citew{Dilts:2006},
  \citew{Gerdts&Hukari:1998}, \citew{Sobolak:2020}.}
  There is some overlap between several of these analyses and ours,
  but none of them makes the same claims we do.  Since
  S\'eli\v{s}-Ql'isp\'e transitivity differs from that of other
  Southern Interior languages in certain respects, a description of
  this system should be of interest to Salishanists.\footnote{The
  language called S\'eli\v{s}-Ql'isp\'e today is primarily a merging
  of Bitterroot Salish (formerly known as Flathead), spoken by people
  whose homeland was the Bitterroot valley south of Missoula, MT,
  with Ql'isp\'e (formerly known as Upper Pend d'Oreille), as spoken
  by people whose homeland was the Jocko River area north of
  Missoula.  Both tribes now live on the Flathead Reservation north
  of Missoula.  S\'eli\v{s}-Ql'isp\'e belongs to a dialect complex
  that also includes Spokane and Kalispel; these dialects comprise a
  single language, but there is no language name that covers all
  three.  The data in this paper comes from Thomason's field notes,
  from materials compiled by the Flathead Culture Committee (now
  renamed as the S\'eli\v{s}-Ql'isp\'e Culture Committee), and from
  the thousand-page 19th-century Jesuit dictionary of the language
  \citew{Mengarinietal.:1877}.}  More generally, the
  S\'eli\v{s}-Ql'isp\'e system is of potential interest to
  theoreticians concerned with types and degrees of transitivity,
  because of the wide variety of constructions -- some of them
  rather unusual -- in which transitivity plays a role.  Our account
  is strictly synchronic and specific to S\'eli\v{s}-Ql'isp\'e; we
  have not carried out any systematic study of the diachronic sources
  of the current structures, or any systematic comparison with
  partially cognate structures in other Salishan languages.


The bulk of this paper consists of a description of nine relevant
constructions (Sections~\ref{thomason_section_2.1}--\ref{thomason_section_2.9}): 
ordinary transitives; ditransitives;
unsuffixed intransitives; intransitives with the
\textsc{antipassive} suffix -(\emph{\'e})\emph{m} (often called
``middle'' in the Salishan literature); transitives with the
\textsc{backgrounded agent} suffix -(\emph{\'e})\emph{m} (often
analyzed as ``passive'' and/or ``indefinite agent'' in the
literature); \textsc{transitive continuatives} in -\emph{m};
\textsc{derived transitives} in -\emph{m\'i}(\emph{n});
intransitives with lexical suffixes; and transitives
detransitivized by the reflexive suffix -\emph{c\'ut}.  These nine
constructions do not exhaust the list of relevant patterns; we have
not yet explored all the constructions that have some connection
with transitivity.  We omit a few detransitivizing constructions,
notably the reciprocal, because they behave basically like
reflexive forms with respect to transitivity.  We also omit
discussion of the so-called ``intransitive reflexives''.  A more
significant omission is the lack of any specific consideration of
interactions between control and transitivity (see
e.g. \citealt{Thompson:1985}; we have as yet too little information
on control features in S\'eli\v{s}-Ql'isp\'e to comment on them
here).  Another major transitivity-related topic that is largely
omitted from our account is the patterning of the various
constructions in discourse.  We will mention interactions between
discourse and transitivity occasionally, but we have not yet
studied enough textual material to draw many systematic conclusions
in this domain.


  After presenting the data, we will discuss ways in which the
  various constructions reflect differing degrees of transitivity,
  and we will offer preliminary suggestions for an overall treatment
  of these differences (\sectref{thomason_section_3}).  We adopt, with modifications, the
  common view of transitivity in which the prototypical transitive
  construction involves a completed transfer of action from a
  definite agent to a definite patient (see
  e.g. \citealt{Hopper&Thompson:1980}).  Some modification of this view is
  necessary for
  S\'eli\v{s}-Ql'isp\'e because here the two main variables that
  correlate with transitivity alternations turn out to be \textsc{aspect} and \textsc{focus on the agent} vs. \textsc{focus on the
    patient}.  Definiteness per se is not as important a variable in
  S\'eli\v{s}-Ql'isp\'e as it is said to be in some other Salishan
  languages, though it does play a role in determining the use of two
  non-prototypical constructions, the antipassive and the
  backgrounded agent, and it plays a minor role in the marking of
  patient noun phrases in ordinary transitive sentences.  As we will
  show in the following descriptions, the ordinary transitive
  represents the prototypical transitive type in
  S\'eli\v{s}-Ql'isp\'e, while other transitive-related forms deviate
  from the prototypical model in various ways.  Although we will not
  explore them in any detail in this paper, the S\'eli\v{s}-Ql'isp\'e
  facts have interesting implications for theories of transitivity
  and for the concept of the morpheme.

  Our primary goal is to understand the interactions between the
  morphology and the sentence-level syntax of the relevant
  constructions.  A secondary goal, one that we can only sketch in
  this paper, is to establish the circumstances under which the
  different constructions are used.  One significant departure from
  most previous analyses of these phenomena in Salishan languages is
  our proposal that three of the constructions contain a suffix
  -(\emph{\'e})\emph{m} which has the effect of reducing transitivity
  in a stem to which it is added -- either reducing transitivity in
  (paradoxically) a morphologically intransitive bivalent stem
  (antipassive) or reducing transitivity in a morphologically
  transitive bivalent stem (backgrounded agent, continuative aspect).
  (See \sectref{thomason_section_2.4} for a brief explanation of valency in
  S\'eli\v{s}-Ql'isp\'e.) That is, we will argue that, for
  S\'eli\v{s}-Ql'isp\'e, it is reasonable to treat all these
  occurrences of -(\emph{\'e})\emph{m} in transitive-related
  constructions in a unified way.  The construction in which
  S\'eli\v{s}-Ql'isp\'e seems to differ most sharply from other
  Salishan languages is the transitive continuative; here our account
  diverges from previous analyses of this language, notably those of
  \citet{Kroeber:1999} and \citet{Vogt:1940}, in
  that we analyze these
  forms as transitives, not intransitives (\sectref{thomason_section_2.7}).

\section{Nine relevant construction types}
\label{thomason_section_2}
In its basic morphological patterns, S\'eli\v{s}-Ql'isp\'e appears to
make a straightforward distinction between transitive and intransitive
predicates.\footnote{In this paper we will use the terms `verb' and
`predicate' interchangeably, and we will also talk about `nouns' and
`noun phrases'.  We use this terminology for convenience; we do not
mean to take a firm position on the question of whether
S\'eli\v{s}-Ql'isp\'e and other Salishan languages have a clear
lexical distinction between nouns and verbs (see e.g. \citealt{Kinkade:1983} and
\citealt{vanEijk&Hess:1986} for discussion of this issue).}  First- and
second\hyp person subjects of intransitive verbs are proclitic particles
that appear at the left edge of the verb complex, and third-person
intransitive verbs have no overt subject marking; by contrast, the
morphological transitive apparatus appears at the right edge of the
verb complex in the order \textsc{tr}-o-s -- that is, first a
transitive suffix, then an object marker, and finally the transitive
subject.

There are three exceptions to this basic transitive pattern.  First,
the sole \textsc{1sg} object marker is a proclitic.  Second, all \textsc{1pl} forms have
a proclitic component \emph{qe/{q\textsuperscript w}o}, which in
transitive constructions occurs in conjunction with a \textsc{1pl} suffix in the
usual subject or object suffix position.  And third, transitive
continuative predicates have completely different sets of subject and
object markers; these will be discussed in \sectref{thomason_section_2.6}.  Third-person
objects, like third-person intransitive subjects, have no overt
marking.\footnote{This pattern naturally leads some specialists to
posit split ergativity in various Salishan languages.  We do not
follow their lead, but we will not address the question in this
paper.}  Except for the \textsc{1pl} proclitic, non-tr.cont predicates in
the basic system are divided cleanly into transitive and intransitive
forms according to their pronominal markers.

\subsection{Ordinary transitive verbs}  %2.1
\label{thomason_section_2.1}

  Ordinary transitive verbs, illustrated in examples (\ref{ex-thomason-1}--\ref{ex-thomason-6}), are aspectually
  noncontinuative.  They consist of a transitive stem to which a
  transitive (+ control) suffix, either -\emph{nt} or -\emph{st}, is
  added.\footnote{These two suffixes differ functionally in some
  Salishan languages, such that the former is noncausative and the
  latter causative.  Semantically causative verbs usually have
  -\emph{st} in S\'eli\v{s}-Ql'isp\'e, but some verbs with this
  suffix are not causative, and in fact we have not found a
  systematic functional difference between the two suffixes in the
  basic transitive system (although the data in
  \citew{Mengarinietal.:1877} indicates that \emph{-st} is used
  consistently for habitual actions). In accordance with the Salishan
  literature more generally, as in
  e.g. \citew[23--24]{Mattina&Montler:1990}, we consider these
  two suffixes to be transitivizers.}  All of the transitive stems in
  examples (\ref{ex-thomason-1}--\ref{ex-thomason-6}) are bare roots, with the exception of \REF{ex-thomason-6}.  Example \REF{ex-thomason-6} consists
  of a root \emph{p\'uk\textsuperscript w} `spill, pour round
  objects' preceded by two prefixes and followed by a lexical suffix
  =\emph{\'us} `fire, face'.


  Note, crucially, the marking of full-word agents and patients in \REF{ex-thomason-2}
  and \REF{ex-thomason-3}: patients are marked optionally by the subordinator
  \emph{{\textltilde}u}, and agents are marked obligatorily by the
  oblique particle \emph{t}.\footnote{We will not consider in this
  paper the question of the status of full words other than the main
  predicate (typically the first word) in the S\'eli\v{s}-Ql'isp\'e
  sentence.  In particular, we do not address the issue of adjunct
  vs. argument status for noun phrases that are translated in English
  as agents and patients.  It is clear that some noninitial full
  words are adjuncts, and some of these adjuncts are regularly marked
  by optional \emph{{\textltilde}u}.  Moreover, the oblique marker
  \emph{t} is attached to words other than agent noun phrases (NPs),
  e.g.  time adverbials.  These facts do not necessarily mean that
  the NPs under consideration here are \textbf{not} arguments of the
  verb; still, their syntactic behavior does resemble the behavior of
  full words that are certainly not arguments.  In any case, the
  status of the ``agent'' and ``patient'' NPs is not crucial for our
  present purposes.  For convenience, and without prejudice, we will
  refer to them simply as agents (or subjects) and patients (or
  objects).  } \citet[52--53]{Kroeber:1999} observes that Colville-Okanagan,
  Kalispel [including S\'eli\v{s}-Ql'isp\'e], and Coeur d'Alene are
  unique in Salish in making this distinction between the case
  marking of transitive agent noun phrases (NPs) and that of patient
  NPs, and that this distinction is obligatory in Kalispel only.  We
  have found no exceptions to the case marking of full-word agents of
  transitive verbs.  We do have example sentences in which an
  indefinite patient NP is marked by \emph{t} rather than by \emph{{\textltilde}u}, but since most patients of transitive verbs,
  whether definite or indefinite, are marked instead by optional \emph{{\textltilde}u}, we treat the \emph{t}-marked patients as
  nondistinctive variants (and see \sectref{thomason_section_3} for some discussion of the
  implications of the \emph{t} marking of indefinite patients).  The
  important point about the case marking of NPs in simple transitive
  constructions is that the patient NP is most intimately linked to
  the verb, as shown by its lack of obligatory case marking; the
  agent NP, by contrast, must be set off from the verb complex by the
  oblique particle.\footnote{The grammatical terminology used in
  this paper is loosely based on that of \citet{Carlson:1972}, with
  modifications as needed.  Like Mattina (e.g. \citealt{Mattina:1987}) and other
  Salishanists, we distinguish grammatical suffixes from lexical
  suffixes by using different boundary symbols, a hyphen preceding a
  grammatical suffix and an equals sign preceding a lexical
  suffix -- and similarly for the few lexical prefixes, e.g. \emph{pu\textglotstop}= `spouse' in example \REF{ex-thomason-20}.}


\ea
\label{ex-thomason-1}
\glll P\'ulstx\textsuperscript w.\\
      p\'uls-st-0-\'ex\textsuperscript w\\
      kill-\textsc{tr}-3.\textsc{obj}-\textsc{2sg}.\textsc{tr.sbj}\\
\glt `you killed him.'
\ex
\label{ex-thomason-2}
\v{C}{\textltilde}pnt\'es {\textltilde}u n\textltilde\'amqe t \v{C}on\'i.\\
\gll \v{c}\textltilde\'ip-nt-0-\'es  {\textltilde}u n\textltilde\'amqe t \v{C}on\'i\\
     hunt-\textsc{tr}-3.\textsc{obj}-3.\textsc{tr.sbj} 2ndary black.bear \textsc{obl} Johnny\\
\glt `Johnny hunted a/the black bear.'
\ex 
\label{ex-thomason-3}
{K'\textsuperscript w}e{\textglotstop}nt\'en {\textltilde}u n\textltilde\'amqe.\\
\gll {k'\textsuperscript w}e\textglotstop-nt-0-\'en             {\textltilde}u  n\textltilde\'amqe\\
     bite-\textsc{tr}-3.\textsc{obj}-\textsc{1sg}.\textsc{tr.sbj} 2ndary          black.bear\\
\glt `I bit the black bear.'
\ex
\label{ex-thomason-4}
{Q\textsuperscript w}o w\'i\v{c}tx\textsuperscript w.\\
\gll {q\textsuperscript w}o w\'i\v{c}-st-\'ex\textsuperscript w\\
     \textsc{1sg}.\textsc{obj}       see-\textsc{tr}-\textsc{2sg}.\textsc{tr.sbj}\\  
\glt `You saw me.'
\ex 
\label{ex-thomason-5}
W\'i\v{c}tmn.\\
\gll w\'i\v{c}-st-\'um-\'en\\
     see-\textsc{tr}-\textsc{2sg}.\textsc{obj}-\textsc{1sg}.\textsc{tr.sbj}\\
\glt `I saw you.'
\ex
\label{ex-thomason-6}
E\textltilde\v{c}pq\textsuperscript w\'osntx\textsuperscript w.\\
\gll e\textltilde-\v{c}-p\'uk\textsuperscript w=\'us-nt-0-\'ex\textsuperscript w\\
     back/again-\textsc{loc:}on-pour.round.objects=fire-\textsc{tr}-3.\textsc{obj}-\textsc{2sg}.\textsc{tr.sbj}\\  
\glt `You pour(ed) them on the fire again.'
\z

\subsection{Ditransitives}   %2.2
\label{thomason_section_2.2}

Examples (\ref{ex-thomason-7}--\ref{ex-thomason-10}) illustrate the second relevant construction type,
noncontinuative ditransitive verbs.  These differ from simple
transitives in that they have a \textsc{relational} (+ control) suffix,
either -\emph{{\textltilde}t} or -\emph{\v{s}\'it}, in place of a
transitive (+ control) suffix -\emph{nt} or -\emph{st}.  The two
relational suffixes differ semantically -- -\emph{\v{s}\'it} is a
benefactive suffix, as in (\ref{ex-thomason-7}--\ref{ex-thomason-9}) (assuming that the recipient wanted a
cat!), while -\emph{{\textltilde}t} has a neutral or negative
connotation, as in \REF{ex-thomason-10} (see \citealt{Carlson:1980} for discussion) -- but they
are often used interchangeably.  Examples (\ref{ex-thomason-7}--\ref{ex-thomason-10}) are all formed to bare
roots, \emph{{x\textsuperscript w}\'ic'} `give' and \emph{m\'aw'}
`break, destroy'.


  It is rare for all three NPs to appear together in a ditransitive
  construction, but when they do appear, as in \REF{ex-thomason-7}, \emph{{\textltilde}u} optionally marks the recipient of the action and
  \emph{t} obligatorily marks the patient, the ``direct''
  object.\footnote{It is in a sense misleading to specify \emph{{\textltilde}u} as marking one object in a transitive
  construction, because this particle also occurs sometimes before
  the oblique marker \emph{t}, as well as before certain subordinate
  clauses and other adjuncts.  But the particle is especially
  frequent with an object NP, and in any case the point is that the
  main object of a verb is normally the only NP in a transitive
  construction that may be preceded by this particle alone, whether
  the main object is the so-called direct object of a monotransitive
  verb or the so-called indirect object of a ditransitive verb.  }
  The agent NP is obligatorily case-marked as an oblique, either by
  the simple oblique marker \emph{t}, as in \REF{ex-thomason-7}, or by the preposition
  \emph{tl'} `from'.  The general pattern resembles that of the
  monotransitives: one NP, in this case the recipient, is closely
  tied to the verb and has no obligatory overt case marking; the
  other NPs are obligatorily set off by oblique markers.
  Predictably, when the recipient is expressed by a pronominal (as in
  \ref{ex-thomason-8}--\ref{ex-thomason-10}), the usual object pronominal form is used.  There is,
  moreover, some variation in the case marking of the patient NP in
  ditransitive constructions when the recipient is a pronoun rather
  than a full-word NP: in this case the patient NP sometimes appears
  with zero case marking, as in \REF{ex-thomason-10}, `He wrecked my car' (but this
  does not seem to be possible with the verb `give').  The general
  rule still holds -- at most one full-word NP is nonoblique, i.e.
  lacking overt case marking -- but the zero-marked position may be
  filled by a full-word patient NP when there is no full-word
  recipient NP.  Note that the verb codes directly for only two
  pronominal arguments; the third, usually the recipient of the
  action, is indicated only by the relational suffix.


  There may be some dialect difference between S\'eli\v{s}-Ql'isp\'e
  and Spokane in the case marking of patient NPs in ditransitive
  constructions: according to \citet[24]{Carlson:1980}, in Spokane the
  marking described here is valid only for ditransitives with the
  relational suffix \emph{-\v{s}\'i}; for ditransitives with
  relational \emph{-{\textltilde}}, Spokane marks the recipient NP
  with a preposition and the patient (``direct object'') with
  optional \emph{{\textltilde}u}.  In S\'eli\v{s}-Ql'isp\'e, the
  normal case marking is the same with both relational suffixes.\largerpage[-1]


\ea 
\label{ex-thomason-7}
X\textsuperscript w\'ic'\v{s}ts {\textltilde}u Mal\'i t p\'us tl' \v{C}on\'i. \\
\gll x\textsuperscript w\'ic'-\v{s}\'it-0-\'es {\textltilde}u Mal\'i t p\'us tl' \v{C}on\'i\\
     give-\textsc{rel.tr}-3.\textsc{obj}-3.\textsc{tr.sbj} 2ndary Mary \textsc{  obl} cat from Johnny\\
\glt `Johnny gave Mary a cat.' \\
\pagebreak
\ex X\textsuperscript w\'ic'\v{s}tmn t p\'us.  \\
\label{ex-thomason-8}
\gll x\textsuperscript w\'ic'-\v{s}\'it-\'um-\'en t p\'us\\
give-\textsc{rel.tr}-\textsc{2sg}.\textsc{obj}-\textsc{1sg}.\textsc{tr.sbj} \textsc{obl} cat\\
\glt `I gave you a cat.'
\ex 
\label{ex-thomason-9}
{K\textsuperscript w}u x\textsuperscript w\'ic'\v{s}tx\textsuperscript w t  p\'us. \\
 \gll {k\textsuperscript w}u x\textsuperscript
w\'ic'-\v{s}\'it-\'ex\textsuperscript w t p\'us\\
\textsc{1sg}.\textsc{obj} give-\textsc{rel.tr}-\textsc{2sg}.\textsc{tr.sbj} \textsc{obl} cat \\
 \glt  `You gave me a cat.'\\
 \ex 
\label{ex-thomason-10}
{K\textsuperscript w}u maw'{\textltilde}ts inp'ip'\'uy\v{s}n. \\
\gll {k\textsuperscript w}u m\'aw-{\textltilde}t-\'es in-p'uy-p'\'uy=\v{s}n\\
     \textsc{1sg}.\textsc{obj} break-\textsc{rel.tr}-3.\textsc{tr.sbj} \textsc{1sg}.\textsc{poss}-\textsc{pl}.-wrinkle=foot \\
\glt `He wrecked my car.' (`He wrecked me my car.')
\z

\subsection{Plain intransitive verbs}  %2.3
\label{thomason_section_2.3}

Plain intransitive verbs, illustrated in examples (\ref{ex-thomason-11}--\ref{ex-thomason-13}), stand in sharp
contrast to simple monotransitive and ditransitive constructions.
First- and second-per\-son subject pronominals are proclitics;
third-person intransitive subjects are not overtly marked.  Full-word
subject NPs pattern exactly like a definite main object of a
transitive verb: they lack obligatory case marking, being marked, if at
all, by the optional particle \emph{{\textltilde}u}.  Unlike indefinite
objects of transitive verbs, full-word subjects of intransitive verbs
never take the oblique marker \emph{t}.  Simple intransitives do not,
of course, have a transitive suffix.  (Some complex intransitive
constructions do have a transitive suffix, but it is always followed
by a detransitivizing suffix; see, for instance, reflexives, as
discussed in \sectref{thomason_section_2.6} and illustrated in examples 
(\ref{ex-thomason-41}--\ref{ex-thomason-44}) below.)\largerpage[-1]

\ea
\label{ex-thomason-11}
K\textsuperscript w \textglotstop\'im'\v{s}.  \\
\gll k\textsuperscript w \textglotstop\'im'\v{s}\\
\textsc{2sg}.\textsc{intr.sbj} move(camp)\\  
\glt `You moved.'
\ex 
\label{ex-thomason-12}
\v{C}n {q\textsuperscript w}oy\'ulex\textsuperscript w.  \\
\gll \v{c}n q\textsuperscript w\'ey=\'ulex\textsuperscript w\\
\textsc{1sg}.\textsc{intr.sbj} be.rich=land\\  
\glt `I am rich.'
\pagebreak
\ex 
\label{ex-thomason-13}
{\textglotstop}ocq\'e{\textglotstop} ({\textltilde}u)  \v{C}on\'i. \\
 \gll 0 {\textglotstop}ocq\'e\textglotstop ({\textltilde}u) \v{C}on\'i\\
3.\textsc{intr.sbj}  go.out 2ndary Johnny\\
 \glt `Johnny went out.'
 \z


\subsection{Intransitives with \textsc{antipassive} -(\emph{\'e})\emph{m}}  %2.4
\label{thomason_section_2.4}

So far the constructions we have discussed are morphosyntactically and
semantically straightforward: the morphology and syntax of the
transitive constructions reflect prototypical transitive semantics,
with completed transfer of action from a definite agent to a (usually)
definite patient, and the plain intransitives lack any such transfer.
(The semantic patterns are not, of course, completely transparent
throughout the language; as in all languages, the general semantic
categories leak.)  With the antipassive construction, illustrated in
examples (\ref{ex-thomason-14}--\ref{ex-thomason-17}), we see more complicated relations between morphosyntax and
semantics.  The form we call antipassive (a term used by, among
others, \citet[31]{Kroeber:1999}, \citet{Darnell:1990}, \citet{Gerdts:1993}, and,
with reservations, \citet[102]{Thompson&Thompson:1992} is often
called `middle' in the Salishan literature, and \citet[158]{Newman:1980}
posits a Proto-Salish suffix *-\emph{m} `middle'.  If this suffix has a
genuinely middle function in some other Salishan languages, with
action that reflects back on and/or affects the verb's subject, then
S\'eli\v{s}-Ql'isp\'e has diverged from the rest of the family in this
respect.  The primary grammatical function of the suffix in
S\'eli\v{s}-Ql'isp\'e is to force an active interpretation of an
intransitive bivalent verb.  Perhaps its most notable function in
discourse is to highlight a switch from one agent to another (see the
discussion in \sectref{thomason_section_3} below).  The antipassive also serves as the usual
citation form for bivalent verbs.  For example, if one asks an elder
what the word for `scrape' is, the answer will most likely be an
antipassive \emph{\textglotstop\'a{\d{x}\textsuperscript w}m} `he
scrapes something'.  It might perhaps be \emph{\textglotstop\'a{\d{x}\textsuperscript w}is} (= \emph{\textglotstop\'a{\d{x}\textsuperscript w}-nt-\'es}) `he scrapes it',
but it will never be a suffixless \emph{es\textglotstop\'a\d{x}\textsuperscript w} `it is scraped'.

At this point we need to introduce the topic of valency in
S\'eli\v{s}-Ql'isp\'e because of its crucial interactions with
transitivity, especially in this construction.  All roots in this
language are intransitive, but they fall into three valency
classes. \textsc{Monovalent} verbs, e.g. \emph{x\textsuperscript w\'uy}
`go', have one lexically specified argument -- an actor, an
experiencer, or some other semantic category, but not a patient; and
\textsc{bivalent} verbs, e.g. \emph{w\'i\v{c}} `see, find', have two
lexically specified arguments, an agent and a patient.  Monovalent
verbs are agent-oriented; bivalent verbs are patient-oriented.  The
third and smallest root class, \textsc{ambi-valent}, comprises
agent\hyp oriented verbs with two lexically specified arguments, agent and
patient; an example is is \emph{\textglotstop\'i{\textltilde}n} `eat'.
Ambi-valent verbs do not differ significantly from bivalent verbs with
respect to transitivity interactions, so we will ignore them in the
rest of this discussion.  The largest root class by far, at least for
action verbs, is the bivalent class.


A monovalent verb that occurs alone or with just an aspect affix has
an active (or at least a non-passive) reading, e.g. \emph{\v{c}n
 x\textsuperscript w\'uy} `I go, I went' and \emph{\v{c}n
 \d{x}\textsuperscript w\'ey-t} `I'm lazy, I have no energy'.  By
contrast, a bivalent verb that occurs with just an aspect affix has a
stative passive reading, e.g. \emph{\v{c}n es-w\'i\v{c}} `I am seen'.
This is the most salient diagnostic for identifying a root as
monovalent or bivalent; see \citew{S.Thomasonetal.:1994} for other diagnostics
and further discussion, including reasons for not classifying bivalent
roots as unaccusatives as some other Salishanists have done
(e.g. \citealt{Gerdts:1991}).


The S\'eli\v{s}-Ql'isp\'e antipassive fits the standard definition of
antipassive by promoting the subject of a bivalent verb to agent and,
in effect, backgrounding or eliminating the lexically specified
patient that is part of a bivalent verb's argument structure.  The
stem to which the antipassive suffix is added may be either a simple
root, as in (\ref{ex-thomason-14}--\ref{ex-thomason-16}), 
or a root with one or more affixes, as in \REF{ex-thomason-17}; but
the antipassive suffix is never added to a stem transitivized by the
derived transitive suffix -\emph{m\'i}(\emph{n}), a fact that is
relevant to our overall analysis of these two patterns (see \sectref{thomason_section_3}
below).


It is crucial to our analysis that stems with the antipassive
suffix -- unlike intransitive stems with no \emph{-(\'e)m} -- have two
arguments in their syntactic frame.  That is, the antipassive suffix
adds a second syntactic argument, in spite of the fact that it is
morphologically intransitive.  In addition to the syntactic behavior
of antipassives (see below), further evidence for this interpretation
lies in the fact that an antipassive added directly to a monovalent
stem (that is, with no intervening derived transitive suffix -\emph{m\'i}(\emph{n})) produces a bivalent causative stem, with a second
agent.  So, for instance, \emph{{k\textsuperscript w} x\textsuperscript
 w\'uym} means `you cause someone to go'.  Compare the corresponding
morphological transitive, also with a causative reading and without a
\textsc{der.trans} suffix, \emph{x\textsuperscript
 w\'uy-nt-x\textsuperscript w} `you cause her to go'.  Compare also the
same verb root in a non-causative transitive bivalent construction
with the \textsc{der.trans} suffix, e.g. \emph{\v{c}-x\textsuperscript
 w\'uy-m-nt-x\textsuperscript w} `you visited him' (with the locative
prefix \emph{\v{c}}- `to').


Bivalent verbs appear most frequently in discourse in straightforward
transitive constructions, as in examples (\ref{ex-thomason-1}--\ref{ex-thomason-6}) 
above.  Antipassives, by
contrast, are bivalent but morphologically intransitive.  Accordingly,
the subject pronominals for antipassives are the usual intransitive
proclitic particles, and full-word subject NPs are marked by optional
\emph{{\textltilde}u}, as in example \REF{ex-thomason-16} (for which the free translation is
`I skinned it and my wife sliced it', `it' being deer meat).  But
since, unlike monovalent intransitives, these are semantically
transitive constructions, they also have notional objects -- usually
indefinite but sometimes definite, as in example \REF{ex-thomason-16}.  When the object is
overtly expressed, as in \REF{ex-thomason-14} and \REF{ex-thomason-15}, it is marked obligatorily by the
oblique proclitic \emph{t}.  Antipassives thus have the opposite
marking from ordinary transitive constructions with two arguments: in
antipassives a subject NP is marked by optional \emph{{\textltilde}u}
and an object NP by obligatory \emph{t}, while in transitive
constructions a subject NP is marked by obligatory \emph{t} and an
object NP by optional \emph{{\textltilde}u}.\footnote{Formally marked
antipassives are not the only verbs that participate in this pattern;
ambi-valent stems also do so.  An example is the ambi-valent verb \emph{\v{c}{\textltilde}\'ip} `hunt', as in intransitive \emph{\v{c}n
 \v{c}{\textltilde}\'ip t n{\textltilde}\'amqe} vs. transitive \emph{\v{c}{\textltilde}pnt\'en {\textltilde}u n{\textltilde}\'amqe}, both
meaning `I hunt(ed) black bear'.} Note that in \REF{ex-thomason-17} the oblique marker
precedes an instrument NP, not an object NP; this common type of
adjunct phrase underlines our point that the oblique marker indicates
a phrase that is less closely linked to the verb, and thus arguably
less important, than the ``main'' NP.

\ea 
\label{ex-thomason-14}
\v{C}n {k'\textsuperscript w}e\textglotstop\'em t n\textltilde\'amqe. \\
\gll \v{c}n {k'\textsuperscript w}e{\textglotstop}-(\'e)m t
n\textltilde\'amqe\\
\textsc{1sg}.\textsc{intr.sbj} bite-\textsc{antip} \textsc{obl}  black.bear \\
\glt `I bit a  black bear.'
\ex 
\label{ex-thomason-15}
Ha {k\textsuperscript w} w\'i\v{c}m t sm\d{x}\'e? \\
\gll ha k\textsuperscript w w\'i\v{c}-(\'e)m t sm\d{x}\'e\\ 
Q \textsc{2sg}.\textsc{intr.sbj} see-\textsc{antip} \textsc{obl} grizzly.bear\\
\glt `Did you see a grizzly bear?'
\ex
\label{ex-thomason-16}
{\d{X}\textsuperscript w}cnt\'en u t'\'elm {\textltilde}u inn\'o{\d{x}\textsuperscript w}n{\d{x}\textsuperscript w}.  \\
\gll {\d{x}\textsuperscript w}\'ic-nt-0-\'en u t'\'el-(\'e)m
{\textltilde}u in-n\'o{\d{x}\textsuperscript w}-n{\d{x}\textsuperscript w} \\
skin-\textsc{tr}-3.\textsc{obj}-1\textsc{sg}.\textsc{trans}.\textsc{subj} and slice-\textsc{antip} 2ndary \textsc{1sg}.\textsc{poss}-wife-\textsc{1sg}.\textsc{tr.sbj} \\
\glt `I skinned it and my wife sliced it.'
\ex 
\label{ex-thomason-17}
Mk\textsuperscript w \v{c}\d{x}\textsuperscript w\'eycpm t an{\textltilde}n'\'i. \\
 \gll m k{\textsuperscript w} \v{c}-\d{x}{\textsuperscript w}\'eyc-p-(\'e)m t
an-\textltilde-n'\'i\v{c}'\\ 
\textsc{fut} \textsc{2sg}.\textsc{intr.sbj} \textsc{loc:}to-cut.off.hair-\textsc{inch}-\textsc{antip} \textsc{obl} \textsc{2sg}.\textsc{poss}-\textsc{dim}-cut\\
 \glt `You'll cut off the hair with your knife.'
 \z


Although the antipassive is clearly an intransitive construction, its
two-ar\-gu\-ment syntactic frame and its ability to include a syntactic
patient as well as an agent places it on the transitivity gradient: it
is less transitive than a prototypical transitive construction (as in
\sectref{thomason_section_2.1}) semantically because it typically has an indefinite patient,
and in any case its agent is the main focus.  As we will argue in \sectref{thomason_section_3}, this transitivity-reducing function unites the antipassive suffix
with the backgrounded agent suffix (\sectref{thomason_section_2.5}) and the transitive
continuative suffix (\sectref{thomason_section_2.6}).

\subsection{Backgrounded agent constructions} %2.5
\label{thomason_section_2.5}

The next construction in our list is the one typically characterized
in the Salishan literature as a passive or indefinite-agent
construction.  In many or most other Salishan languages this
characterization is accurate (see e.g. \citealt[25--28]{Kroeber:1999} for
discussion and for a characterization of this construction as Agent
Demotion), but the cognate construction in S\'eli\v{s}-Ql'isp\'e is
clearly active and transitive, and quite often the agent is definite
(though indefinite agents are much more common with this
construction).  Morphosyntactically, the construction differs from
ordinary transitives only in that the backgrounded agent suffix -(\emph{\'e})\emph{m} (or its allomorph -\emph{t}; see below) replaces the
usual transitive subject suffix.  That is, the suffix -(\emph{\'e})\emph{m} is the subject suffix, and it is always preceded by a transitive
suffix -- {-\emph{nt}}, -\emph{st}, or relational (ditransitive) -\emph{{\textltilde}t} or -\emph{\v{s}it}.\footnote{But see \sectref{thomason_section_2.6} below:
there is some evidence that the transitive continuative suffix \emph{-m} is sometimes followed by an unstressed backgrounded agent suffix
(\emph{-\'e})\emph{m}, and that the two contiguous \emph{m} suffixes
coalesce phonologically into a single [m].  This hypothesized
coalescence distinguishes this pair of \emph{m} suffixes from the
suffix sequence unstressed \emph{-m}(\emph{\'in}) `derived transitive' +
\emph{m} `transitive continuative', in which both \emph{m}'s are always
pronounced, either as a long [m:] or (more often) as [m{\textschwa}m].
} The case marking of subject and object NPs, as in \REF{ex-thomason-18}, \REF{ex-thomason-19}, \REF{ex-thomason-21}, and
\REF{ex-thomason-22}, is identical to that of any other transitive sentence, with the
object optionally marked by \emph{{\textltilde}u} and the subject
obligatorily marked by \emph{t}.  (Example \REF{ex-thomason-18} means `One-Night told
Qeyqey\v{s}i', not vice versa.  Zero marking of the patient in \REF{ex-thomason-21} is
permitted because the recipient, the ``indirect object'', is
pronominal.)

\begin{sloppypar}
No overt object suffixes occur between the transitive suffix and the
backgrounded agent suffix allomorph -(\emph{\'e})\emph{m}.  This means
that only the \textsc{1sg} object proclitic \emph{{k\textsuperscript w}u} and a
zero-marked third-person object can occur with this allomorph.
However, these forms are functionally identical to and in
complementary distribution with backgrounded agent forms with the
suffix allomorph -\emph{t}, which does permit a preceding overt object
marker; examples are in \REF{ex-thomason-22} and \REF{ex-thomason-23}.  That is, the forms with -\emph{t}
occur always and only with \textsc{1pl} and 2nd-person objects.  We therefore
treat this -\emph{t} as an allomorph of the backgrounded agent suffix,
an analysis also found elsewhere in the literature (see
e.g. \citealt[25--28]{Kroeber:1999}, with reference to Interior
Salish generally, and \citealt[63]{Thompson&Thompson:1992}, with specific
reference to Thompson; Kroeber considers the construction to be a true
passive, while Thompson \&\ Thompson treat it as an indefinite-agent
construction).
\end{sloppypar}

\ea 
\label{ex-thomason-18}
C\'untm Qeyqey\v{s}\'i t N{k'\textsuperscript w}us{k\textsuperscript w}{k\textsuperscript w}\textglotstop\'e, $\ldots$ \\
 \gll c\'un-nt-0-\'em Qeyqey\v{s}\'i t
n{k'\textsuperscript w}\'u\textglotstop-s-{k\textsuperscript w}-{k\textsuperscript w}\textglotstop\'e(c) \\
say-\textsc{tr}-3.\textsc{obj}-\textsc{backgrnd.ag} Qeyqey\v{s}\'i \textsc{obl} one-\textsc{nom}-night-night \\
 \glt `One-Night told Qeyqey\v{s}i,  \ldots '
 \ex 
\label{ex-thomason-19}
{K\textsuperscript w}u {k'\textsuperscript w}e{\textglotstop}nt\'em t sm\d{x}\'e. \\
\gll {k\textsuperscript w}u {k'\textsuperscript
 w}e{\textglotstop}-nt-\'em t sm\d{x}\'e\\
\textsc{1sg}.\textsc{obj}bite-\textsc{tr}-\textsc{backgrnd.ag} \textsc{obl} grizzly.bear\\
\glt `The grizzly bear bit me.'
\ex
\label{ex-thomason-20}
Espu{\textglotstop}p\'ulstm.  \\
\gll es-pu\textglotstop=p\'uls-st-0-\'em \\
\textsc{aspect}-spouse=kill-\textsc{tr}-3.\textsc{obj}-\textsc{backgrnd.ag}\\
\glt `Her husband got killed.' (=`Someone killed her spouse')
\ex 
\label{ex-thomason-21}
{K\textsuperscript w}u p\'ul{\textltilde}tm is{k\textsuperscript w}\'is{k\textsuperscript w}s. \\
\gll  {k\textsuperscript w}u p\'uls-{\textltilde}t-0-\'em
in-s{k\textsuperscript w}\'is{k\textsuperscript w}s\\
\textsc{\textsc{1sg}.obj} kill-\textsc{rel.tr}-3.\textsc{obj}-\textsc{backgrnd.ag} \textsc{1sg}.\textsc{poss}-chicken\\
\glt `My chickens got killed.' (= `Someone killed me my chickens.')
\ex 
\label{ex-thomason-22}
Qe n\v{c}cn\'i\v{c}i{\textltilde}lt t s\v{c}q'i{q'\textsuperscript w}\'e.  \\
 \gll qe n-\v{c}ic(n)=\'i\v{c}n-{\textltilde}ul-l-t t
s-\v{c}-q'i-{q'\textsuperscript w}\'ay\\
\textsc{1pl} \textsc{loc}:in-arrive=back-\textsc{tr}-
\textsc{obl} \textsc{nom}--\textsc{loc}:to\textasciitilde{}pl-black\\
\textsc{1pl}.\textsc{obj}-\textsc{backgrnd.ag}\\
 \glt `The blackfeet caught up with
us.'\footnote{The transitive suffix -\emph{{\textltilde}\'ul} in this
sentence is an allomorph of the standard transitive suffixes,
occurring always and only with a \textsc{1pl} or \textsc{2pl} object.}
\ex 
\label{ex-thomason-23}
N'em {\textltilde}c'nc\'it. \\
\gll n'em \textltilde\'ic'-nt-s\'i-t\\ 
\textsc{fut} whip-\textsc{tr}-\textsc{2sg}.\textsc{obj-backgrnd.ag}\\
\glt  `You'll be whipped.'
\z


Although this construction is an ordinary active transitive in
S\'eli\v{s}-Ql'isp\'e, it does have one prototypical functional
characteristic of passives (see e.g. \citealt{Shibatani:1985}): as our label
suggests, it indicates backgrounding of the agent.  This is not a new
observation; \citet[58]{Thompson&Thompson:1992}, for instance, interpret
the Thompson cognate construction similarly, remarking that the
indefinite-subject forms (as they analyze them) `serve to shift focus
from the transitive subject to the object'.  At least one discourse
function of the S\'eli\v{s}-Ql'isp\'e construction appears to be
identical to that of Moses-Columbia, as described in \citew{Kinkade:1989}.
Kinkade argues that the construction serves to track participants,
being used to indicate a less prominent agent throughout a discourse.
The Qeyqey\v{s}i story from which example  \REF{ex-thomason-18} is taken illustrates this
feature very neatly.  The overall discourse environment is
story-telling about a prominent tribal member named Qeyqey\v{s}i,
specifically about his wild younger days when he and his friend
One-Night repeatedly got into trouble.  The particular story in which
\REF{ex-thomason-18} occurs follows one in which Qeyqey\v{s}i himself is the major
character; but in this later story, One-Night is the main instigator
of the prank.  In spite of One-Night's greater prominence in this
context, however, transitive verbs referring to his actions
consistently have the backgrounded agent suffix throughout this rather
lengthy story.  The reason surely is that Qeyqey\v{s}i himself is the
primary character in the overall discourse environment, so that
One-Night's agent role is consistently downplayed by means of the
backgrounded agent construction.  Although this story sequence is an
especially clear example of the participant-tracking function
described by Kinkade, the same phenomenon recurs in
S\'eli\v{s}-Ql'isp\'e texts.  The point that needs to be underlined
here is that there is nothing indefinite about One-Night.  It is of
course true that indefinite agents are typically less prominent than
other participants in discourse, e.g.  when the patient is 1st or 2nd
person \REF{ex-thomason-19}, \REF{ex-thomason-22} or when no particular agent is specified \REF{ex-thomason-20}, \REF{ex-thomason-21}; but
the common factor in these (and other) examples is backgrounding of
the agent, not indefiniteness.

The backgrounded agent construction, like the antipassive and the
transitive continuative construction, falls below a prototypical
transitive construction on a gradient scale of transitivity, thanks to
its typically (though not universally) indefinite agent.  We will
return to this topic in \sectref{thomason_section_3}.

\subsection{Transitive continuatives}  %2.6.
\label{thomason_section_2.6}

The transitive continuative construction is the most interesting of
all the S\'eli\v{s}-Ql'isp\'e transitivity-related constructions,
thanks to the complications it presents.  We will describe and
illustrate the construction before discussing the complications.

The transitive continuative suffix -\emph{m} does not co-occur with
the transitive apparatus in monotransitive forms (e.g.  examples \ref{ex-thomason-24}--\ref{ex-thomason-28}).
Instead, it occurs after a bivalent stem -- i.e. after a bivalent root
\REF{ex-thomason-24}, \REF{ex-thomason-28} or a bivalent stem produced by the derived transitive suffix
-\emph{m\'i}(\emph{n}) (\ref{ex-thomason-25}--\ref{ex-thomason-27}) -- and it is never preceded (or
followed) by an object suffix or an agent suffix.  A transitive
continuative verb is always preceded by a prefix that varies between
the shapes \emph{es-} and \emph{s-}, and this variation is
problematic.  Treating this prefix as basically \emph{es-} would mean
that the prefix is an `actual' aspect marker and would make transitive
continuatives parallel to the regular intransitive continuative
monovalent form \emph{es}-STEM-\emph{m\'i}, as in
e.g. \emph{es-lap-m\'i} `s/he's traveling by boat'\footnote{In spite
of their shared /m/ segments, the intransitive continuative suffix and
the transitive continuative suffix are not morphologically related
either synchronically or diachronically. The parallelism consists of
the aspect prefix combined with a continuative suffix.  } -- an
appealing symmetry.  But analyzing the basic form of the prefix as the
nominalizer \emph{s-} is also appealing because, as we will see, the
subjects of transitive continuative verbs are expressed by possessive
affixes, which (elsewhere) can only be added to nominal stems.  If the
basic form of the prefix is \emph{s-}, however, it is difficult to
account for the variation phonologically: the \emph{s-} variant occurs
most often after a particle or full word ending in a vowel, and a
regular rule deletes a word-initial vowel in this context but there
is no phonological rule that inserts a prothetic vowel \emph{e} before
a word-initial \emph{s}.  As others have pointed out (notably
\citealt{Kroeber:1999}), there has been some conflation of these two
prefixes in S\'eli\v{s}-Ql'isp\'e, and this partial conflation might
have contributed to the relatively recent development of the
construction in its current form.  We provisionally analyze the basic
form of the prefix as \emph{es-} and consider it to have properties of
both the `actual' aspect prefix and the nominalizer.

The agent of a transitive continuative verb is expressed by a
possessive affix -- a prefix (\textsc{1sg}, \textsc{2sg}), a preposed particle (\textsc{1pl}),
or a suffix added after the \textsc{tr.cont} suffix -\emph{m} (\textsc{2pl},
3).  The patient is expressed in two different ways: either it is a
normal object marker (\textsc{1sg}) or it is an intransitive subject particle
(\textsc{2sg}, \textsc{2pl}).  In S\'eli\v{s}-Ql'isp\'e, \textsc{1pl} and third-person patients
provide no evidence for the ``basic'' marking of notional patients in
this construction, because third-person objects and third-person
intransitive subjects are all zero-marked, and the preposed part of
the \textsc{1pl} object is invariant \emph{qe} and thus identical to the \textsc{1pl}
intransitive subject particle.

In ditransitive continuative constructions the transitive suffix does
appear, specifically a relational transitive suffix -\emph{{\textltilde}t} or 
-\emph{\v{s}\'it}, which immediately precedes the \textsc{tr.cont} suffix 
(example \REF{ex-thomason-29}).  Otherwise the ditransitive forms are 
morphologically identical to the monotransitive forms.

Syntactically, the transitive continuative is identical to an ordinary
transitive construction: subject NPs are obligatorily marked by
oblique \emph{t} \REF{ex-thomason-26}, \REF{ex-thomason-28} and object NPs are 
optionally marked by \emph{{\textltilde}u} \REF{ex-thomason-24}, 
(\ref{ex-thomason-27}--\ref{ex-thomason-29}).  (In \REF{ex-thomason-29}, the 
fact that the recipient of the action is a pronominal is what allows the patient 
NP to receive optional \emph{{\textltilde}u} marking.)

\ea 
\label{ex-thomason-24}
Ies{\textcrlambda}'e\textglotstop\'em ulul\'im.  \\
\gll in-es-{\textcrlambda}'e\textglotstop-m ulul\'im \\
\textsc{1sg}.\textsc{poss}-\textsc{actual/nom}-look.for-\textsc{tr.cont} money \\
\glt `I'm looking for money.'
\ex 
\label{ex-thomason-25}
{K\textsuperscript w}u es\'ay'mtmms. \\
\gll  
{k\textsuperscript w}u es-\textrevglotstop\'ay'm-t-m\'i(n)-m-s \\
\textsc{1sg}.\textsc{obj} \textsc{actual/nom}-angry-\textsc{statv-der.tr-tr.cont}-3.\textsc{poss} \\
\glt `He's mad at me.'
\ex 
\label{ex-thomason-26}
Pes\'ay'mtmms t \v{C}on\'i.  \\
\gll p es-\textrevglotstop\'ay'm-t-m\'i(n)-m-s t \v{C}on\'i \\
\textsc{2pl}.\textsc{intr.sbj} \textsc{actual/nom}-angry-\textsc{statv-}
 \textsc{obl} Johnny \\
\textsc{der.tr-tr.cont}-3.\textsc{poss}
\glt `Johnny's mad at you guys.' 
\ex 
\label{ex-thomason-27}
Ies\textrevglotstop\'ay'mtmm {\textltilde}u \v{C}on\'i. \\
\gll in-es-\textrevglotstop\'ay'm-t-m\'i(n)-m {\textltilde}u \v{C}on\'i \\ 
\textsc{1sg}.\textsc{poss}-\textsc{actual/nom}-angry-\textsc{statv-der.tr-tr.cont} 2ndary Johnny \\
\glt `I'm mad at Johnny.'
\ex 
\label{ex-thomason-28}
Es{x\textsuperscript w}\'epms {\textltilde}u s\'ic'm t is{q\textsuperscript w}s\'e{\textglotstop}. \\
 \gll es-{x\textsuperscript w}\'ep-m-s {\textltilde}u s\'ic'm t
   in-s-{q\textsuperscript w}s\'e{\textglotstop} \\
\textsc{actual/nom}-spread- 2ndary
blanket \textsc{obl} \textsc{1sg}.\textsc{poss}-\textsc{nom}-son\\
\textsc{tr.cont}-3.\textsc{poss} \\
 \glt `My son is spreading the blanket.'
\ex 
\label{ex-thomason-29}
{K\textsuperscript w} ies\v{c}\'e{x\textsuperscript w}{\textltilde}tm {\textltilde}u as\'ic'm. \\
   \gll {k\textsuperscript w} in-es-\v{c}'\'e{x\textsuperscript
 w}-{\textltilde}t-m {\textltilde}u an-s\'ic'm \\
\textsc{2sg}.\textsc{intr.sbj} \textsc{1sg}.\textsc{poss}-\textsc{actual/nom} 2ndary \textsc{2sg}.\textsc{poss}-blanket-dry-\textsc{rel.tr-tr.cont} \\
   \glt `I'm drying your blanket for you.'
\z

As mentioned above, our analysis of this continuative construction as
transitive differs from previous analyses, in particular those of
\citet{Kroeber:1991} and \citet{Vogt:1940}, who treat the construction
as intransitive.  In later work Kroeber distinguishes the history of
the construction (definitely nominalized and intransitive) from its
synchronic status in S\'eli\v{s}-Ql'isp\'e, which may indeed be
transitive \citep[357]{Kroeber:1999}.  According to Kroeber, a transitive verb “is
one that contains a Transitive or Ditransitive suffix, or at least
inflects with Object pronominals.  All other predicates are
intransitive” \citep[29]{Kroeber:1991}.  This definition straightforwardly excludes
monotransitive continuative constructions from the transitive
category; however, as we will try to show, the definition is too
restrictive, in part because it does not take relevant syntactic
patterns into account.


The construction has two properties that suggest intransitivity.
First, and most obviously, it lacks the transitive suffix in the
monotransitive form; and second, the use of the \textsc{2sg} and \textsc{2pl}
intransitive subject proclitics for second-person notional patients
makes the construction look intransitive.  A form like
\emph{{k\textsuperscript w} isw\'i\v{c}m} (\emph{{k\textsuperscript w}
in-es-w\'i\v{c}-m}, lit. \textsc{2sg}.\textsc{intr.sbj} \textsc{1sg}.\textsc{poss}-\textsc{actual/nom}-see-\textsc{tr.cont}) would be glossed by Vogt and
Kroeber as `you are my seeing', whereas for us the translation is
literally as well as freely `I am seeing you'.

\begin{sloppypar}
The construction has two properties that are compatible with
either a transitive or an intransitive analysis: the ambiguity in the
marking of \textsc{1pl} and third-person notional patients, already mentioned
above, and the optional \emph{{\textltilde}u} case marking on the notional
object NP, as in \REF{ex-thomason-24} and \REF{ex-thomason-26}.  
The sentences in \REF{ex-thomason-24} and \REF{ex-thomason-27} could be
glossed either as ordinary transitives, `I'm looking for money' and
`I'm mad at Johnny', respectively, or literally in the Vogt\slash Kroeber
style, `money is my looking-for' and `Johnny is my being mad at'.
\end{sloppypar}

However, the construction has four properties that make it look
transitive.  First, the \textsc{1sg} object proclitic appears where the
notional object is `me'.  Second, a full-word agent NP is obligatorily
marked by oblique \emph{t}, as expected in a transitive but
emphatically not in an intransitive sentence; this marking in turn
shows that the apparently ambiguous optional \emph{{\textltilde}u}
marking on the other possible full-word NP must indicate the object,
not an intransitive subject, because notional full-word subject NPs
are invariably marked by \emph{t} in this construction.  Third, as
noted above, the transitive suffix appears obligatorily in two-goal
transitive continuative forms (e.g. \REF{ex-thomason-29}).  And fourth, given the
crosslinguistic links between possessive and agentive marking, the
expression of the notional subject by possessive pronominals suggests
that they are, indeed, agents (compare, for instance, English \emph{I
wrecked his car} and \emph{my wrecking of his car}).  This property is
suggestive, but it cannot be considered diagnostic for the analysis of
any particular language.  A possibly relevant fifth property is the
fact that the transitive continuative suffix -\emph{m} occurs
immediately after the derived transitive suffix
-\emph{m\'i}(\emph{n}), which otherwise precedes only a transitive
suffix.  (However, this property might perhaps be dismissed on the
ground that the co-occurrence of these two suffixes could mean simply
that what we're calling the transitive continuative suffix has a
detransitivizing effect, an analysis that would fit with the
Vogt/Kroeber interpretation.)


The two intransitive-like properties, the absence of a transitive
suffix in monotransitive continuative forms and the use of 2nd-person
intransitive subject proclitics, are balanced by two of the
transitive-like properties, the presence of a transitive suffix in
ditransitive continuative forms and the use of the \textsc{1sg} object
proclitic.  The crosslinguistic tendency toward linking of transitive
agents and possessives does not provide solid evidence for our
analysis.  That leaves us with one property which, in our view, argues
strongly for a transitive analysis, namely, the case marking of
full-word subject and object NPs.  As we have seen in 
Sections~\ref{thomason_section_2.1}--\ref{thomason_section_2.5} (and
will see below in 
Sections~\ref{thomason_section_2.7}--\ref{thomason_section_2.9}), 
this case marking is consistent
throughout the language in identifying subject NPs and object NPs in
both transitive and intransitive constructions.  If the transitive
continuative construction is not to be viewed as transitive, there is
an inconsistency in case marking NPs that has no explanation.


By contrast, we do have an explanation for at least one of the two
intransitive-like properties of this construction -- the use of \textsc{2sg} and
\textsc{2pl} intransitive proclitic pronominals to indicate the notional
patient.  Since, in monotransitive continuative forms, there is no
transitive suffix, there is nothing to attach an object suffix to.  In
fact, the \textsc{tr.cont} suffix replaces the entire transitive
apparatus, so there is also no agent suffix, which must follow an
object suffix in a normal transitive form.  Obviously, then, patients
must be expressed by some other means.  This presents no problem for
the \textsc{1sg} object, which is a proclitic already, or for a third-person
object, which has no overt marking, or for a \textsc{1pl} object, which in
ordinary transitive forms has both proclitic and suffixed components
(so that the proclitic can take over the entire function).  But how
are 2nd-person objects to be expressed, given that the usual suffix
position is not available?  There are three other sets of person
markers: transitive subject suffixes, possessive affixes, and
intransitive subject proclitics.  The transitive subject suffixes are
unavailable for the morphological reason just given, even aside from
the poor notional fit.  The possessive affixes are unavailable because
that set is already in use for the subject of the verb.  This leaves
only the intransitive subject proclitics, if a 2nd-person marker is to
be used at all; and so that is what we find.  Notice, moreover, that
an analogous explanation will not account for the use of the \textsc{1sg}
object proclitic if the construction is viewed as intransitive: since
both the \textsc{1sg} object and the \textsc{1sg} intransitive subject are proclitics,
both are available -- in contrast to the second person, where only the
intransitive subject particles can be pressed into service as object
markers in this construction.\footnote{Tony Mattina (p.c. 1992) has
suggested a different analysis of the transitive continuative forms,
as a `genitive' construction.  He points out that in
S\'eli\v{s}-Ql'isp\'e, as in Colville-Okanagan, there are
constructions like (S\'eli\v{s}-Ql'isp\'e) \emph{{k\textsuperscript w}
in\d{x}m\'en\v{c}} `I like you' and \emph{{k\textsuperscript w}u
an\d{x}m\'en\v{c}} `you like me', with pronominal marking identical to
that of the transitive continuative forms -- possessive affix for
notional agent, \textsc{2sg} proclitic intransitive subject vs. \textsc{1sg} object for
notional object -- but with no actual/nominalizer prefix and no
-\emph{m} suffix.  \citet[32]{Vogt:1940} also comments on links
between transitive continuative verbs and possessed nouns.  Exploring
these connections is beyond the scope of the present paper, but they
obviously must be considered in a complete analysis of the phenomena.
We do not believe, however, that they will require a change in our
analysis of the transitive continuative construction.  }

The other intransitive-like property of the transitive continuative,
the lack of a transitive suffix in monotransitive forms, is what it
appears to be: a signal that the forms in question are less transitive
than their completive counterparts.  On our analysis, adding the
transitive continuative suffix does not change the valency of the
transitive stem, but it does reduce the degree of transitivity
associated with the action.  Unlike the antipassive and the
backgrounded agent construction, the transitive continuative
construction does not serve to focus attention on the agent or the
patient.  Instead, its role is to signal a change in aspect, a change
that reduces the transitive force of the verb in that the action is
not completely transferred from the agent to the patient.  We will
discuss this further in \sectref{thomason_section_3}.

\subsection{Derived transitives}  %2.7.
\label{thomason_section_2.7}

The derived transitive suffix -\emph{m\'i}(\emph{n}) is added to a
monovalent stem, either a root or a derived stem.  Its function is to
add an argument, a patient, to the verb's argument structure; it thus
effects a change in valency.  This suffix is followed immediately by
the transitive apparatus -- transitive suffix, object suffix (if
any), and subject suffix -- or by the antipassive, a detransitivizing
suffix, or the transitive continuative suffix (see \sectref{thomason_section_2.6} above).
In other words, this suffix creates a bivalent stem.  It presents no
particular morphosyntactic complications: complete transitive verbs
that contain this suffix are straightforward transitive forms, both
morphologically and syntactically, and detransitivized verbs that
contain this suffix follow the usual patterns for such constructions
(see e.g. Sections~\ref{thomason_section_2.8}--\ref{thomason_section_2.9} below, 
especially example \REF{ex-thomason-44}, in which the derived
transitive suffix occurs twice).  \citet[430]{Mattina:1982} observes that
Colville-Okanagan stems derived with the cognate suffix never
participate in ditransitive constructions; there is no such
restriction in S\'eli\v{s}-Ql'isp\'e, as example \REF{ex-thomason-33} illustrates.

\ea
\label{ex-thomason-30}
\v{S}{\textcrlambda}'mst\'ex\textsuperscript w. \\
\gll \v{s}{\textcrlambda}'-m\'i(n)-st-0-\'e{x\textsuperscript w} \\
all.kinds-\textsc{der.tr}-\textsc{tr}-3.\textsc{obj}-\textsc{2sg}.\textsc{tr.sbj} \\
\glt `You get all kinds [of things].'
\ex 
\label{ex-thomason-31}
\v{C}{x\textsuperscript w}\'uymntm {\textltilde}u Mal\'i t \v{C}on\'i. \\
 \gll  \v{c}-{x\textsuperscript w}\'uy-m\'i(n)-nt-0-\'em {\textltilde}u Mal\'i t \v{C}on\'i \\
\textsc{loc}:to-go-\textsc{der.tr-trans}-3.\textsc{obj}-\textsc{backgrnd.ag} 2ndary Mary \textsc{obl} Johnny \\
 \glt `Johnny visited Mary.'
 \ex 
\label{ex-thomason-32}
E{\textltilde}pta{\d{x}\textsuperscript w}mis. \\
\gll e{\textltilde}--pta{\d{x}\textsuperscript w}-m\'i(n)-nt-0-\'es \\ again/back-spit-\textsc{der.tr-trans}-3.\textsc{obj}-3.\textsc{tr.sbj} \\
\glt `He spat it out again.'
\ex 
\label{ex-thomason-33}
Wic\'inm{\textltilde}ts ask\textsuperscript w\'is{k\textsuperscript w}s. \\
\gll w\'i\textglotstop=c\'in-m\'i(n)-{\textltilde}t-0-\'es
 an-s-k\textsuperscript w\'is-{k\textsuperscript w}s \\
finish=mouth-\textsc{der.tr-rel.tr}-3.\textsc{obj}-3.\textsc{tr.sbj}
\textsc{2sg}.\textsc{poss}-\textsc{nom-redup}-chicken \\
\glt `He ate up your chickens.'
\ex 
\label{ex-thomason-34}
{K\textsuperscript w} ya{\textglotstop}m\'im. \\
\gll {k\textsuperscript w} ya\textglotstop-m\'i(n)-\'em \\ 
\textsc{2sg}.\textsc{intr.sbj} gather-\textsc{der.tr-antip} \\    
\glt `You gathered rocks.'
\z

This suffix has been analyzed in various ways in the literature.  It
is not clear to us which, if any, of these interpretations are
incompatible with ours; the apparent divergence may be due in large
part to nonsubstantive terminological differences.  In addition, of
course, the suffix may function in less transparent ways in other
Salishan languages.  We will mention a few representative analyses
here, but will not attempt to sort out the differences in any detail.
Vogt appears to analyze the suffix as a transitivizer \citep[59--60]{Vogt:1940},
though his analysis of it is complicated (and made somewhat unclear)
by his treatment of Kalispel transitive continuative forms as
intransitives (see our example \REF{ex-thomason-25} above for a typical co-occurrence of the
derived transitive and transitive continuative suffixes).  Kinkade
treats the cognate Moses-Columbia suffix as an intransitive suffix,
specifically the middle suffix -\emph{m}; the resulting stem is then
transitivized, in his analysis, by the addition of the causative
suffix \citep[195]{Kinkade:1981}.  Gerdts' (\citeyear{Gerdts:1993}) analysis of the analogous
construction in Halkomelem looks very similar to Kinkade's, except
that his middle category is her antipassive (see e.g. her example (45)).
The Kinkade/Gerdts approach does not at first glance seem well suited
to the S\'eli\v{s}-Ql'isp\'e facts.  The S\'eli\v{s}-Ql'isp\'e
transitive suffixes, including the so-called causative -\emph{st} as
well as -\emph{nt} and the relational ditransitive suffixes, are normally
added to stems that are already bivalent; bivalent roots are
lexically specified, while lexically monovalent roots, together with
stems that are detransitivized by lexical or other detransitivizing
suffixes, normally appear with the transitive apparatus only after the
derived transitive suffix -\emph{m\'i}(\emph{n}) has been added.  (There
is also no obvious preference for -\emph{st} over -\emph{nt} after this
suffix in S\'eli\v{s}-Ql'isp\'e.)


This generalization requires a caveat, however, because there is
evidence in S\'eli\v{s}-Ql'isp\'e that the transitive suffixes can
indeed add a syntactic argument directly to a monovalent verb stem.
As we have seen, the derived transitive suffix followed by a
transitive suffix increases the valency of the stem by adding a
patient to its argument structure.  But a transitive suffix added
directly to a monovalent stem also increases the valency of the stem,
in this case by adding a second agent -- that is, it produces a
causative stem, as we saw in \sectref{thomason_section_2.4}.  As a reminder of that
discussion, compare examples (\ref{ex-thomason-35}--\ref{ex-thomason-37}):

\ea 
\label{ex-thomason-35}
\v{c}n x\textsuperscript w\'uy \\
\gll \v{c}n x\textsuperscript w\'uy \\
\textsc{1sg}.\textsc{intr.sbj} go \\
\glt `I go'
\ex 
\label{ex-thomason-36}
\v{c}x\textsuperscript w\'uymn \\
\gll \v{c}-x\textsuperscript w\'uy-m\'i(n)-nt-\'en \\
\textsc{loc:}to-go-\textsc{der.tr-tr}-\textsc{1sg}.\textsc{tr.sbj} \\
\glt `I visit her' (lit. `I go to her')
\ex 
\label{ex-thomason-37}
x\textsuperscript w\'uyn \\
\gll x\textsuperscript w\'uy-nt-\'en \\
go-\textsc{tr}-\textsc{1sg}.\textsc{tr.sbj} \\
\glt `I make him go'
\z

Example \REF{ex-thomason-35} is a plain intransitive monovalent verb. In \REF{ex-thomason-36} a derived
transitive suffix, and thus a patient, has been added to the verb's
argument structure, producing a bivalent stem; and the further
addition of the transitive suffix -\emph{nt} forms a verb with two
syntactic arguments, an agent and a patient.  Example \REF{ex-thomason-37} contrasts with \REF{ex-thomason-36}
formally in that \REF{ex-thomason-37} lacks the derived transitive suffix; instead, the
addition of the transitive suffix forms a causative verb by adding a
second syntactic agent/actor to the verb.  Moreover, as we also saw in
\sectref{thomason_section_2.4}, a form \emph{\v{c}n xw\'uym} `I make someone go' is also
possible: in this case the antipassive suffix -\emph{\'em} also adds a
second agent to the verb's argument structure and thus produces a
bivalent verb.  Both this antipassive construction and transitive
constructions like \REF{ex-thomason-37} are rare in S\'eli\v{s}-Ql'isp\'e discourse,
unlike derived transitive constructions, which are common.

Note, finally, that the derived transitive suffix may appear either
after a lexical suffix, thus increasing the valency of a monovalent
stem (e.g. example \REF{ex-thomason-33}), or before a lexical suffix, in which case the
potentially transitive stem formed by this suffix loses its syntactic
patient (e.g. example \REF{ex-thomason-39} below).

\subsection{Transitive-prone stems detransitivized by lexical suffixes}  %2.8.
\label{thomason_section_2.8}

The remaining two constructions that we want to illustrate are two
types of verbs in which stems that are usually followed by transitive
suffixes or the antipassive are detransitivized.  This section
concerns the effect of certain lexical suffixes, as in examples (\ref{ex-thomason-38}--\ref{ex-thomason-40}).
These suffixes may be added to bivalent roots, as in \REF{ex-thomason-38}, or to stems
that have had their valency increased by the addition of the derived
transitive suffix, as in \REF{ex-thomason-39} and \REF{ex-thomason-40}.  In other words, a lexical suffix
is added to the stem instead of a more usual transitive suffix.  There
are no morphosyntactic problems here: the derived stems take
intransitive subject particles, as expected in an intransitive
construction; a subject NP is marked by optional \emph{{\textltilde}u}
\REF{ex-thomason-40}; and an object NP is marked by obligatory \emph{t} (\ref{ex-thomason-38}--\ref{ex-thomason-39}).

\ea 
\label{ex-thomason-38}
{K\textsuperscript w} plsq\'e t \v{s}m\'en'. \\
\gll  k\textsuperscript w p\'uls=sq\'e t \v{s}m\'en' \\
\textsc{2sg}.\textsc{intr.sbj} kill=person \textsc{obl} enemy \\
\glt `You killed an enemy.' (lit. `You person-killed an enemy.')
\ex 
\label{ex-thomason-39}
Nt{\d{x}\textsuperscript w}msq\'a t \d{x}{\textcrlambda}'c\'is. \\
\gll n-t\'o{\d{x}\textsuperscript w}-m\'i(n)=sq\'a t
 \d{x}{\textcrlambda}'c\'in-s \\
\textsc{loc}:in-straight-\textsc{der.tr}=domestic.animal \textsc{obl} horse-3.\textsc{poss} \\
\glt `He turned his horse around.'
\ex  
\label{ex-thomason-40}
{\v{C}{x\textsuperscript w}imsq\'e {\textltilde}u Mal\'i.} \\
\gll \v{c}-{x\textsuperscript w}\'uy-m\'i(n)=sq\'e {\textltilde}u Mal\'i \\
\textsc{loc}:to-go-\textsc{der.tr}=person 2ndary Mary \\
\glt `Mary visited someone.'
\z

These constructions resemble antipassives syntactically in that the
lexical suffix does not co-occur directly with transitive suffixes;
instead, it is added either to a lexically bivalent root or to a
derived bivalent stem.  But where the antipassive is formed by a
semantically empty suffix -\emph{\'em}, the constructions of interest
here are formed by a lexical suffix with (often) concrete semantic
content.  A more significant difference between the two construction
types is that a stem modified by a lexical suffix may become
transitive-ready again by the addition of the derived transitive
suffix, as in example \REF{ex-thomason-33} above.  As we have seen, this is not possible
with an antipassive.


A common proposal in the Salishan literature is that verbs like those
in (\ref{ex-thomason-38}--\ref{ex-thomason-40}) contain an incorporated noun -- that is, that the lexical
suffixes are in fact incorporated noun stems.  Such an analysis would
of course account for their monovalent status, and a few of the 100+
lexical suffixes in S\'eli\v{s}-Ql'isp\'e have full-word nominal
counterparts; the lexical suffix -\emph{sq\'e}, for instance, is
obviously related to the noun \emph{sq\'elix\textsuperscript w}
`person, Indian'.  In order not to expand the present paper beyond
reasonable bounds, we will not consider the implications of this
interpretation here, in spite of its (indirect?) relevance to the
general topic of transitivity.

\subsection{Transitive-prone stems detransitivized by the reflexive
 -\emph{c\'ut}}   %2.9.
\label{thomason_section_2.9}

The final construction we will consider is the reflexive in
-\emph{c\'ut}, which -- like reflexives in many other
languages -- detransitivizes the potentially transitive stem to which
it is added.  As with verbs detransitivized by lexical suffixes, these
reflexives are straightforward intransitives: the pronominal subject
is the usual intransitive subject proclitic \REF{ex-thomason-41}, \REF{ex-thomason-43}, and a full-noun
subject NP is marked optionally by \emph{{\textltilde}u} \REF{ex-thomason-42}.  For
obvious semantic reasons, the reflexive takes no overt object NP.  The
reflexive construction differs strikingly from two of the four
transitivity-reducing constructions we saw above: unlike the
antipassive and lexical-suffix constructions, instead of replacing the
usual transitive apparatus, the reflexive suffix is added to it,
immediately after the transitive suffix.\footnote{The backgrounded
agent suffix -\emph{\'em} also follows a transitive suffix, as does
the transitive continuative suffix -\emph{m} if the verb is
ditransitive.}  In other words, the reflexive suffix replaces the
(object and) transitive subject suffix(es).  Like lexical-suffix
constructions, but unlike the antipassive, a reflexive may be
re-transitivized by the addition of the derived transitive suffix, as
in \REF{ex-thomason-44}.

\ea 
\label{ex-thomason-41}
\v{C}n ct'ipmnc\'u tl' es\v{s}\'it'.  \\
\gll \v{c}n c-t'y\'i-p-m\'i(n)-nt-c\'ut tl' es\v{s}\'it' \\
\textsc{1sg}.\textsc{intr.sbj} \textsc{loc}:hither-fall-\textsc{inch-der.tr-tr-refl} from tree \\
\glt `I came down from the tree.'
\ex 
\label{ex-thomason-42}
Qsnc\'u {\textltilde}u \v{C}on\'i. \\
\gll qs-nt-c\'ut {\textltilde}u \v{C}on\'i \\ 
scratch-\textsc{tr-refl} 2ndary Johnny \\
\glt `Johnny scratched himself.'
\ex 
\label{ex-thomason-43}
\v{C}n esplsc\'uti. \\
\gll \v{c}n es-p\'uls-st-c\'ut-m\'i \\
\textsc{1sg}.\textsc{intr.sbj} \textsc{actual}-kill-\textsc{tr-refl-intr.cont} \\
\glt  `I am killing myself.'
\ex 
\label{ex-thomason-44}
{K\textsuperscript w}u \v{c}ta{\d{x}\textsuperscript w}lmnc\'utmntm. \\
\gll {k\textsuperscript w}u \v{c}-ta{\d{x}\textsuperscript
 w}l-m\'i(n)-nt-c\'ut-m\'i(n)-nt-\'em \\ 
\textsc{1sg}.\textsc{obj} \textsc{loc}:to-start-\textsc{der.tr-tr-refl-der.tr-tr-backgrnd.ag} \\
\glt `He came up to me.'
\z


This completes our survey of nine S\'eli\v{s}-Ql'isp\'e constructions
that are relevant to an analysis of the language's transitivity
system.  The next step is to try to pull the various constructions
together into a less fragmented system.

\section{An analysis of transitivity in S\'eli\v{s}-Ql'isp\'e}  %3
\label{thomason_section_3}

In this section we will propose an analysis in which several of the
transitivity-related constructions illustrated in
\sectref{thomason_section_2} fit
together in a coherent overall picture.  We should begin by noting
that plain intransitives -- those without an antipassive suffix or
another suffix that derives an intransitive verb -- fall outside the
transitive system entirely; they are included only to show what a
basic intransitive construction is like, with its subject proclitics
and its full-word subject NP marked only by optional \emph{{\textltilde}u}.


As we said in our introduction, two main variables turn out to
correlate interestingly with transitivity alternations in
S\'eli\v{s}-Ql'isp\'e.  First, there is a systematic morphosyntactic
distinction between semantically transitive constructions with a
\textsc{focus on the agent} and those with a \textsc{focus on the
patient}; and second, \textsc{aspect} plays a role in conditioning
transitivity alternations.  On our analysis, the ordinary
(noncontinuative) transitive construction carries no particular
emphasis on agent or patient, and no special marking of aspect: it is
the neutral transitive construction, and the closest to a prototypical
transitive construction that involves a completed transfer of action
from a definite agent to a definite patient.  The object NP is most
closely linked to the verb, as indicated by its lack of obligatory
case marking; in a ditransitive form, only one object NP, usually the
recipient (the ``indirect object''), may lack case marking.  A
full-word subject NP in a transitive construction is marked
obligatorily by oblique \emph{t}.  This neat picture is complicated by
the influence of definiteness, a feature that often affects
transitivity in other languages (including elsewhere in Salish): an
indefinite patient NP may be marked with oblique \emph{t}.  This
alternative marking, though it is not at all consistent in
S\'eli\v{s}-Ql'isp\'e, indicates in effect a reduction in the
transitive force of the verb -- a deviation from the prototypical
transitive.


The next three constructions are all characterized by a suffix
-(\emph{\'e})\emph{m}.  We propose to treat all three of these
suffixes as a single morpheme -\emph{em} with one general function and
with specific interpretations linked to the various morphological
environments in which it occurs: the antipassive -(\emph{\'e})\emph{m}
occurs in absolute final position, without a preceding transitive
suffix and without an actual aspect/nominalizer prefix plus possessive
agent; the backgrounded agent -\emph{\'em} occurs in absolute final
position after a transitive suffix and without a nominalizing prefix
plus possessive agent; and the transitive continuative -\emph{m}
occurs word-finally except for a possessive agent suffix, and it
always co-occurs with an aspect/nominalizer prefix plus a possessive
agent.  In other words, the three specific functions (designated by
our three labels) of these three -(\emph{\'e})\emph{m}  suffixes are
 predictable from their morphological context within a particular
 verb form; the three allomorphs of our proposed -\emph{em}
 morpheme are in complementary distribution.


The primary function of the proposed -\emph{em} morpheme is to signal
a reduction of transitivity -- a deviation from the prototypical
transitive as represented by the neutral S\'eli\v{s}-Ql'isp\'e
transitive construction.  The -({\em \'e}){\em m} suffixes reflect two
different kinds of deviation from the prototypical transitive: they
indicate either a focus on one of the two main participants in the
action -- i.e. the agent or the patient (or perhaps, in a
ditransitive verb, the recipient) -- or a change in aspect that
affects the transitive force of the verb.  The transitive continuative
is the sole member of the aspect-changing category.  In the focus
category, the participant that is highlighted is predictable from the
morphological context in which the suffix occurs.


The antipassive -(\emph{\'e})\emph{m} emphasizes the agent -- so much
so that it removes the patient argument from the verb's morphology.
The result is that the sole argument in the verb itself is the agent,
though the stem remains bivalent.  The resulting intransitive
construction is partly analogous to transitive stems that are
detransitivized by a lexical suffix; these too highlight the agent and
have no pronominal patient marking in the verb.  The reflexive in
-\emph{c\'ut} also fits here functionally and syntactically, its
formation differing from the other two agent-focusing constructions
only in that it retains the transitive suffix.  Example \REF{ex-thomason-16} illustrates
one use of the agent-highlighting antipassive construction.  This
sentence, which means `I skinned it and my wife sliced it', has an
antipassive (\emph{t'\'elm} `sliced') preceded by an ordinary
transitive verb.  With the second verb comes a change of agent, a
switch that is highlighted by the use of the antipassive.  Note that
Vogt's characterization of what we call the antipassive as occurring
with an indefinite object \citeyearpar[31]{Vogt:1940} would not capture this usage, since the `it' in question refers to the same deer throughout the
sentence; the difference is the switch in agents.  Vogt was partly
right, because antipassive constructions very often do include
indefinite patients, but definiteness is not (in our view) the primary
factor.


In the backgrounded agent construction, the -\emph{\'em} focuses on the
patient.  This is evident, for instance, throughout the particular
Qeyqey\v{s}i tale from which example \REF{ex-thomason-18} is taken: as described above,
Qeyqey\v{s}i is the main character in all the stories about him, even
this one, where his friend One-Night is the instigator of the prank
and the agent of most of the transitive verbs.  Qeyqey\v{s}i's more
prominent overall status is highlighted by the use of the backgrounded
agent construction throughout for all verbs in which One-Night is the
agent.


As mentioned above, the transitive continuative construction does not
participate in the argument-focusing functions of the other two
manifestations of the proposed -\emph{em} morpheme.  Instead, its
role is to signal an aspectual deviation from prototypical
transitivity: this construction reduces transitivity by indicating
that the action is not completely transferred from an agent to a
patient.  The reduced transitivity of this construction is reflected
morphologically in its one clear intransitive-like feature, the lack
of a transitive suffix in monotransitive continuative forms.


All three -(\emph{\'e})\emph{m}  formations, then, can be viewed as
 deviating from a prototypical ordinary transitive to a form that is
 lower on the transitivity gradient~-- either with unbalanced emphasis
 on one participant or with a deviation from the prototypical
 completive aspect.  It is interesting to note that only the
 patient-highlighting formation, the backgrounded agent construction,
 remains straightforwardly transitive morphologically.  By contrast,
 the agent-highlighting antipassive is morphosyntactically
 intransitive, and the transitive continuative construction, though
 transitive, is morphologically peculiar for a transitive verb.


We were initially tempted to combine all four constructions with
\emph{m} suffixes into a single morpheme -- the three just discussed
and also the derived transitive construction in -\emph{m\'i}(\emph{n})
(see \citealt{S.Thomason&Everett:1993}).  The derived transitive
construction might also be viewed as highlighting the patient, since
it adds a patient to the verb's argument structure.  But because it
increases the valency of a verb root or derived stem, it is difficult
to argue that it reduces transitivity; moreover, it is incompatible
with the other three \emph{m} constructions phonologically.  All four
suffixes surface frequently, perhaps most frequently, simply as
-\emph{m}, which is their predictable form unless they are stressed.
But the stressed allomorph -\emph{\'em} of our proposed -\emph{em}
morpheme cannot be reconciled with the stressed allomorph of the
derived transitive.

The idea of combining two or more of these -(\emph{\'e})\emph{m}
 suffixes into one morpheme is of course not new, although our
 particular interpretation and our grouping of all three into a
 single morpheme are, as far as we know, unique.  For instance, some
 authors connect the antipassive and the backgrounded agent suffixes;
 examples are \citet{Kuipers:1967} (Squamish), \citet{Darnell:1990}
 (Squamish, with an analysis that, like ours, involves de-emphasis of
 one argument in each case), and \citet[185]{Gerdts:1989}
 (Halkomelem).  Other authors, e.g. \citet[32]{Vogt:1940}
 (Kalispel), \citet[158--159]{Newman:1980}, and
 \citet[294]{Kroeber:1991}, group the antipassive and the
 transitive continuative suffixes together.  Still others,
 e.g. \citet[105]{Kinkade:1981} (Moses-Columbia), consider the
 antipassive and the derived transitive suffix to be the same.

The remaining two constructions discussed above, the effect of certain
lexical suffixes on transitivity and the detransitivization of stems
by the reflexive suffix -\emph{c\'ut}, are obviously morphologically
distinct from our -\emph{em} morpheme, but they share the function of
reducing the transitivity of stems to which they are added.  These two
constructions therefore contribute to the overall picture of gradient
transitivity in S\'eli\v{s}-Ql'isp\'e.

Our analysis ends here: this is as far as we have proceeded in our
effort to work through the complex S\'eli\v{s}-Ql'isp\'e facts related
to transitivity.  We should close by emphasizing again that our
analysis is necessarily incomplete.  Aside from remaining gaps in our
understanding of the constructions we have already examined, there are
still other constructions that must be studied before we can aim at a
complete analysis of the system.  But we hope to have shown, at least,
that there are interesting interrelationships among
transitivity-related constructions that seem at first glance to be
quite disparate.

\section*{Abbreviations}

Besides the abbreviations from the Leipzig Glossing Conventions, this chapter uses the following
abbreviations:\medskip\\

\noindent\begin{tabularx}{\textwidth}{@{}lQ@{}}
2ndary & `secondary in importance', a complement or subordinate to
  the main predicate \\
\textsc{actual} & actual aspect (as in continuative forms and certain stative forms) \\
\textsc{backgrnd.ag} & backgrounded agent \\
\textsc{cont} & continuative aspect \\
\textsc{der.tr} & derived transitive (a transitivizing suffix) \\
\textsc{dim} & diminutive \\
\textsc{inch} & inchoative \\
\textsc{rel} & relational (indicating that there is a recipient or other ``indirect object'') \\
\textsc{statv} & stative \\
\end{tabularx}


\section*{Acknowledgments}

% Thomason is most
%    grateful to elders and members of the S\'eli\v{s}-Ql'isp\'e
%    Culture Committee of St. Ignatius, Montana, for permission to
%    publish this paper, which is an extensively revised version of
%    \citew{S.Thomason&Everett:1993}.  Besides examining written and
%    audio materials prepared by the Culture Committee, she has worked
%    extensively with many elders: {\dag}Louis Adams, {\dag}Clara
%    Bourdon, {\dag}Alice Camel, {\dag}Joe Cullooyah, {\dag}Mike
%    Durglo, Sr., {\dag}Joe Eneas, {\dag}Dorothy Felsman,
%    {\dag}Margaret Finley, Sophie Haynes, {\dag}Dolly Linsebigler,
%    {\dag}Felicite ``Jim'' Sapiel McDonald, {\dag}Agnes Pokerjim Paul,
%    {\dag}John Peter Paul, {\dag}Noel Pichette, {\dag}Josephine
%    Quequesah, Stephen Small Salmon, Shirley Trahan, {\dag}Eneas
%    Vanderburg, {\dag}Joseph Vanderburg, Lucy Vanderburg, {\dag}Janie
%    Waubansee, {\dag}Harriet Whitworth, and {\dag}Clarence Woodcock.
%    Thomason is also very grateful to S\'eli\v{s}-Ql'isp\'e Culture
%    Committee staff who have helped facilitate her meetings with the
%    elders over many years: the three Directors, {\dag}Clarence
%  Woodcock, {\dag}Tony Incashola, Sr., and Sadie Peone, Office Manager
%  {\dag}Gloria Whitworth, and Longhouse caretaker Richard Alexander.
%  Both authors acknowledge the many curriculum developers, language
%  teachers, and students who, for the past twenty years and counting,
%  have been building on the earlier work of the Culture Committee to
%  develop a more comprehensive language curriculum and to mount a
%  heroic effort to reclaim this critically endangered language.

   We are both also grateful to Tony Woodbury and ✝\,Ken Hale for
   early discussions of some of the data in this paper and for
   offering valuable comments about the analysis; we have made use of
   some of their suggestions in the overall analysis in \sectref{thomason_section_3}.  
   The two authors of this paper have not worked together on this or
   other projects since shortly after the decade when we were
   colleagues at the University of Pittsburgh, but the first author
   views the current version of this paper as arising from the
   wonderfully stimulating intellectual connection we forged during
   that decade.  She wishes to express her gratitude to the second
   author for that connection and for more than thirty-five years of
   friendship.  She is one of the greatest admirers of his stellar
   record of sustained fieldwork and linguistic documentation
   combined with valuable theoretical contributions, and of his
   linguistic insights, his intellectual courage, his remarkable
   ability to meet vituperation with courteous reasoned discussion,
   and his terrific sense of humor.

\printbibliography[heading=subbibliography,notkeyword=this]
\end{document}
