\documentclass[output=paper,colorlinks,citecolor=brown]{langscibook}
\title{Anarchy, power, festschrifts, and universals}
\ChapterDOI{10.5281/zenodo.12665903}
 
\author{Stefan Müller\orcid{0000-0003-4413-5313}\affiliation{Humboldt-Universität zu Berlin}}

%\abstract{This paper discusses Dan Everett's perception in Europe (without the UK), anarchy, and
%  power, explains why this book is the first and last festschrift that has been published by Language
%  Science Press, and suggests the one and only universal.\itdopt{improve}}

\abstract{This paper discusses the concept of anarchy as the absence of power and power misuse by
  one of the most influential anarchists and his followers. I also discuss universals and the case
  of Pirahã. It is argued that there may not be any real non-trivial universals on the sentence
  level, but that there is a strong candidate for a universal on the text level: the festschrift universal. I also explain why
  Dan Everett is the first, last, and hence only person on this planet to get a Language Science
  Press festschrift.}


\IfFileExists{../localcommands.tex}{
   \addbibresource{../localbibliography.bib}
   \usepackage{orcidlink}
\usepackage{tabularx,multicol}
\usepackage{url}
\urlstyle{same}

\usepackage{siunitx}
\sisetup{group-digits = none}

\usepackage{langsci-branding} 
\usepackage{langsci-optional}
\usepackage{langsci-lgr}
\usepackage{langsci-tbls}
\usepackage{langsci-gb4e}

% Müller
\usepackage{tikz-qtree}
\usepackage{hologo}

% 3_pullum.tex
\usepackage{langsci-textipa}

% 8_levine
\usepackage{bm}
\usepackage{umoline}
\usepackage{pifont}
\usepackage{pstricks,pst-node,pst-tree}
\usepackage{ulem}
\usepackage{mathrsfs}
\usepackage{bussproofs}

% 14_kornai
\usepackage[matrix,arrow]{xy}
\usepackage{subcaption}

\usepackage[linguistics, edges]{forest}
\usetikzlibrary{arrows, arrows.meta}

   \SetupAffiliations{output in groups = false,
                   orcid placement = after,
                   separator between two = {\bigskip\\},
                   separator between multiple = {\bigskip\\},
                   separator between final two = {\bigskip\\}
                   }

% ORCIDs in langsci-affiliations 
\definecolor{orcidlogocol}{cmyk}{0,0,0,1}
\RenewDocumentCommand{\LinkToORCIDinAffiliations}{ +m }
  {%
    \,\orcidlink{#1}%
  }

\makeatletter
\let\thetitle\@title
\let\theauthor\@author
\makeatother

% Cite and cross-reference other chapters
\newcommand{\crossrefchaptert}[2][]{\citet*[#1]{chapters/#2}, Chapter~\ref{chap-#2} of this volume} 
\newcommand{\crossrefchapterp}[2][]{(\citealp*[#1]{chapters/#2}, Chapter~\ref{chap-#2} of this volume)}
\newcommand{\crossrefchapteralt}[2][]{\citealt*[#1]{chapters/#2}, Chapter~\ref{chap-#2} of this volume}
\newcommand{\crossrefchapteralp}[2][]{\citealp*[#1]{chapters/#2}, Chapter~\ref{chap-#2} of this volume}

\newcommand{\crossrefcitet}[2][]{\citet*[#1]{chapters/#2}} 
\newcommand{\crossrefcitep}[2][]{\citep*[#1]{chapters/#2}}
\newcommand{\crossrefcitealt}[2][]{\citealt*[#1]{chapters/#2}}
\newcommand{\crossrefcitealp}[2][]{\citealp*[#1]{chapters/#2}}


\newcommand{\sub}[1]{\textsubscript{\scriptsize\textrm{#1}}}
% Müller
\newcommand{\page}{}

\let\citew\citet
\def\underRevision{Revise and resubmit}
\let\textbfemph\emph

%% % taken from https://tex.stackexchange.com/a/95079/18561
\newbox\usefulbox

\makeatletter
\def\getslant #1{\strip@pt\fontdimen1 #1}

\def\skoverline #1{\mathchoice
 {{\setbox\usefulbox=\hbox{$\m@th\displaystyle #1$}%
    \dimen@ \getslant\the\textfont\symletters \ht\usefulbox
    \divide\dimen@ \tw@ 
    \kern\dimen@ 
    \overline{\kern-\dimen@ \box\usefulbox\kern\dimen@ }\kern-\dimen@ }}
 {{\setbox\usefulbox=\hbox{$\m@th\textstyle #1$}%
    \dimen@ \getslant\the\textfont\symletters \ht\usefulbox
    \divide\dimen@ \tw@ 
    \kern\dimen@ 
    \overline{\kern-\dimen@ \box\usefulbox\kern\dimen@ }\kern-\dimen@ }}
 {{\setbox\usefulbox=\hbox{$\m@th\scriptstyle #1$}%
    \dimen@ \getslant\the\scriptfont\symletters \ht\usefulbox
    \divide\dimen@ \tw@ 
    \kern\dimen@ 
    \overline{\kern-\dimen@ \box\usefulbox\kern\dimen@ }\kern-\dimen@ }}
 {{\setbox\usefulbox=\hbox{$\m@th\scriptscriptstyle #1$}%
    \dimen@ \getslant\the\scriptscriptfont\symletters \ht\usefulbox
    \divide\dimen@ \tw@ 
    \kern\dimen@ 
    \overline{\kern-\dimen@ \box\usefulbox\kern\dimen@ }\kern-\dimen@ }}%
 {}}
\makeatother

% 1_intro.tex

% For the block quote:
\definecolor{linequote}{RGB}{224,215,188}
\definecolor{backquote}{RGB}{249,245,233}

\NewDocumentEnvironment{myquote}{ +m }
  {%
    \begin{tblsfilled}{}[black!12]
    #1%
  }
  {\end{tblsfilled}}

% 2_gibson.tex


% Example(s) Environments
% 12pt, No new-lines after example number is printed

\newcounter{examplectr}
\newcounter{fnexamplectr}

% Note: don't use subexamples in footnotes.

% This line is to overcome a bug in cmu-art style: it prints counter
% values to the aux file using \theaux... rather than using \the...
\def\theauxexamplectr{\theexamplectr}

\newcounter{subexamplectr}
\def\theauxsubexamplectr{\thesubexamplectr}
\def\theauxfnexamplectr{\thefnexamplectr}

\renewcommand{\theexamplectr}{\arabic{examplectr}}
% This command causes example numbers to appear without following periods

\renewcommand{\thefnexamplectr}{\roman{fnexamplectr}}
% This command causes example numbers to appear without following periods

\renewcommand{\thesubexamplectr}{\theexamplectr\alph{subexamplectr}}
% This command gives the number of an example and subexample as e.g. 1a, 2b

\newlength{\wdth}
\newcommand{\strike}[1]{\settowidth{\wdth}{#1}\rlap{\rule[.5ex]{\wdth}{1pt}}#1}

\newcommand{\exref}[1]{(\ref{#1})}
% This command puts reference numbers with parentheses
% surrounding them 

% The environment ``examples'' gives a list of examples, one on each line,
% numbered with a lower case alphabetic character
\newenvironment{examples}%
   { \vspace{-\baselineskip}
     \begin{list}%
     \textrm{\alph{subexamplectr}.}%
     {\usecounter{subexamplectr}
     \setlength{\topsep}{-\parskip}
     \setlength{\itemsep}{-2pt}
     \setlength{\leftmargin}{0.5in}
     \setlength{\rightmargin}{0in} } }%
   { \end{list}}

% The environment ``myexample'' outputs an arabic counter ``examplectr''
% surrounded by parentheses.
\newenvironment{myexample}
   { \vspace{20pt}
     \noindent
     \begin{minipage}{\textwidth}    % minipage environment disallows
                 % breaks across pages

     \refstepcounter{examplectr}     % step the counter and cause this
                 % section to be referenced by the
                 % counter ``examplectr''
     (\arabic{examplectr})}%
   { \vspace{20pt}
     \end{minipage}}

\newenvironment{myfnexample}
   { \vspace{2pt}
     \noindent
     \begin{minipage}{\textwidth}    % minipage environment disallows
                 % breaks across pages

     \refstepcounter{fnexamplectr}     % step the counter and cause this
                 % section to be referenced by the
                 % counter ``examplectr''
     (\roman{fnexamplectr})}%
   { \vspace{2pt}c
     \end{minipage}}
    
\newcommand*\circled[1]{\tikz[baseline=(char.base)]{
            \node[shape=circle,draw,inner sep=2pt] (char) {#1};}}

\newcommand{\data}[1]{\textit{#1}}
\newcommand{\nodata}[1]{#1}
\newcommand{\blank}{\rule{1.2em}{0.5pt}}
\newcommand{\pt}[1]{\ensuremath{\mathsf{#1}}}
\newcommand{\ptv}[1]{\ensuremath{\textsf{\textsl{#1}}}}
\newcommand{\sv}[1]{\ensuremath{\mathcal{#1}}}

\newcommand{\sX}{\sv{X}}
\newcommand{\sF}{\sv{F}}
\newcommand{\sG}{\sv{G}}
\newcommand{\greekp}{\upvarphi}
\newcommand{\greekr}{\uprho}
\newcommand{\greeks}{\upsigma}
\newcommand{\MultiLine}[1]{\ensuremath{\begin{array}[b]{@{}l@{}}#1\end{array}}}
\newcommand{\LexEnt}[3]{#1; \ensuremath{#2}; \syncat{#3}}

\newcommand{\LexEntBroken}[3]
  {\Shortstack
      {%
        {#1;} 
        {\ensuremath{#2};} 
        {\syncat{#3}}%
      }%
  }

\newcommand{\grey}[1]{\colorbox{mycolor}{#1}}
\definecolor{mycolor}{gray}{0.8}

\newcommand{\gap}{\longrule}
\newcommand{\gp}{\gap}
\newcommand{\vs}{\raisebox{.05em}{\ensuremath{\,\upharpoonright}}}

\newcommand{\E}{ε}

\newcommand{\EBob}[1]{\textsl{#1}}

\newcommand{\B}{\textbf}
\newcommand{\f}{{\color{green}f}}  % Question what does f do? It does not have any output in the
                                % original PDF
%\newcommand{\Lemma}{{\color{pink}Lemma}}
\newcommand{\Lemma}{\ensuremath{\vdots\hskip.5cm\vdots}\noLine}

%\newcommand{\calP}{{\color{pink}calP}} % Sebastian
\newcommand{\calP}{\ensuremath{\mathcal{P}}}


\newcommand{\maru}[1]{\ooalign{\hfil#1\/\hfil\crcr
      \raise.05ex\hbox{\LARGE\mathhexbox20D}}}


%\newcommand{\sem}[2][M\!,g]{\mbox{$[\![ \mathrm{#2} ]\!]^{#1}$}}
\newcommand{\sem}{\ensuremath}

%
\newcommand{\trns}[1]{\textbf{#1}\xspace}
\newcommand{\bs}{{\textbackslash}}
\newcommand{\bsl}{{\bs}}
\newcommand{\fb}[1]{\textsubscript{#1}}
\newcommand{\syncat}[1]{\ensuremath{\mathrm{#1}}}
\newcommand{\term}[1]{\textit{#1}}
\newcommand{\LemmaAlt}{\ensuremath{\vdots\hskip.5cm\vdots}}
\NewDocumentCommand{\VanLabel}{m}{\MakeUppercase{#1}}

   %% hyphenation points for line breaks
%% Normally, automatic hyphenation in LaTeX is very good
%% If a word is mis-hyphenated, add it to this file
%%
%% add information to TeX file before \begin{document} with:
%% %% hyphenation points for line breaks
%% Normally, automatic hyphenation in LaTeX is very good
%% If a word is mis-hyphenated, add it to this file
%%
%% add information to TeX file before \begin{document} with:
%% %% hyphenation points for line breaks
%% Normally, automatic hyphenation in LaTeX is very good
%% If a word is mis-hyphenated, add it to this file
%%
%% add information to TeX file before \begin{document} with:
%% \include{localhyphenation}
\hyphenation{
    Ber-ti-net-to
    caus-a-tive
    fest-schrift
    Fest-schrift
    Hix-kar-ya-na
    In-do-ne-sian
    mor-pho-phon-o-log-i-cal
    Mo-se-tén
    par-a-digm
    phra-ses
    Que-chua
}

\hyphenation{
    Ber-ti-net-to
    caus-a-tive
    fest-schrift
    Fest-schrift
    Hix-kar-ya-na
    In-do-ne-sian
    mor-pho-phon-o-log-i-cal
    Mo-se-tén
    par-a-digm
    phra-ses
    Que-chua
}

\hyphenation{
    Ber-ti-net-to
    caus-a-tive
    fest-schrift
    Fest-schrift
    Hix-kar-ya-na
    In-do-ne-sian
    mor-pho-phon-o-log-i-cal
    Mo-se-tén
    par-a-digm
    phra-ses
    Que-chua
}

   \boolfalse{bookcompile}
   \togglepaper[23]%%chapternumber
}{}

\begin{document}
\maketitle

\section{Anarchy and power} 

Noam Chomsky is not only known for his linguistic work but also for his political views. He is an
anarcho-syndicalist. His political followers like anarchistic ideas since anarchy is the absence of
power. Humans live in self-regulated communities without oppression by a state or by a group of
people who somehow gained an advantage at a certain point and then have (mis-)used it to indoctrinate, command, influence, or
exploit other people.

\begin{figure}
\includegraphics[width=\linewidth]{figures/Chompsky-Anarchist-Bookstore-20151017-11-19-16.jpg}
\caption{Noam Chomsky sometimes presents his more dangerous ideas using a cover name. This one is
  probably inspired by the chimpanzee Neam Chimpsky. Anarchist
  bookstore in London, 2015, picture: Stefan Müller}
\end{figure}

But what is described in the movie \emph{Grammar of happiness} \citep{OW2012a} and even more clearly
in \crossrefchaptert{3_pullum}, is exactly the opposite. Chomsky and others have made the statement that all languages may license an infinite number of utterances in principle (\citealt*[\page 1571]{HCF2002a}, \citealt[\page 4]{EH2005a-u}, \citealt*[\page
7]{HNG2005a}), however there are several languages that seem to contradict this claim (see
\crossrefchaptert[\pageref{page-non-infinite-languages-start}--\pageref{page-non-infinite-languages-end},
Section~\ref{sec-discovering-amazonian}]{3_pullum} for a recent overview). Instead of
admitting the mistake and restating the claim, which would be a real sign of greatness, Chomsky and
other linguists from the US and Brazil have started a campaign against a single person with the aim of destroying that
person's scientific career and harming his integrity. Given the situation the field of linguistics is currently in,
this aggressive approach must be seen as a sign of weakness on the part of the attackers.

For somebody who is interested in languages and linguistics, such linguistic wars \citep{Harris93a}
must have a repellent effect: Don't sleep, there are snakes!    

\section{Everett in Europe}

While Pullum's piece reads well, as all of his papers do, it is also depressing. How could this happen?
How could an anarchist gain so much power? [For non-linguists: No, the answer is not: ``because he or
his school of thought have always had better arguments than others.'' At least not for the past 30
years.\footnote{
Every time the Chomskyan framework has come too close to what other branches of syntax research are doing,
Chomsky has changed fundamental assumptions about the architecture of the human language faculty. All
derivational models so far have been fundamentally incompatible with psycholinguistic insights. This is really
surprising since Mainstream Generative Grammar sees itself as research on one of our cognitive capabilities. So
psycholinguistic evidence should be part of the empirical facts on which linguistic theories are
built. See \citet{Wasow2021a} and \citet{BM2021a} for psycholinguistic facts and criticism about the
architecture of Minimalism.

See also \citet{LLJ2000b,LLJ2000a,LLJ2001a} on Chomskyan ``revolutions'' in the Minimalist era.
}]

One of the reasons I wrote this contribution is that
there is good news: The measurable power of Chomsky drops suddenly after a distance of 5000~km from
MIT. It almost reaches the European border but not quite, although it does extend to non-European countries like
the UK, as described with respect to Oxford University by \crossrefchaptert[Section~\ref{sec-Oxford-cancelation}]{3_pullum}. 

Europe has treated Everett quite differently from what is written in Pullum's paper. Everett has given talks at
various Mainstream Generative Grammar (MGG) institutions like the Zentrum Allgemeine
Sprachwissenschaft (ZAS) in Berlin. And he was invited to
speak at the annual meeting of the Deutsche Gesellschaft für Sprachwissenschaft (DGfS), the analog of the
LSA, and dominated by MGG researchers. The invited speaker is chosen by local organizers of the
conference and in 2010 it was organized by the ZAS and the Humboldt-Universität zu Berlin.\footnote{
\url{https://dgfs.de/jahrestagung/berlin_2010/programm_pv.htm}, 2024--09--23.
}
The same DGfS conference had a workshop on recursion with Tecumseh Fitch, one of the authors of
\citet*{HCF2002a}, as an invited speaker. The house was full and I remember lively
discussions. Science as it should be.

Everett has been invited to Potsdam, which is also a stronghold of MGG, several times (2014, 2018). I also
remember events with Ted Gibson at the ZAS where he discussed Pirahã.

I learned about the movie \emph{Grammar of happiness} from my late colleague Gisbert Fanselow, one of
the best German grammarians, who also worked in MGG. He told me that he was watching the movie with his
students during the last lecture before Christmas. I then started to do the same. Given the power
structure of the field, I normally would not have done this, but since he did this as a serious MGG grammarian, I
felt that I was able to do the same. The fact that I initially would not have dared to watch the movie with
students reminded me of GDR times, during which an official statement with the wrong content would have
ended one's scientific career. But the fact that Gisbert had watched it indicates that the situation in
Germany differed from that in the US.

%2014 Potsdam

%April 29, 2013 – ‘Towards Ethnogrammar’, Max Planck Institute, Leipzig, Germany  

%May 17, 2012 – ‘Language, the Cultural Tool’, Colloquium: "Exploring the Syntax-Semantics Interface" Heinrich-Heine-Universitat, Dusseldorf, Germany

%May 15, 2012 – ‘Language, the Cultural Tool’, Leiden University, Netherlands.

%January 9--20, 2012 – ‘Field Methods,’ The National Winter Linguistics Course LOT; Tilburg
%University, The Netherlands.

So, the conclusion is: Things do not have to be the way they are in the US. While there are conflicts between the camps,
they seem to be more civil and also more fruitful here. There have been joint workshops about
Construction Grammar and Minimalism at the Freie Universität Berlin (2007, with Richard Kayne, Adele
Goldberg, Gereon Müller, Anatol Stefanowitsch, and others),\footnote{
\url{https://www.geisteswissenschaften.fu-berlin.de/izeus/media/program_comparing_languages_workshop.pdf},
2024--09--23.
} and a workshop on progress
in linguistics with researchers from various frameworks present.\footnote{
\url{https://hpsg.hu-berlin.de/Events/HPSG2013/progling.html}, 2024--09--23.
} There have been framework comparison
events in Bergen, Norway (2005, PhD School Languages and Theories in Contrast), and Utrecht,
Netherlands (2009, Comparing Frameworks). People talk to each
other instead of talking about each other. Or rather in addition to. Starting in the 90s, there was an empirical
turn in which researchers did not focus on the intricate suggestions developed by hardcore Minimalists
but did more empirically oriented work instead.


\section{Language Science Press and festschrifts}

When Martin Haspelmath and I founded Language Science Press, we installed the rule that we would
not publish festschrifts. The rationale behind this was that nobody in his or her right mind would
publish a paper that could be published in \emph{Language} or in the \emph{Journal of Linguistics} in a
festschrift. Festschrift papers are usually focused on the person to be honored, they describe how
person X was important in the life of the author, how funny, honest, what a person of integrity X is. How helpful X was
as a supervisor. Sometimes
unpublished material that has been lying in some drawer for decades is recycled. This was the case
with the only festschrift article I ever published \citep{MuellerDefaults}.\footnote{
  Now it has a new citation. Including the citation in this paper, there are now seven citations on
  Google Scholar. All of them are self-citations. The paper is about embedding and recursion, by the way -- that you
  cannot do it in inheritance networks, not even with defaults. So, maybe I am the only one who
  finds this relevant or the stuff in festschrifts is usually ignored. Either way, this is a further
  argument against festschrifts.
}
This rejection of festschrifts is something I learned from the most famous German MGG researcher:
Manfred Bierwisch. If I am not mistaken, he never published anything in a festschrift.

When we agreed on the no-festschrift rule, we left an escape hatch open: of course people can do a
normal edited volume on a certain topic and give this to somebody as a present. But it should be a
normal peer reviewed volume. In general, festschrifts are bad for Language Science Press, since they are
expensive. Collections are more expensive than monographs since twenty different authors have twenty
different ways of writing strange \LaTeX{} code, misciting, doing funny things with figures, or of causing havoc
in other unseen ways. Festschrifts are even worse since the authors are usually well-established scholars
in the field, which means that all the problems mentioned above become increasingly severe combined with
dramatically longer response times. Festschrifts usually come with strict deadlines, which stands in the
way of enforcing quality standards. Our usual procedure of community proofreading/editing cannot be
applied since the ``non-festschrift'' has to remain a secret until the day of presentation. 

So far, Language Science Press has published six non-festschrifts
\citep{BS2017a-ed,BBDV2020a-ed,LS2021a-ed,MS2022a-ed,BHZ2017a-ed,BBDGN2018a-ed}, some of which were published
in my series \citep{BHZ2017a-ed,BBDGN2018a-ed}.
%
% 
% https://langsci-press.org/catalog/book/360
% 
% https://langsci-press.org/catalog/book/275 , 276, 277
% 
% https://langsci-press.org/catalog/book/159 , 115
%
%
% BS2017a-ed    OGS      Anders Holmberg      1 fett Dedication page und Vorwort
% BBDV2020a-ed  OGS      Ian Roberts          9 ?? catalog
% BBDV2020b-ed  OGS                           10
% BBDV2021a-ed  OGS                           11
% LS2021a-ed    OGS      Susi Wurmbrand       preface, acknowledgements 
%
% MS2022a-ed    EURO SLA Florence Myles       6 catalog
%
% BHZ2017a-ed   EOTMS    Stephen R. Anderson  3 preface, catalog
% BBDGN2018a-ed EOTMS    Bernard Fradin       4 clean but review
%
In the case of \citet{BBDGN2018a-ed}, the book itself is clean: there is no mention of a festschrift or
tribute. But then a review appeared in \emph{Language} by \citet{Bauer2020a-u} mentioning that the
volume is a festschrift and who it honored. 

The first five volumes of a new series are run through the press directors. After this we trust the
series editors of the respective series to continue their good work. Some of the festschrifts appeared
this way: I saw a tweet by Susi Wurmbrand saying thank you for her festschrift \citep{LS2021a-ed}.


In the case of \citet{BHZ2017a-ed}, I missed the statement in the preface that the book is a tribute,
but I saw the dedication in the catalog entry. While the book cannot be changed after publication, I
changed the catalog entry and informed the editor of the book. This led to quite an email
discussion and since the case was lost anyway, we put the dedication back in the catalog. 
Martin Haspelmath wrote an email to me back then
(p.\,m.\,2017--05--23): ``Besser machen bedeutet viel\-leicht: In Zukunft überhaupt keine Festschrift mehr
akzeptieren.''\footnote{
%Maybe improving things would mean we no longer accept festschrifts.OK
%Maybe improving things would mean strictly refusing to accept any festschrifts in the future." (finde ich aber irgendwie weird)
%
%oder
%
``Maybe improving things would mean rejecting festschrifts, without exception, in the future.''
}

The current volume was also an interesting case. I explained the no-festschrift policy several
times to the editors. When I saw the first outline of the chapters, I remarked that ``A journey into Dan Everett's
brain'' sounds a bit too festschrifty. Geoff Pullum sent me a draft of his paper and told me
about the workshop where he would be presenting this paper. I almost fell off my chair when I
read the email since the URL of the workshop -- probably widely distributed -- contained the forbidden keyword
``festschrift''.\footnote{
\url{https://tedlab.mit.edu/everett_festschrift_2023.html}, 2023--06--13. Note that the title of the
page is \emph{Everett Festschrfit 2023} with a typo. I guess this was done on purpose to confuse the
Language Science Press search robots which constantly monitor the web to find breaches of the
no-festschrift rule.
} After all these discussions: Language Science Press does not do festschrifts! The
workshop was great, and I especially enjoyed the journey into Dan Everett's brain. This was not
sloppy festschrift chitchat but serious science with brain images and so on. Still, it was specially
tailored to Dan Everett and perhaps unpublishable in ``normal'' journals. Thus the planned volume would
scream festschrift in every aspect, so that I felt it would be best to call it what it is, a festschrift,
and then officially end the seemingly never-ending nightmare of Language Science Press (non-)festschrifts. 

The team of press directors changed as of 2022--01--01 and Oliver Czulo stepped in for
Martin Haspelmath. We discussed allowing festschrifts if the editors pay for it. Something like the
10,000€ that is usually charged by profit-oriented publishers.\footnote{
Brill charges €10,000/\$12,200 for 100,000 words/250 pages (\url{https://brill.com/page/oacharges},
2024--09--23). Cambridge charges £10,500 (US\$13,000, €12,000) for 120,000 words and up to 85 figures
(\url{https://www.cambridge.org/core/services/open-access-policies/open-access-books/gold-open-access-books}, 2024--09--23).
} 
 We abandoned this idea because it would suggest that we publish low-quality work for money. After endless discussions, we
finally decided on the 5th of June 2023 to never ever do a festschrift again (starting 2024). 



% https://www.collegepublications.co.uk/tributes/

% 302 Emails with festschrift in the body.


\section{Universals}

The interesting fact about festschrifts is that they create an infinite amount of work. This is somewhat
surprising since the number of words and the number of references per chapter are strictly
finite. Maybe Friedrich Engels's insight is correct that a certain increase in quantity may result in a new
quality \citep[\page 349]{Engels1873a-u}. This goes against everything mathematicians tell us, but who knows.
On top of the amount of work caused in the editing process alone, we have had endless (!!!) discussions
with authors and editors -- some even involving Language Science Press's advisory board -- about what it means to publish a non-festschrift with Language Science
Press. Since the discussions are endless, the editing process~+ discussion is definitely endless. This
is not just what math tells us but is also supported by our feelings.


One of the weaker arguments for Universal Grammar, innate language-specific knowledge that helps
learners in language acquisition, is the claim that there are language
universals \parencites[\page 237--238]{Pinker94a}[\page 33]{Chomsky98a-u}. Whether there are such
universals and whether they require the assumption of domain-specific innate knowledge to be explained is an ongoing debate \parencites{Hawkins88a-ed,PF2000a,EL2009a,EL2009b}[Section~13.1]{MuellerGT-Eng5}. The response of researchers
working in Mainstream Generative Grammar to claims about languages that seem to contradict
putative universals is: Yes, but you cannot argue with unanalyzed data \citep[\page
454]{Freidin2009a},\footnote{
``Data alone cannot speak to the validity of explicit proposals
about the content of UG. What is required is an explicit analysis of data that follows from a
precisely formulated fragment of a grammar. This is a comment about methodology, independent of any
particular linguistic theory. In science there is no alternative to providing explicit analysis of
data. The discussion of UG in this article misses the mark entirely.''
This statement is false. If a proposal is made that our linguistic machinery allows us to produce an
infinite number of sentences and that it follows for all languages that sentences of arbitrary
length may be formulated in all languages, then one language that has a maximal sentence length is a
counterexample. (See \crossrefchaptert[\pageref{page-non-infinite-languages-start}--\pageref{page-non-infinite-languages-end}, Section~\ref{sec-discovering-amazonian}]{3_pullum} for a list of putatively finite languages.) If it is stated that Subjacency is a principle that holds for all languages, then
it is sufficient to point out that there are examples of extraposition in German that show that this
type of non-local dependency cannot cross just two NP boundaries but arbitrarily many \parencites{Mueller2004d}[Section~13.1.5]{MuellerGT-Eng5}. To make such
claims about data, no elaborated formalized grammar is necessary. Some understanding of traditional
grammar is sufficient. Sometimes MGG researchers state that examples of a
certain kind are predicted by their theory not to be possible. It is then sufficient to find such
examples without having a theory about these examples oneself. My dissertation and habilitation are
full of such examples \citep{Mueller99a,Mueller2002b}, as well as of an alternative theory.
}
meaning that if you look at an OSV language long enough, you will realize that it is underlyingly SVO.\footnote{
See \citet[\page 141]{Chomsky65a} and \citet{Kayne94a-u} for the claim that all languages are
underlyingly SVO,
% Chomsky said that the base is universal and then provided rules for English.
\citet{McCawley70a-u} argued for an underlying VSO order, \citet[]{Bach71a-u} and \citet{Ross73a-u} argued for OV, and
\citet{Haider2000a,Haider2010a,Haider2020a} claimed that SVO languages are derived from an
underlying SOVO pattern.

Note that I am one of those myself. After ten years of working in a more what you see is what you
get (WYSIWYG) setting of linearization-based HPSG \citep{Mueller2004b}, I developed an analysis of seemingly multiple frontings that assumes SOV
to be the underlying order of German clauses \parencites{Mueller2005c,Mueller2005d,MuellerGS}. That
German is an SOV language is the consensus among linguists working on Germanic. See
\citet{MuellerGermanic} for a discussion of \citet{Dryer2013c}, whose classification is built on
surface occurrences.
}
Ironically, using ``unanalyzed data'' is a very common practice among MGG
grammarians. Often, they just cherry-pick arbitrary facts from papers describing understudied
languages. See \citet{Fanselow2004b} for some criticism related to this matter.
 
As a syntactician one might be inclined to think that grammars should at least require the concept of valence. But
if \citet{KM2012a} are right, the Iroquoian language Oneida does not even have syntactic valence. So
what we seem to be left with is the triviality that humans combine linguistic material to form
larger units (Merge, \citealp{HCF2002a}, \citealp[\page 52]{MuellerCoreGram}), without any implications about possible sentence length. Note that we seem to
require unheaded, flat structures for phenomena like \emph{student after student after student} \parencites{Matsuyama2004a}{Jackendoff2008a}{Bargmann2015a}[Section~4.1]{MuellerCxG}. So
the universal would be that we combine stuff. Nothing more. Not even a constraint on binary branching.

But note that I have found a different universal. A universal holding on the text level. 
\ea
Observation holding at least for (English, French, and German):\\
Festschrifts cause an infinite amount of work.
\z
I think this is a universal.
\ea
Universal 1 (conjecture):\\
All festschrifts in all languages cause an infinite amount of work.
\z
Of course more research on this (to be published elsewhere) is needed. But I strongly believe that
this conjecture also holds for languages like Pirahã\il{Pirahã}: If we took a festschrift in
any language, say English, and translated it into Pirahã, the situation would not improve. Given that
the amount of work needed for the English draft is infinite, adding a translation into another
language would not make it finite. 

Of course there is the question of festschrifts originally written in the language of the final
submission. What the result would be for the case of Pirahã is difficult to predict. In order to get
an answer here, close collaboration with Dan Everett seems necessary. The first problem is that
Pirahã does not have a writing system/culture. I guess the Pirahã people would not see the point in giving
somebody a festschrift. But note that my statement is a statement about the cognitive abilities of
scientists. Thus: If there were Pirahãian scientists producing festschrifts, these would cause an
infinite amount of work. The older and more established Pirahãian scientists would not use reference
managers, they would be sloppy and forgetful with respect to sources, and so on.

So, I think, I have found the only true universal: It is a virtual necessity that fest\-schrifts lead to a
disaster.

\section{Conclusions}

Language Science Press therefore invites everybody to publish their festschrifts with the
competition and to publish either unedited low-quality stuff or create high
costs for our competitors.\footnote{
  De Gruyter is a good candidate, as they do not seem to care about quality. For example, the target article
  \citet{Trinh2009a-u} in Theoretical Linguistics contains 14 occurrences of the phrase \emph{in
    other word}, which should have been \emph{in other words}. Trinh's reply to replies
  \citeyearpar{Trinh2010a-u} contains three occurrences of this phrase. The phrase occurs 31 times in
  Trinh's MIT thesis \citeyearpar{Trinh2011a}, which is based on the discussion paper, and 29 times in the book that was finally published by De Gruyter
  \citep{Trinh2019a}. The book contains an unbelievable amount of further typos. No editor, copy editor, or
  supervisor seems to have read the papers, the thesis, or the book. The content is also wanting:
  Trinh argues for a VP analysis of German verbal complexes. An earlier MIT dissertation with the
  same approach was heavily criticized by \citet{ReisSternefeld2004}. None of the arguments of \citeauthor{ReisSternefeld2004} were taken up by Trinh (ignorance of the supervisors, failure of the series editors
  at De Gruyter). Reis \& Sternefeld's criticism and most of the field of Germanic syntax
  (MGG or alternatives) were completely ignored. See also \citet[\page 505]{ReisSternefeld2004} on
  the phenomenon of MIT theses exclusively citing work from the narrow MIT bubble.
}
The press managers of Language Science Press have decided that there will not be any further
(non-)festschrifts as of 2024. Since this volume is the first official festschrift of Language
Science Press, Dan Everett gets the first, last, and only official festschrift published by Language
Science Press. A truly outstanding achievement. 

  
% \section*{Abbreviations}
% \begin{tabularx}{.45\textwidth}{lQ}
%  \ldots  & \\
%  \ldots  & \\
% \end{tabularx}
% \begin{tabularx}{.45\textwidth}{lQ}
%  \ldots  & \\
%  \ldots  & \\
% \end{tabularx}


\section*{Afterthoughts}

Of course all papers published up to now in Language Science Press non\hyp fest\-schrifts are exceptionally good. So they
cannot be taken as examples of what I said above: papers of low quality, not really about linguistics,
with strange argumentations, in a funny festschrift style, and never cited. There is one paper that is an
exception: this one. I hope it will never be cited but often read.

\section*{Acknowledgements}

I thank Sebastian Nordhoff for the final impulse to stop the festschrift madness. Blame him! I also
want to thank Sebastian for sending me a list of \hologo{BibTeX} entries for Language Science Press
festschrifts, which he created with ChatGPT. The list was completely useless but fun. See
\crossrefchaptert{16_piantadosi} and \citet{MuellerLLMs} for more on ChatGPT. I thank
Dan Everett for being the topic of my Christmas movie. I must have watched it at least 20 times by now.


%\section*{Contributions}
%John Doe contributed to conceptualization, methodology, and validation.
%Jane Doe contributed to writing of the original draft, review, and editing.


{\sloppy\printbibliography[heading=subbibliography,notkeyword=this]}
\end{document}


%      <!-- Local IspellDict: en_US-w_accents -->
