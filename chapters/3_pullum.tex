\documentclass[output=paper,colorlinks,citecolor=brown
% ,hidelinks
% showindex
]{langscibook}

\ChapterDOI{10.5281/zenodo.12665907}
%%% GKP:  The following line produces an error message about an
%%%       incomplete \ifx, but the error is apparently harmless.
\author{Geoffrey K. Pullum\orcid{0000-0002-7748-8847}\affiliation{University of Edinburgh}}
\title{Daniel Everett on Pirahã syntax}
\abstract{Daniel Everett's generalizations about the syntax of the
Brazilian indigenous language Pirah{\~a} in 2005 provoked not just a
linguistic dispute but also an international campaign of vilification
and abuse against him. Yet many other languages have been claimed to
have the properties he attributes to Pirah{\~a} (basically, absence
of devices like hypotaxis and clausal coordination). Nevins, Pesetsky,
and Rodrigues attempt to represent Everett as having dishonestly
concealed earlier evidence of hypotaxis in the language. They are
not successful. Later attempts by others to exhibit self-embedding
in Pirah{\~a} syntax fare even worse. The issue has little general
importance for linguistics, since nothing important about language
or humanity hangs on whether an upper bound on sentence length exists.
In pursuing the matter, Everett's accusers have done him a gross
injustice.}


\IfFileExists{../localcommands.tex}{
   \addbibresource{../localbibliography.bib}
   \usepackage{orcidlink}
\usepackage{tabularx,multicol}
\usepackage{url}
\urlstyle{same}

\usepackage{siunitx}
\sisetup{group-digits = none}

\usepackage{langsci-branding} 
\usepackage{langsci-optional}
\usepackage{langsci-lgr}
\usepackage{langsci-tbls}
\usepackage{langsci-gb4e}

% Müller
\usepackage{tikz-qtree}
\usepackage{hologo}

% 3_pullum.tex
\usepackage{langsci-textipa}

% 8_levine
\usepackage{bm}
\usepackage{umoline}
\usepackage{pifont}
\usepackage{pstricks,pst-node,pst-tree}
\usepackage{ulem}
\usepackage{mathrsfs}
\usepackage{bussproofs}

% 14_kornai
\usepackage[matrix,arrow]{xy}
\usepackage{subcaption}

\usepackage[linguistics, edges]{forest}
\usetikzlibrary{arrows, arrows.meta}

   \SetupAffiliations{output in groups = false,
                   orcid placement = after,
                   separator between two = {\bigskip\\},
                   separator between multiple = {\bigskip\\},
                   separator between final two = {\bigskip\\}
                   }

% ORCIDs in langsci-affiliations 
\definecolor{orcidlogocol}{cmyk}{0,0,0,1}
\RenewDocumentCommand{\LinkToORCIDinAffiliations}{ +m }
  {%
    \,\orcidlink{#1}%
  }

\makeatletter
\let\thetitle\@title
\let\theauthor\@author
\makeatother

% Cite and cross-reference other chapters
\newcommand{\crossrefchaptert}[2][]{\citet*[#1]{chapters/#2}, Chapter~\ref{chap-#2} of this volume} 
\newcommand{\crossrefchapterp}[2][]{(\citealp*[#1]{chapters/#2}, Chapter~\ref{chap-#2} of this volume)}
\newcommand{\crossrefchapteralt}[2][]{\citealt*[#1]{chapters/#2}, Chapter~\ref{chap-#2} of this volume}
\newcommand{\crossrefchapteralp}[2][]{\citealp*[#1]{chapters/#2}, Chapter~\ref{chap-#2} of this volume}

\newcommand{\crossrefcitet}[2][]{\citet*[#1]{chapters/#2}} 
\newcommand{\crossrefcitep}[2][]{\citep*[#1]{chapters/#2}}
\newcommand{\crossrefcitealt}[2][]{\citealt*[#1]{chapters/#2}}
\newcommand{\crossrefcitealp}[2][]{\citealp*[#1]{chapters/#2}}


\newcommand{\sub}[1]{\textsubscript{\scriptsize\textrm{#1}}}
% Müller
\newcommand{\page}{}

\let\citew\citet
\def\underRevision{Revise and resubmit}
\let\textbfemph\emph

%% % taken from https://tex.stackexchange.com/a/95079/18561
\newbox\usefulbox

\makeatletter
\def\getslant #1{\strip@pt\fontdimen1 #1}

\def\skoverline #1{\mathchoice
 {{\setbox\usefulbox=\hbox{$\m@th\displaystyle #1$}%
    \dimen@ \getslant\the\textfont\symletters \ht\usefulbox
    \divide\dimen@ \tw@ 
    \kern\dimen@ 
    \overline{\kern-\dimen@ \box\usefulbox\kern\dimen@ }\kern-\dimen@ }}
 {{\setbox\usefulbox=\hbox{$\m@th\textstyle #1$}%
    \dimen@ \getslant\the\textfont\symletters \ht\usefulbox
    \divide\dimen@ \tw@ 
    \kern\dimen@ 
    \overline{\kern-\dimen@ \box\usefulbox\kern\dimen@ }\kern-\dimen@ }}
 {{\setbox\usefulbox=\hbox{$\m@th\scriptstyle #1$}%
    \dimen@ \getslant\the\scriptfont\symletters \ht\usefulbox
    \divide\dimen@ \tw@ 
    \kern\dimen@ 
    \overline{\kern-\dimen@ \box\usefulbox\kern\dimen@ }\kern-\dimen@ }}
 {{\setbox\usefulbox=\hbox{$\m@th\scriptscriptstyle #1$}%
    \dimen@ \getslant\the\scriptscriptfont\symletters \ht\usefulbox
    \divide\dimen@ \tw@ 
    \kern\dimen@ 
    \overline{\kern-\dimen@ \box\usefulbox\kern\dimen@ }\kern-\dimen@ }}%
 {}}
\makeatother

% 1_intro.tex

% For the block quote:
\definecolor{linequote}{RGB}{224,215,188}
\definecolor{backquote}{RGB}{249,245,233}

\NewDocumentEnvironment{myquote}{ +m }
  {%
    \begin{tblsfilled}{}[black!12]
    #1%
  }
  {\end{tblsfilled}}

% 2_gibson.tex


% Example(s) Environments
% 12pt, No new-lines after example number is printed

\newcounter{examplectr}
\newcounter{fnexamplectr}

% Note: don't use subexamples in footnotes.

% This line is to overcome a bug in cmu-art style: it prints counter
% values to the aux file using \theaux... rather than using \the...
\def\theauxexamplectr{\theexamplectr}

\newcounter{subexamplectr}
\def\theauxsubexamplectr{\thesubexamplectr}
\def\theauxfnexamplectr{\thefnexamplectr}

\renewcommand{\theexamplectr}{\arabic{examplectr}}
% This command causes example numbers to appear without following periods

\renewcommand{\thefnexamplectr}{\roman{fnexamplectr}}
% This command causes example numbers to appear without following periods

\renewcommand{\thesubexamplectr}{\theexamplectr\alph{subexamplectr}}
% This command gives the number of an example and subexample as e.g. 1a, 2b

\newlength{\wdth}
\newcommand{\strike}[1]{\settowidth{\wdth}{#1}\rlap{\rule[.5ex]{\wdth}{1pt}}#1}

\newcommand{\exref}[1]{(\ref{#1})}
% This command puts reference numbers with parentheses
% surrounding them 

% The environment ``examples'' gives a list of examples, one on each line,
% numbered with a lower case alphabetic character
\newenvironment{examples}%
   { \vspace{-\baselineskip}
     \begin{list}%
     \textrm{\alph{subexamplectr}.}%
     {\usecounter{subexamplectr}
     \setlength{\topsep}{-\parskip}
     \setlength{\itemsep}{-2pt}
     \setlength{\leftmargin}{0.5in}
     \setlength{\rightmargin}{0in} } }%
   { \end{list}}

% The environment ``myexample'' outputs an arabic counter ``examplectr''
% surrounded by parentheses.
\newenvironment{myexample}
   { \vspace{20pt}
     \noindent
     \begin{minipage}{\textwidth}    % minipage environment disallows
                 % breaks across pages

     \refstepcounter{examplectr}     % step the counter and cause this
                 % section to be referenced by the
                 % counter ``examplectr''
     (\arabic{examplectr})}%
   { \vspace{20pt}
     \end{minipage}}

\newenvironment{myfnexample}
   { \vspace{2pt}
     \noindent
     \begin{minipage}{\textwidth}    % minipage environment disallows
                 % breaks across pages

     \refstepcounter{fnexamplectr}     % step the counter and cause this
                 % section to be referenced by the
                 % counter ``examplectr''
     (\roman{fnexamplectr})}%
   { \vspace{2pt}c
     \end{minipage}}
    
\newcommand*\circled[1]{\tikz[baseline=(char.base)]{
            \node[shape=circle,draw,inner sep=2pt] (char) {#1};}}

\newcommand{\data}[1]{\textit{#1}}
\newcommand{\nodata}[1]{#1}
\newcommand{\blank}{\rule{1.2em}{0.5pt}}
\newcommand{\pt}[1]{\ensuremath{\mathsf{#1}}}
\newcommand{\ptv}[1]{\ensuremath{\textsf{\textsl{#1}}}}
\newcommand{\sv}[1]{\ensuremath{\mathcal{#1}}}

\newcommand{\sX}{\sv{X}}
\newcommand{\sF}{\sv{F}}
\newcommand{\sG}{\sv{G}}
\newcommand{\greekp}{\upvarphi}
\newcommand{\greekr}{\uprho}
\newcommand{\greeks}{\upsigma}
\newcommand{\MultiLine}[1]{\ensuremath{\begin{array}[b]{@{}l@{}}#1\end{array}}}
\newcommand{\LexEnt}[3]{#1; \ensuremath{#2}; \syncat{#3}}

\newcommand{\LexEntBroken}[3]
  {\Shortstack
      {%
        {#1;} 
        {\ensuremath{#2};} 
        {\syncat{#3}}%
      }%
  }

\newcommand{\grey}[1]{\colorbox{mycolor}{#1}}
\definecolor{mycolor}{gray}{0.8}

\newcommand{\gap}{\longrule}
\newcommand{\gp}{\gap}
\newcommand{\vs}{\raisebox{.05em}{\ensuremath{\,\upharpoonright}}}

\newcommand{\E}{ε}

\newcommand{\EBob}[1]{\textsl{#1}}

\newcommand{\B}{\textbf}
\newcommand{\f}{{\color{green}f}}  % Question what does f do? It does not have any output in the
                                % original PDF
%\newcommand{\Lemma}{{\color{pink}Lemma}}
\newcommand{\Lemma}{\ensuremath{\vdots\hskip.5cm\vdots}\noLine}

%\newcommand{\calP}{{\color{pink}calP}} % Sebastian
\newcommand{\calP}{\ensuremath{\mathcal{P}}}


\newcommand{\maru}[1]{\ooalign{\hfil#1\/\hfil\crcr
      \raise.05ex\hbox{\LARGE\mathhexbox20D}}}


%\newcommand{\sem}[2][M\!,g]{\mbox{$[\![ \mathrm{#2} ]\!]^{#1}$}}
\newcommand{\sem}{\ensuremath}

%
\newcommand{\trns}[1]{\textbf{#1}\xspace}
\newcommand{\bs}{{\textbackslash}}
\newcommand{\bsl}{{\bs}}
\newcommand{\fb}[1]{\textsubscript{#1}}
\newcommand{\syncat}[1]{\ensuremath{\mathrm{#1}}}
\newcommand{\term}[1]{\textit{#1}}
\newcommand{\LemmaAlt}{\ensuremath{\vdots\hskip.5cm\vdots}}
\NewDocumentCommand{\VanLabel}{m}{\MakeUppercase{#1}}

   %% hyphenation points for line breaks
%% Normally, automatic hyphenation in LaTeX is very good
%% If a word is mis-hyphenated, add it to this file
%%
%% add information to TeX file before \begin{document} with:
%% %% hyphenation points for line breaks
%% Normally, automatic hyphenation in LaTeX is very good
%% If a word is mis-hyphenated, add it to this file
%%
%% add information to TeX file before \begin{document} with:
%% %% hyphenation points for line breaks
%% Normally, automatic hyphenation in LaTeX is very good
%% If a word is mis-hyphenated, add it to this file
%%
%% add information to TeX file before \begin{document} with:
%% \include{localhyphenation}
\hyphenation{
    Ber-ti-net-to
    caus-a-tive
    fest-schrift
    Fest-schrift
    Hix-kar-ya-na
    In-do-ne-sian
    mor-pho-phon-o-log-i-cal
    Mo-se-tén
    par-a-digm
    phra-ses
    Que-chua
}

\hyphenation{
    Ber-ti-net-to
    caus-a-tive
    fest-schrift
    Fest-schrift
    Hix-kar-ya-na
    In-do-ne-sian
    mor-pho-phon-o-log-i-cal
    Mo-se-tén
    par-a-digm
    phra-ses
    Que-chua
}

\hyphenation{
    Ber-ti-net-to
    caus-a-tive
    fest-schrift
    Fest-schrift
    Hix-kar-ya-na
    In-do-ne-sian
    mor-pho-phon-o-log-i-cal
    Mo-se-tén
    par-a-digm
    phra-ses
    Que-chua
}

   \boolfalse{bookcompile}
   \togglepaper[23]%%chapternumber
}{}

%   \is{Cognition} % add "Cognition" to subject index for this page
%   \il{Latin}     % add "Latin" to language index for this page

\begin{document}
\maketitle
\label{chap-3_pullum}

%%% GKP: This document tends to get unfortunate page breaks,
%%%      with big white spaces before the trees (6), (7), and (9).
%%%      There's no general solution to this; it varies whenever
%%%      prose is added or subtracted during revision.
%%%      We can only pray that we can get good page breaks in the
%%%      final publication.

\section{Everett's dangerous idea}\label{intro}

The war on Daniel Everett's reputation and research began soon after
the fall of 2005, when he he gave a two-part language tutorial session
on the Brazilian indigenous language Pirah{\~a} at the annual meeting
of the Linguistics Association of Great Britain in Cambridge, England
(September 1--2), and published an article entitled ``Cultural constraints
on grammar and cognition in Pirah{\~a}'' in the August-October issue of
\textit{Current Anthropology} (\textit{CA}). The publisher of \textit{CA},
the University of Chicago Press, put out a news release about the article
which led to some newspaper stories. The surprising result was that in
the following years Everett was subjected to bitter attacks impugning
not just his work but his integrity and character. The attacks emerged
first within the linguistics community, but have come to the attention
of a much wider public, particularly among admirers of Noam Chomsky.

The Pirah{\~a} are an uncompromisingly independent tribe of indigenous
Amazonian people living a subsistence-level low-technology lifestyle
on the banks of the Maici river in Amazonas state. They hardly interact
with mainstream Brazilian society at all, and show no interest in
reading, writing, counting, history, politics, or religion.

Their language appears unrelated to any other now spoken, and they
have remained resolutely monolingual in it for at least 200 years,
despite occasional contacts with other indigenous people, and
acquaintance with three generations of American missionaries, and
sporadic and superficial contacts with mainstream Brazilian river
traders. A very small number of Pirah{\~a} men have a smattering of
Portuguese and can act as interlocutors for Pirah{\~a} villages that
come into occasional contact with Portuguese-speaking Brazilian river
traders. \citet{Sakel12} calls them ``gatekeepers,'' and provides
some interesting data on their very rudimentary Portuguese (she
also notes some use of a local pidgin based on the Tupian language
Nheengatu). But the women speak only Pirah{\~a}, and the gatekeepers
basically shelter the vast majority of the Pirah{\~a} community
(including most of the men) from needing even a minimal competence
in Portuguese.

The Pirah{\~a} language is linguistically unusual in several ways,
from its tiny phonemic system and unusual phonology to its complete
absence of numerals and pure color terms. But although Everett's
statements on these points raised some linguists'
eyebrows,\footnote{\label{dobrin}
   See \citealt{DobrSchw21} for an interesting discussion of the ways
   in which knowledge is based in fieldwork, and how differing
   assumptions about things like how to devise glosses contributed
   to the conflict between Everett and his critics on the quantifier
   issue.}
they did not provoke anger. What did, and what motivated the surprising
events described in Section~\ref{war} below, was sentence structure.
This might seem an unlikely trigger for angry diatribes and libelous
allegations (at least for anyone who did not know the history of
generative syntax chronicled in \citealt{Harris21}).

It is highly relevant that all production of Pirah{\~a} is oral:
though an orthography has been devised, no member of the community
has shown any interest in learning to read or write. And oral
discourse in the language shows no signs of such familiar syntactic
phenomena or devices that writers use in constructing long sentences.
Everett reports that there are no signs of no multiple coordination
(\data{It takes} [\data{skill, nerve, initiative, and courage}]),
complex determiners ([[[\data{my}] \data{son's}] \data{wife's}]
\data{family}), stacked modifiers (\data{a}~[\data{nice,} [\data{cosy,}
[\data{inexpensive} [\data{little cottage}]]]]), or -- most significant
of all -- reiterable clause embedding (\data{I~thought} [\,\data{you
already knew} [\,\data{that she was here}\,]\,]). These are the primary
constructions that in English permit sentences of any arbitrary finite
length to be constructed, yielding the familiar argument that the set
of all definable grammatical sentences in English is
infinite.\footnote{\label{infinity}
   The soundness of the argument even for English can be questioned:
   Pullum and Scholz \citeyearpar[115--124]{PullScho10} argue that
   the claim of an actually infinite number of sentences cannot be
   sustained. But we can set that theoretical point aside here,
   concentrating on more concrete matters like whether the language
   permits embedding of clauses within clauses.}

\label{page-non-infinite-languages-start}Linguists versed in syntactic
typology were not the ones who expressed shock at the syntactic facts:
similar claims had long been made about other languages, sparking no
particular controversy. The anthropologist Brent Berlin, commenting
on the \textit{CA} paper (p.\,635, one of eight invited responses
published with the article) expresses no surprise about the absence
of subordination, and quotes a remark by Foley (\citeyear[177]{Foley86})
about the Papuan language Iatmul, where ``Linking of clauses is at the
same structural level rather than as part within whole.''

The late Kenneth Hale (1934--2001), a long-time MIT faculty member,
argued as early as the mid 1970s that the Australian language Warlpiri
could not even be said to have phrase structure, which would necessarily
entail it did not have syntactically subordinate clauses. Hale's work,
together with that of R.\,M.\,W.\ Dixon, founded a rich subdiscipline
of work on Australian languages, particularly the Pama-Nyungan family.
The literature is too large for a proper survey here, but suffice it to
say that examination of the example sentences presented in works on
Pama-Nyungan languages such as \citet{Hale76}, \citet{Nash80},
\citet{Dixon81}, \citet{AustBres96}, and \citet{Pensalfini04},
one finds no sign of any embedded complement clauses. Sentences
seem to consist solely of word-level constituents, word order often
being astonishingly free. There are signs of what might be non-finite
secondary predications at main clause margins which could perhaps be
called ``functionally dependent'' but ``structurally unembedded'' as
Austin and Bresnan suggest \citeyearpar[228, esp.\ n.\,13]{AustBres96},
but there is none of the clause subordination familiar from English
and other languages of the sort Benjamin Lee Whorf called
``Standard Average European.''

The relevant literature goes far beyond the work on Pama-Nyungan.
More than four decades ago the syntactic typology
specialist Talmy Giv{\'o}n \citeyearpar[298]{Givon79} wrote in very
general terms about languages of ``preindustrial, illiterate societies
with relatively small, homogeneous social units'' in which
``subordination does not really exist.'' Kalm{\'a}r
(\citeyear[esp.\ pp.\,157--159]{Kalmar85}), citing Giv{\'o}n,
elaborates further, giving several earlier references and raising the
interesting possibility that Canadian Inuktitut is in the process of
developing subordinate clauses for the first time in writing on serious
subjects.

\citet{Mithun84} studies the noticeable avoidance of subordination
in highly agglutinative languages employing polysynthesis in their
verb structures. She focuses on Gunwinggu (=\,Kunwinjku, citing 1951
and 1964 sources), Kathlamet (from a 1911 source), and Mohawk (from
her own contemporary informant work), and observes that they all
resist resorting to subordination, some almost completely. Evans and
Levinson (\citeyear[Section~6]{EvanLevi09}) take the view that quite
generally in Bininj Kun-wok (of which Kunwinjku can be regarded as a
dialect variant) there is no clausal embedding, and morphological
embedding is possible only to one degree. They also note (p.\,442)
that Kayardild (another Pama-Nyungan language) allow subordination,
``but caps it at one level of nesting'': the subordination cannot be
employed to put clauses inside clauses inside clauses and thus make
sentences arbitrarily long.

Mithun offers an interesting conjecture about why even one-level
subordination is avoided in such languages: in oral-only languages
it should perhaps not be seen as implying any shortcoming or lack on
their part, but rather an indication that once languages are written,
the necessarily slower composition and reception of the written form
leads to the development of new syntactic tools ``to compensate for
the loss of mechanisms inherent in skillful oratory'' such as
intonational phrasing (p.\,509).

Many other instances could be cited of linguists commenting long before
2005 on languages in which arbitrary sentence extensibility seems not
to be possible. And not just languages of hunter-gatherer cultures
but also languages of early antiquity in Europe and Asia: comments
about the lack of true hypotaxis can be found in literature on
early Akkadian, Old Chinese, Homeric Greek, and Proto-Uralic.

The late Wayne O'Neil (1931--2020), an MIT faculty member like Hale,
published a paper in \citeyear{ONeil77} arguing that early Old English
also showed no signs of clause embedding. Writers would just tack an
additional clauses on the end of a main clause, very loosely attached
(very much as in Pama-Nyungan). Once Old English speakers were able
``to take advantage of the leisure for the composition and decomposition
of sentences that being able to read and write afforded them,'' O'Neil
says, ``they took advantage of it in the simplest possible way \ldots\
by simply adjoining sentences to sentences, sometimes without even
deleting the shared nominal'' (\citealt{ONeil77}:210). The implication
is that before Old English was written, subordination was basically
absent from the language.\label{page-non-infinite-languages-end}

The claims referenced in the last half-dozen paragraphs may or may not
be correct in their detailed analytical claims; I am not trying to
evaluate them here. My point is merely that they provide descriptions
of languages in which it looks as if it would not be possible to
construct sentences of arbitrary length, and they have been sitting
uncontroversially on library shelves for decades. It is peculiar that
things changed so dramatically in 2005, and that the reaction was so
extreme, given that Everett was merely making a point about Pirah{\~a}
that had been repeatedly made before about other languages.

What had changed? The answer is that a paper co-authored by Marc
Hauser, Noam Chomsky, and W.\ Tecumseh Fitch had been published in
the prestigious general scientific journal \textit{Science}: Hauser
et al.\ (\citeyear{HauChoFit02}), henceforth HCF. The paper contains
a lot of evolutionary biology and zoology, and it is reasonable to
assume that the first-named author did most of the writing. Fitch was
an associate in Hauser's lab at Harvard, and Chomsky may have been
added more as a co-signatory, without having a role in detailed work
on the paper's content (this attributional matter is not irrelevant
in the light of the findings of scientific misconduct against Hauser
five years later; see footnote \ref{misconduct} below).

HCF included an informally phrased conjecture about what Chomsky calls
``Universal Grammar'' (UG). The conjecture was that the \textsc{sole}
aspect of linguistic structure attributable to a biologically rooted
``faculty of language in the narrow sense,'' unique to \textit{Homo sapiens},
is a special cognitive capacity for unbounded combining of mental
syntactic representations through repeated applications of a posited
binary set-formation operation called ``Merge.''\footnote{\label{recursion}
   In HCF and a voluminous subsequent literature these matters are discussed
   in terms of ``recursion.'' I will avoid the use of this term (which
   HCF nowhere defines) because linguists' use of it is a morass of
   confusion, as \citet{Lobina14} correctly points out. In mathematical
   logic, ``recursion'' refers to either definition by induction or
   computational routines that invoke themselves (\citealt{Soare96},
   esp.\ 286--89), and ``recursive'' is used of sets to mean ``having
   a decidable membership problem.''
   Linguists use ``recursion'' to refer either to self-embedding in
   phrase structure, or to iterated application of the ``Merge''
   operation, or to HCF's conjectured mental syntactic combinatory
   capacity, and they use ``recursive'' as a predicate of rules or
   grammars. I focus instead on the relatively clear issue of
   \textsc{what kinds of expressions the grammar permits}.}
To motivate this idea for a general scientific readership, HCF pointed
to a putatively self-evident fact about human language (p.\,1571):

\begin{quote}
The core property of discrete infinity is intuitively familiar to every
language user\ldots\ There is no longest sentence (any candidate sentence
can be trumped by, for example, embedding it in ``Mary thinks that \ldots''),
and there is no non-arbitrary upper bound to sentence length. In these
respects, language is directly analogous to the natural numbers\ldots
\end{quote}
Notice the phrase ``every language user,'' which suggests we are talking
about every language of biologically normal human beings anywhere on
earth. Note also HCF's claim that the human ``faculty of language in the
narrow sense'' must ``construct an infinite array of internal expressions
from the finite resources of the conceptual-intentional system'' (p.\,1578).

The content of the quotations above is entirely in line with Chomskyan
ideas, though it is plausible to assume that Hauser drafted much of the
article's text. The claims in HCF simply restate more emphatically a
view that stemmed from Chomsky's earliest work and had been standard
fare in linguistics textbooks for decades. Nearly half a century before,
Chomsky (\citeyear{Chomsky56}:113) had claimed that the key purpose
of a grammar was to project a finite corpus ``to an infinite set of
grammatical sentences,'' and over the next decade this became a part
of the usual motivation for generative grammar. Ronald Langacker
(\citeyear{Langacker68}:31), for example, was merely elaborating on
it when he wrote that ``The set of well-formed sentences in English
is infinite, and the same is true of every other language,'' adding
the standard argument that given a sentence of any length you can
construct a longer one by embedding it as a \data{that}-clause.
HCF was merely echoing such statements.

Two years before HCF, Lasnik (\citeyear{Lasnik00}:3) had put things
even more assertively in a syntax textbook, calling the availability
of infinitely many sentences a ``central'' universal of language:
\begin{quote}
Infinity is one of the most fundamental properties of human languages,
maybe the most fundamental one. People debate what the true universals
of language are, but indisputably, infinity is central.
\end{quote}
And six months before Everett's \textit{CA} article was published,
Sam Epstein and Norbert Hornstein (\citeyear{EpstHorn05}) cited HCF in
a letter (intended for publication in \textit{Science} but published
in \textit{Language} instead) defending the Chomskyan program and asserting
that ``human language is a highly structured formal combinatorial system
and, in addition, the number of discrete well-formed sentences generated
by the system is infinite.'' They continued (p.\,4):
\begin{quote}
This property of discrete infinity characterizes \mbox{\textsc{every}}
human language; none consists of a finite set of sentences. The unchanged
central goal of linguistic theory over the last fifty years has been and
remains to give a precise, formal characterization of this property and
then to explain how humans develop (or grow) and use discretely infinite
linguistic systems. [Emphasis in original -- GKP.]
\end{quote}
This differs from earlier claims only in being even more strident and
explicit.

The trouble for Everett was that by the mid 2000s, endorsing HCF's view
of the biological basis of language had become something of a test of
loyalty to the Chomskyan mainstream conception of syntax. Everett's
simple descriptive observation (with its many precedents in unnoticed
earlier literature) had become an ideologically dangerous idea.

Some attempts were made to answer it by reinterpreting HCF in a way
that could allow Everett's claims to be true without being relevant.
The tactic is to neutralize the dangerous idea by asserting that only
a vastly weaker hypothesis was ever really at issue. The main attack
on Everett in the refereed literature, \citet{NevPesRod09a}, briefly
mentions such a reinterpretation, claiming that under theories of the
sort HCF assumed, ``what is at stake is in fact the \textsc{general}
ability to build phrases that contain phrases as subparts'' and nothing
more (pp.\,366--67, fn.\,11). This retrospectively interprets HCF as
saying merely that phrases may contain other phrases. That must involve
Merge applying to objects formed by Merge, and that can be called
``recursion,'' vindicating HCF.

There are two problems, though. First, HCF's actual claim about
languages was never simply that some phrases can contain certain other
phrases (which could be entirely compatible with an upper bound on
sentence length). The reference to a literal infinity of sentences
quoted above (``There is no longest sentence'') is crystal clear. Second,
the notion that phrases may contain other phrases is absurdly weak:
no one ever doubted it, and no one could think it merited publication
in \textit{Science}.

Chomsky has nonetheless essayed a retreat to an even weaker thesis
(or at least a less empirically accessible one), which does not say
anything about languages at all. He has maintained in various interviews
that HCF was merely suggesting that there was a genetically inherited
mental capacity of our species that \textsc{would} permit humans to
learn languages with arbitrary sentence length, \textsc{if} they chose
to use it. Whether or not speakers of attested languages show signs
of using it is, Chomsky now claims, a total irrelevance. Speaking to
a 2016 interviewer, Chomsky stated that we can dismiss the evidence of
Pirah{\~a} syntax because ``if some tribe were found in which people
wear a patch over one eye and hence do not use binocular vision, it
would tell us nothing at all about the human faculty of
vision.''\footnote{\label{sorrentino}
   ``Chomsky: We are not apes, our language faculty is innate.''
   Interview with Filomena Fuduli Sorrentino, \textit{La Voce di New York},
   4 October 2016, online at
   \url{https://lavocedinewyork.com/en/2016/10/04/chomsky-we-are-not-apes-our-language-faculty-is-innate/}}

\citet[792--794]{Hornstein19} expounds this view at
greater length idea for anyone who didn't get the memo the first time.
He distinguishes ``Greenberg universals,'' to which evidence about languages
can be relevant, from ``Chomsky universals,'' which apparently await future
advances in neurophysiology for support or refutation. Unfortunately,
putting it this way reduces to nothing more than saying that there
must be some special combinatorial ability (HCF's ``faculty of language
in the narrow sense'') built into our brains somehow. The view makes no
testable predictions except that some sort of linguistic ability will
exist in normal humans; but we knew that when we arrived at the lab.

In the interview with Filomena Sorrentino mentioned above, Chomsky makes
an additional revealing remark.  Sorrentino asked him, ``Is there something
especially interesting about the Pirah{\~a} language?'' and he said:
\begin{quote}
The interesting properties of Pirah{\~a} have been studied in depth for
many years in a wide range of languages, most prominently by Everett's
mentor, MIT linguist Kenneth Hale, one of the leading figures in the
study of indigenous languages, who has produced many important studies
of these topics from the 1960s.
\end{quote}
There are some straightforward untruths here -- Chomsky's MIT colleague
Kenneth Hale, though admired by Everett and everyone else who knew him,
never served as ``Everett's mentor,'' since Everett's MA and PhD theses on
Pirah{\~a} had been completed before the two men met, and Hale never
worked on Pirah{\~a} at all -- but notice that Chomsky seems to be
acknowledging the existence of a language with no apparent syntactic
embedding. As mentioned above, Hale did point out in the 1970s that
Warlpiri lent no support to any theory of hierarchical constituent
structure, which would imply the absence of subordinate clause
constituents, and at that time Chomsky saw no reason to attack him
for it. It was only his pique at seeing HCF contradicted that motivated
his going on the offensive against Everett.

\citet{Everett05} was really just drawing the attention of syntactic
theorists to a pre-existing conflict. For decades linguists had been
drawing motivation for generative grammars from the proposition that
all human languages had infinite numbers of grammatical sentences.
Pirah{\~a} provides a particularly clear and much publicized case of
a language lacking the key syntactic constructions that could support
the truth of such claims. For those aggressively committed to the
totality of Chomsky's program, especially those knowing little of the
syntactic literature from two or three decades earlier, this message
had to be addressed by attacking the messenger.

The public part of the war on Everett began with a long paper about
his work first circulated in 2007 and ultimately published by
\textit{Language} in 2009. It was written by David Pesetsky of MIT,
Andrew Nevins, then at Harvard (now University College London), and
Cilene Rodrigues, then at Emmanuel College, Boston (now the Pontifical
Catholic University of Rio de Janeiro). I will refer to this trio
as NP\&R.

NP\&R's paper \citep{NevPesRod09a} contains lengthy discussion of a
topic about which I will say hardly anything: the extent to which,
and the ways in which, culture can influence grammar. Everett holds
that a single feature of Pirah{\~a} cultural life -- their focus on
immediate experience rather than remote considerations like the distant
past, the far future, or the abstractions of mathematics or philosophy
-- predicts a whole slew of properties of their language. I doubt it,
as do NP\&R. But it is not their disagreeing with Everett that I will
be concerned with here. In Section 3 I will turn to the rather meager
results of their search for false syntactic claims in \citet{Everett05},
but first I review some of the ancillary actions they and others took,
and the way they instigated and promoted a remarkably vicious attack
on Everett's character and integrity in the years that followed.
I will survey the events only briefly in the next section, without
attempting to be exhaustive.

\section{Character assassination and career disruption}\label{war}

The obvious course of action for linguists who felt Everett's
\textit{CA} paper must be mistaken would have been to engage with
him collaboratively to find out more about relevant properties of the
Pirah{\~a} language. This was not the path chosen by NP\&R. Their
paper was written without contact with either Everett or anyone else
who knew the Pirah{\~a} language. This made it wholly an exercise in
textual exegesis. And it did not stop at addressing factual claims;
from the start it employed thinly veiled inferences and accusations
of prejudice, dishonesty, and even research misconduct.

The suggestion NP\&R made was in essence that Everett's early
descriptive writings on Pirah{\~a} did offer evidence of subordinate
clauses (along with various other things like numerals, quantifiers,
and color names), so his 2005 position was a suspiciously unsupported
and possibly mendacious retraction of earlier views.

Despite mentioning the idea that HCF had only ever intended a weak
claim about phrases containing other phrases (pp.\,366--67, fn.\,11),
NP\&R only made that point in passing; their central aim was to argue
that in 2005 Everett was telling lies about \textsc{clausal} embedding,
and that one could learn this by simply looking at his work of a
quarter-century before, where he did tell the truth. In the refereed
paper they published in \textit{Language} (\citeyear{NevPesRod09a})
they could only adumbrate the claim of dishonesty, but in less
constrained channels they and others were less guarded: emails, tweets,
blogs, remarks to journalists, and posts on Facebook can slip the
surly bonds of scholarly decency.

The attack mounted by NP\&R, and taken up by other anti-Everett
linguists, was not the worst that a social scientist ever suffered;
the libeling of anthropologist Napoleon Chagnon and geneticist James
Neel by Patrick Tierney (\citeyear{Tierney00}) was surely
worse.\footnote{\label{tierney}
   Tierney falsely alleged that Chagnon and Neel had deliberately
   exacerbated a fatal measles epidemic among the Yanomam{\"o} people
   in pursuit of some kind of eugenics experiment. For a time
   anthropologists Leslie Sponsel and Terence Turner persuaded the
   American Anthropological Association to support these charges and
   condemn Chagnon and Neel. See \citealt{Dreger11} for detailed
   research on the whole sordid story of this affair, and a vindication
   of Chagnon and Neel. Tierney is now regarded as totally discredited.}
But the trashing of Daniel Everett runs a fair second for nastiness.

Tom Bartlett of \textit{The Chronicle of Higher Education} heard
about it from linguists that he interviewed in 2012. His account of
linguists' behavior \citep{Bartlett12} is not edifying, but fully
accords with my knowledge and experience of the events. He speaks
of a linguistics discipline ``populated by a deeply factionalized
group of scholars who can't agree on what they're arguing about
and who tend to dismiss their opponents as morons or frauds or both.''
Other disciplines have disputes too, he admits, but even so,
``linguists seem uncommonly hostile.'' If anything, Bartlett somewhat
understated things; the following subsections refer to documentable
incidents that he did not even mention.

\subsection{The BCS lecture}\label{river}

In the fall of 2006 Professor Edward Gibson arranged for Daniel
Everett to give a lecture on Pirah{\~a} syntax in the Brain and
Cognitive Sciences department (BCS) at MIT. David Pesetsky, of MIT's
Department of Linguistics and Philosophy, contacted Gibson by email.
Details of the interaction are disputed,\footnote{\label{gibson}
   Pesetsky asked Gibson to assure him that he was not forwarding the
   email exchange to anyone else, and Gibson gave that assurance.
   Gibson has since honored Pesetsky's wish to keep his emails private.
   When I asked Pesetsky to show me the emails, he refused, so I have
   only Gibson's broad paraphrase of them as my source.}
but Gibson reports Pesetsky as apparently thinking that Everett held
reprehensible views about the Pirah{\~a} people, mentioning a claim
that the Pirah{\~a} talk like chickens and act like monkeys.
Gibson knew the latter remark.  It was from a page headed
``Pirah{\~a}: The People'' on the University of Pittsburgh
website,\footnote{\label{pittsburgh}
  In 2007 it was still accessible at
  \url{http://amazonling.linguist.pitt.edu/people.html} but it did
  not survive Everett's subsequent moves to other universities and
  seems not to have been preserved by the Wayback Machine archiving
  site.}
and reported a contemptuous remark by Brazilian merchants who traveled
the Maici river and occasionally traded with men from Pirah{\~a}
villages. Everett wrote: ``The local traders say they `talk
like chickens and act like monkeys'.'' He was quoting, not endorsing
the characterization; he despised the ignorance of the people who
repeated the saying. Gibson pointed out that an unendorsed direct
quotation entailed nothing about Everett's views, but when the first
draft of NP\&R's paper was circulated about three months
later,\footnote{\label{firstrelease}
   LingBuzz, 8 March 2007,
   \url{https://ling.auf.net/lingbuzz/000411/v1.pdf?_s=AES_1bvQN0ZRFPhy}}
it contained a statement that the authors felt a ``general discomfort
with the overall presentation of Pirahã language and culture'' that
Everett gave, and in a footnote (p.\,51, fn.\,74) it repeated the
quote from the river traders.

The extent of NP\&R's hostility to Everett's views and suspicion about
his relations with indigenous Brazilians became much more explicit
on Tuesday 28 November 2006, when Gibson sent out a formal announcement
of Everett's lecture to the mailing lists for linguists and BCS people
at MIT and Harvard. Immediately Andrew Nevins (who had never met
Everett, and refused when Gibson later suggested a meeting) sent out
a scathing email from his Harvard account to the same lists about the
expected content of the talk.\footnote{\label{recipient}
   At the time I had a Radcliffe Institute email address that David
   Pesetsky had kindly added to the MIT visitors' email list to keep
   me informed about colloquia during a sabbatical at Harvard, so I
   was an accidental recipient of Nevins's email. He had tried to reach
   the MIT Brain and Cognitive Sciences list as well as the lists for
   the two linguistics departments, but found it closed to external
   senders.}
The subject line was ``enough is enough'' and it opened by saying:
\begin{quote}
\small\raggedright
\texttt{although david, cilene and i are working on a paper about the
linguistic features of piraha, i thought some of you should see
some of the more obvious counterexamples to everett's cultural
claims before his talk at mit on friday, especially since we may
not be allowed to ask questions without being cut off.}
\end{quote}
He then gave a link to Everett's ``Pirah{\~a}: the people'' and said:
``have a look at this archived web page from just over 6 years ago.
Did the Piraha change since then, or did Everett?'' -- an indication
that NP\&R were going to try to show that Everett was not just wrong,
he was lying about facts he had previously acknowledged. After giving
a few links to Brazilian anthropological literature,  Nevins ended
with a sarcastic parody of advertising copy:
\begin{quote}
\small\raggedright\texttt{You, too, can enjoy the spotlight of mass media
and closet exoticists! Just find a remote tribe and exploit them
for your own fame by making claims nobody will bother to check!}
\end{quote}

This struck me as like an intrusion into linguistic science of the
sort of attack ads typically seen in political election campaigns.
I commented on it in a discussion of the issue on Language Log
the next day,\footnote{\label{languagelog}
   `Fear and loathing on Massachusetts Avenue,' on Language Log,
   29 November 2006, online at
   \url{http://itre.cis.upenn.edu/~myl/languagelog/archives/003837.html}}
speculating on whether the attack might be motivated by a combination of
Chomskyan orthodoxy, liberal hypersensitivity regarding ethnic minorities,
and academic prejudice against missionaries.

The talk attracted a large audience. Nevins, Pesetsky, and Rodrigues
all attended, and so did Marc Hauser, the lead author of HCF. Hauser
was well acquainted with Nevins, who regularly attended Hauser's lab
meetings at the time. Ironically, seven months after Nevins's email
about ``claims nobody will bother to check,'' Harvard investigators
began to check some of Hauser's claims about primate behavior, and
within four years he had been found responsible for serious research
misconduct and had lost his professorship and quit
academia.\footnote{\label{misconduct}
   In July 2007 investigators entered Hauser's lab while he was away,
   seizing computers, video records, and documents. By August 2010
   they had found him solely responsible for ``eight instances of
   scientific misconduct,'' including ``problems involving data
   acquisition, data analysis, data retention, and the reporting of
   research methodologies and results.'' After a year's leave of
   absence, Hauser learned that he would not be allowed to return
   to teaching at Harvard, or maintain a laboratory, or apply for grants.
   He resigned effective 1~August 2011. Later a separate investigation by
   the federal government's Office of Research Integrity found in September
   2012 that he had fabricated data, manipulated results, and wrongly
   described experiments supported by several federal grants (see DHSS
   notice 77\,FR\,54917, 09/06/2012). \citet{Gross11} provides a detailed
   discussion of the Harvard investigation and its aftermath.}

\subsection{Refusal of research permits}

In 2007, Everett received an unexpected phone call from the distinguished
journalist Larry Rohter, who had been South American bureau chief for
\textit{The New York Times} since 1999. Rohter was in the office of
the director (\textit{presidente}) of FUNAI
(Funda{\c{c}}{\~a}o Nacional do {\'I}ndio, later renamed
Funda{\c{c}}{\~a}o Nacional dos Povos Ind{\'\i}genas), the Brazilian
government agency charged with overseeing the welfare and protection of
the country's indigenous people.
He had in his hands a letter written to FUNAI by Cilene Rodrigues.
Rohter read the Portuguese text to Everett over the phone.

The letter expressed objections to Everett's linguistic research
and his representation of Pirah{\~a} culture.  It may also have
expressed the view that he was not a suitable person to be permitted
to work with Brazilian Indians. I have not seen the letter, and
Rodrigues did not respond when I asked her for a copy of it, but
Rodrigues's role in the interaction with FUNAI is confirmed in an
article in \textit{The New York Times},\footnote{\label{schuessler}
   ``How Do You Say `Disagreement' in Pirah{\~a}?'' by Jennifer
   Schuessler, \textit{The New York Times}, 21 March 2012.}
which reports that ``She declined to elaborate on the contents of the
letter, which she said was written at Funai's request and did not
recommend any particular course of action,'' and that ``asked about
her overall opinion of Dr. Everett’s research, she said, `It does not
meet the standards of scientific evidence in our field'.''

A few years earlier, Napoleon Chagnon's enemies had managed to persuade
FUNAI to deny him permission to visit the Yanomam{\"o} people in Brazil
(see \citealt{Dreger11}).
Something similar now appeared to happen to Everett. The next time he
applied for permission to bring some researchers to the Pirah{\~a}
territory (which, ironically, he had originally assisted FUNAI in
demarcating in order to protect the Pirah{\~a}s' right to their
land), he found that he was denied. He was later able to get permission
from the local FUNAI office to visit the area merely as an aide and
interpreter to a film team during the making of the 2012 documentary
film \textit{The Grammar of Happiness},\footnote{\label{happiness}
   On YouTube at \url{https://www.youtube.com/watch?v=5NyB4fIZHeU} and
   also via SLICE at \url{https://www.youtube.com/watch?v=_LAR6eeiVtY}}
but his applications to do grant-supported field research on the language
met with negative decisions.

Everett flew to Bras{\'\i}lia to discuss the situation, accompanied by
the doyen of Amazonian research, the late Aryon Rodrigues (1925--2014),
who had been a mentor to him during his doctoral studies. They had set
up a meeting with the national director of FUNAI, M{\'a}rcio Meira, but
Meira did not show up. Instead he sent a deputy had no power to make
executive decisions. Everett was thus cut off from visiting the people
he had known intimately for more than thirty
years.\footnote{\label{residence}
   Everett lived in Pirah{\~a} villages for 10 days in 1977; 3 weeks
   in 1978; 6 weeks in 1979; 8 months in 1980; 4 months each year
   from 1981 to 1985; a total of 12 months during 1986--1988; a total
   of 36 months during 1989--1999; 20 months during 1999-2001; and
   three months during 2001--2009, a total of just over 100 months.}
Among other things, this was a material loss for the Pirah{\~a},
because every time Everett arrived in their village he would bring
medicines and other valued items.

\subsection{Chomsky's ``charlatan'' insult}

In early 2009 Noam Chomsky was interviewed about the dispute by
\textit{Folha de S.~Paulo}, the the largest-circulation newspaper in
Brazil, and with evident irritation he told the interviewer (see the
issue of 1 February 2009):
\begin{quote}
Ele virou um charlat{\~a}o puro, embora costumava ser um bom linguista
descritivo. {\'E} por isso que, at{\'e} onde eu sei, todos os linguistos
s{\'e}rios que trabalham com linguas brasilieiras ignoram-no.

[``He became a pure charlatan, although he used to be a good descriptive
linguist. That is why, as far as I know, all the serious linguists
who work on Brazilian languages ignore him.'']
\end{quote}

The petty abuse of the first sentence is followed by a piece of
dishonesty: since Chomsky has never worked on Brazilian indigenous
languages and has never discussed any detailed work by those who have,
he has no knowledge of the wider community of Amazonianists (many of
them missionaries, others secular linguists or anthropologists in a
variety of universities in Europe, Australia, and the Americas),
and therefore has no grounds for assessing Everett's standing among
Amazonianists. The truth is that Everett's expertise has never been
questioned by the linguists with whom he has worked, or by any of the
roughly twenty researchers who have spent time with him among the
Pirah{\~a} to do research, or by any of the few outsiders who (like
Steven Sheldon) have actually made progress on learning the
Pirah{\~a} language.\footnote{
   Chomsky had perhaps forgotten that Everett had mentioned the
   lack of syntactic embedding in Pirah{\~a} during a personal
   conversation with him at MIT 25 years earlier; see Everett
   (\citeyear{Everett07}:12, fn.\,7).  I return to this briefly
   in section~\ref{prehistory} below.}

Chomsky continued with a clearly unverifiable claim about Everett's
private thoughts and hopes:
\begin{quote}
Everett espera que os leitores n{\~a}o entendam a deferença entre a GU no
sentido t{\'e}cnico (a teoria do componente gen{\'e}tico da linguagem
humana) e no sentido informal, que dis respeito {\`a}s propriedades comuns
a todas as l{\'\i}nguas.

\noindent
[``Everett hopes that the readers do not understand the difference between
UG in the technical sense (the theory of the genetic component of human
language) and the informal sense, which concerns properties common to all
languages.'']
\end{quote}
Chomsky is alluding to his reinterpretation of HCL's ``recursion'' claims
as having never been about languages, but only about the genetically
transmitted human ability to acquire language. He is claiming that
Everett wanted to fool \textit{CA} readers into paying attention to
sentence structure when really he knew the focus should have been on
genetics and neurophysiology.

But HCF never provided any genetic or neurophysiological facts about the
human language capacity that Everett could have focused on. As Everett
noted in a response to NP\&R, if the ``genetic component'' is the issue
on the table, then Chomsky's claim seems virtually empty: humans simply
have whatever special thing it is that permits them to acquire and use
language (see \citealt{Everett09}:439). Since he was motivated by what
HCL actually said (``There is no longest sentence,'' etc.), he concentrated
on ``properties common to all languages.'' That isn't charlatanry.

\subsection{Rodrigues's overt accusation of racism}

Later in 2009, Rodrigues increased the rhetorical temperature some
more. She explicitly alleged in a magazine interview with the German
journalist Malte Henk that Everett held racist beliefs: ``Everett ist
ein Rassist.  Er stellt die Pirah{\~a} auf eine Stufe mit Primaten''
[``Everett is a racist.  He puts the Pirah{\~a} on a level with
primates''].\footnote{
   \textit{GEO} magazine (Gruner\,+\,Jahr, Hamburg, Germany),
   January 2010, p.\,59.}
%  \url{https://www.yumpu.com/de/document/read/3590148/musterstrecke-geo-stand10506-dan-everett-books}}
By ``primates'' she clearly means apes and monkeys, unless she has
forgotten that all humans are primates.\footnote{\label{denial}
   In an email to Everett, Rodrigues denied ever making the statement,
   but Malte Henk stands by his claim about what she said to him on the
   record; see Everett (\citeyear{Everett13}:13).}

As \citet{Bartlett12} remarks, ``When you read Everett’s two books about
the Pirah{\~a}, it is nearly impossible to think that he believes they
are inferior. In fact, he goes to great lengths not to condescend.''
He does indeed.  He stresses their sharp intelligence, ingenuity, strong
group identity, rich social life, and ability to grasp complex discourse.
He lived with them, hunted with them, raised his three children among
them, talked with them endlessly, and learned from them during periods
of residence totaling well over eight years. His many accounts of
interaction with them (most engagingly in \citealt{Everett08}) often
evince admiration, and never for a moment suggest he sees them as
racially inferior beings.

But accusations of racism are potent weapons in contemporary
intellectual and political debate, whether grounded or not --
more powerful than any points about syntactic analysis could be.

\subsection{The fraud libels}

While working on his \citeyear{Bartlett12} article, Tom Bartlett asked
Nevins for some comments on the war on Everett. Nevins refused to be
interviewed, but emailed back: ``it seems you've already analyzed this
kind of case!'' -- appending a link to an earlier Bartlett story about
Diederik Stapel.

The implied defamatory claim here is extreme. Stapel is famously an
admitted fraudster. He voluntarily returned his PhD certificate to
the University of Amsterdam because he acknowledged that his scientific
misconduct had been ``inconsistent with the duties associated with the
doctorate.'' So far 58 of his papers in social psychology have been
retracted on grounds that the data were either manipulated or -- in
at least 30 cases -- simply invented out of thin air. Stapel would
invent whole tables of data with no empirical basis at all, and
published many reports of experimental studies that were never
conducted. Nevins is equating Everett's eight years of immersive
fieldwork and data analysis with the proven scientific misconduct of
a man described in \textit{The New York Times} (26 April 2013) as
``the biggest con man in academic science.''

At the time Nevins sent his message to Bartlett, Everett was a dean
at Bentley University and happened to be chairing an investigation
into allegations against a professor of accounting: Professor James
E.\ Hunton, who ultimately resigned in December 2012. By 2016 at
least 37 of Hunton's papers had been retracted under suspicions of
wholesale invention of data and publishing reports of studies that
had never been conducted.\footnote{\label{retraction}
   See \textit{Retraction Watch},
   \url{https://retractionwatch.com/2016/05/12/former-accounting-prof-adds-4-more-retractions-total-exceeds-37/}}
Bentley, therefore, had a well-functioning procedure for dealing with
research misconduct, which could have been used against Everett if
anyone had come up with a scintilla of evidence about fraud or or other
research misconduct.

Tom Roeper of the University of Massachusetts, Amherst, also directly
and publicly accused Everett of fraud. Speaking about Everett on camera
to the makers of \textit{The Grammar of Happiness}, he said: ``I think he
knows he's wrong, that's what I really think.'' With a knowing smile, he
added: ``I think it's a move that many, many intellectuals make to get a
little bit of attention.''\footnote{\label{roeper}
   For a bookmarked location of Roeper's remark in the SLICE release
   of the film, retitled as ``Decoding Amazon: life of the Pirah{\~a},''
   go to \url{https://youtu.be/_LAR6eeiVtY?t=1323}}
Roeper's claim is not just that Everett is wrong, but that he
\textsc{knows} he's wrong, and is telling lies ``to get a little bit
of attention.''

\subsection{Illegality accusations}

In Brazil, the allegations started to reach further than simply positing
dishonesty. Rumors were spread that for decades Everett had been working
illegally, never obtaining the required permits for working in
Indian areas. Denny Moore, an American linguist resident in Brazil,
made forceful allegations along these lines to me in personal conversation
and subsequent email (May 2019) and made further remarks on the topic
in a Facebook comment in January 2024.

The suggestion that Everett had never complied with the full legal
requirements is implausible on its face, because if it were true then
his failure to obtain a FUNAI permit after Rodrigues's letter of 2007
would have been of no importance. Everett arrived in Brazil in 1977
and was granted permanent resident status under an agreement between
the Brazilian government and the Summer Institute of Linguistics
(SIL), so he can visit the country without a visa whenever he wishes.
But doing research on the Pirah{\~a} reservation without a FUNAI
permit would be illegal. The only reason Everett has not been able
to do any field research among the Pirah{\~a} since 2009 is that he
strictly respects the law -- as one would expect, given the crucial
necessity for him to have access to indigenous Amazonian areas.

In 1977 all SIL missionaries were allowed to live among indigenous
populations (Desmond Derbyshire had been with the Hixkaryana under
such terms since 1955 when I met him). In 1978 the government canceled
the contract with SIL and all missionaries had to leave indigenous
lands. At that point Everett became a graduate student at the State
University of Campinas (UNICAMP), and in that capacity, with the help
of Aryon Rodrigues, he received written authorization from the director
of FUNAI to return to the area, and spent a year living in a Pirah{\~a}
village with his American wife Keren (now Keren Madora) and three
children -- not a visit that could have been accomplished furtively.

Eventually FUNAI reached an understanding with SIL that allowed all
of its members to continue working in indigenous villages, not as
missionaries but in order to do linguistic research and translate
morally uplifting works into indigenous languages. That blanket
permission for SIL members covered Everett after he completed the PhD
at UNICAMP, until 2001. During that period he never needed to fill
out the permit application forms used by university academics, whether
Brazilian or foreign, which is why (as suspicious Brazilian researchers
have found) searches in the public record for his applications via
that channel come up with no results.

In 2001 Everett left SIL. Since then, when doing grant-supported research
as a faculty member at the University of Pittsburgh (1988--1999) or
the University of Manchester (2001--2006), he has entered the country
on the basis of his permanent resident status (contrary to some
allegations, he has never entered Brazil on a tourist visa), and he
obtained permission for visits to indigenous areas through
close contacts with FUNAI.

There are different ways for permanent residents to work: they can apply
to the national office of FUNAI, or go through a local FUNAI office in
the appropriate region provided Bras{\'\i}lia does not object. They can
also visit at the request of an indigenous group, which FUNAI is required
to accept. One way or another, Everett has always had the needed permits,
and two national-level directors of FUNAI (including the much-respected
Apoena Meirelles) visited Everett while he lived with the Pirah{\~a},
which would hardly have happened if he was an illegal foreign
interloper. He has a letter from FUNAI thanking him for his work,
and a short article praising his work
appeared in a magazine in 2012\footnote{\label{caetano}
Marcelo Moraes Caetano, ``Indagado pelos Pirah{\~a},'' \textit{Revista
da Cultura} 61, August 2012, p.\,33.}
and was archived on the FUNAI website.

There was an occasion in 2007 when Everett was with the Pirah{\~a}
along with several students and a local FUNAI official with a grudge
against him reported that they were there illegally. A heavily armed
team of military police made the long river journey through a rainstorm
to get to the relevant Pirah{\~a} village and arrest him. Everett greeted
them in fluent Portuguese, showed them his permanent residence document
and his letter from the local FUNAI office. The policemen relaxed, and
posed smiling for a photo with members of Everett's team.
A few days later in Porto Velho, he was called in by the FUNAI office
there over the same incident, and again satisfied the organization that
he had done everything legally.

Everett is not and never has been the subject of any civil suit or
criminal indictment for illegal presence in an indigenous area. Yet
allegations that he is a notorious lawbreaker continue to be spread
by linguists in Brazil. The strong antipathy felt by many Brazilian
academics to North American missionaries may be partly to blame, since
Everett is still thought of as associated with that role, more than
two decades after he left SIL.

\subsection{The Nevins/Carvalho/R{\"o}ssler video}

A conference was held at the Federal University of Rio de Janeiro in
2013 that was devoted entirely to work arguing that Everett was wrong.
Everett heard about the planning for it, and offered to attend the
conference at his own expense, but he was told he would not be welcome.
During the same period (August 2013) Nevins took the opportunity to
work with Emerson Carvalho and Eva-Maria R{\"o}ssler to produce a
video\footnote{\label{augustovideo}
   Online since 2013 at
   \url{https://www.youtube.com/watch?v=J3jWI4cPRMg}}
which seems to have the primary purpose of further damaging Everett's
reputation. It is represented as an interview with two representatives
of ``the leadership'' of the Pirah{\~a} (in truth they live an
anarchist socio-political life with no political leaders). The main
speaker throughout the video is Jose Augusto Diarroi, nicknamed
``Ver{\~a}o'' by Portuguese speakers because of his SIL contacts
(\textit{ver{\~a}o} means ``summer''), who falsely represents himself
as member of the Pirah{\~a} community. His father was Pirah{\~a}, but
his mother was not, and he was raised elsewhere, never acquiring more
than a smattering of the Pirah{\~a} language. Sitting beside him is
a native Pirah{\~a} speaker whose name is given as Yapohen (not a
possible Pirah{\~a} name) but is actually Hiaho{\'a}i. Very few
Pirah{\~a} utterances are heard in the entire interview, and none are
glossed in the subtitles.

Augusto tells tales about Everett engaging in activities seemingly
drawn from the worst stereotypical charges against bad missionaries,
claiming that Everett had terrorized the people he lived among,
threatening them that God would kill them all if they did not come
to Jesus and convert to being ``true believers,'' and so on. Nevins's
voice can be heard saying things like ``Wow!'' from time to time.
If any of what he says were true, Augusto would not be one to tell
about it, because he never lived in a Pirah{\~a} village during any
time when Everett was there.

At certain points Augusto attempts to elicit some contributions from
Hiaho{\'a}i, who is visibly reluctant to speak, and says nothing for
a long time. When he is eventually prompted to say a few things in
Pirah{\~a}, Augusto pretends quite unconvincingly to translate them,
turning a few seconds of Pirah{\~a} into several minutes of Portuguese.
What he represents as translations are total fabrications. A version
of the video with transcription supertitles of the Pirah{\~a} utterances
was uploaded by Miguel Salinas in 2019.\footnote{\label{salinasvideo}
   Online at \url{https://www.youtube.com/watch?v=xeEAufXg8fc}}
See Everett and Gibson (\citeyear{EverGibs19}:781, fn.\,3) for brief
discussion of some of this video, with examples of the mistranslations.

\subsection{Cancelation at Oxford}
\label{sec-Oxford-cancelation}

The work that NP\&R have put into representing Everett as a
disreputable person and untrustworthy scholar has not had significant
material effects on his career: he has served successfully as a
department head, dean of arts and sciences, and acting provost, and
unlike Hauser or Hunton he remains a tenured full professor to this
day. Nevertheless, NP\&R have created a kind of folklore, a vague
shadow of disrepute, which continues to have effects. Mud sticks, if
you throw enough of it. One of Everett's daughters reports having met
people in Brazil who say, ``Oh, you're the daughter of that racist
guy.''\footnote{\label{newscientist}
   Interview with Liz Else and Lucy Middleton, \textit{New Scientist},
   19 January 2008, p.\,44.}
And substantive professional consequences do result from this atmosphere
of negativity.

For example, on 12~March 2017 Everett offered to give a talk to the
linguists at the University of Oxford the following September -- at
no cost to Oxford because he was planning to visit the UK anyway. The
planned lecture was not to have been about Pirah{\~a} syntax,
incidentally, but about paleoanthropology and the emergence of language
in early humans. His offer was greeted with enthusiasm by the head
of the linguistics faculty, Professor Aditi Lahiri, who promptly let
her colleagues know the good news. But within hours her acceptance
was withdrawn in a rather awkward email message.

The next day Everett learned the reason: two junior faculty had
objected by email as soon as they learned of the tentative plan,
citing potential ``reputational damage'' to Oxford if Everett were
to speak there.\footnote{\label{yorick}
   This was reported to Everett by the late Yorick Wilks (1939--2023)
   in an email, 13 March 2017, which I have seen. Wilks stated that
   he had seen the objectors' emails but did not name them.}
It is hard to believe someone would think a visiting speaker could
be so toxic that his mere appearance would inflict reputational
damage on Britain's oldest university, often ranked number one in
the world. But this is the sort of strange fruit the long campaign
against Everett has borne.

\subsection{The double review of \textit{Recursion Across Domains}}

The conference in Rio de Janeiro in 2013 resulted in a book entitled
\textit{Recursion Across Domains} \citep{AmMaNeRo18}. The central aim
of the conference and the book was to publish studies saying Everett
was wrong, and he was never invited to submit a reply to its
criticisms. But the editors of the Linguistic Society of America's
journal \textit{Language} invited Everett together with his
collaborator Edward Gibson to write a review of the book (it appeared
as \citealt{EverGibs19}). When this became known to Everett's
opponents, the editors promptly came under pressure to alter their
decision. After some consultation they made the unprecedented decision
to give the book two review articles in the same issue. Several
potential reviewers who were thought likely to take a more anti-Everett
and pro-Chomsky line were sounded out but declined. Finally Norbert
Hornstein agreed to take on the task.

Hornstein (\citeyear{Hornstein19}) admitted with admirable frankness
(p.\,791) that he knows nothing at all about the empirical content
of the book -- topics like the syntax of South American languages
and experimental developmental psycholinguistics. In fact he says:
``Facts usually make me itchy\ldots\ My allergies will lead me to pass
lightly over many of the specific empirical findings in what follows.''
His main qualification was clearly that he could be relied upon to
support the Chomskyan line, and that he did. (See Section~\ref{ppsection}
below for a discussion of one chapter from the book that Hornstein
naively accepted as sound.)

Further pressure on the editors of \textit{Language} induced them to do
one additional thing regarding the same book that as far as I can see was
unprecedented: \textit{Language} (like most scholarly journals) does not
publish aggrieved responses to book reviews submitted by authors whose
work is criticized.
But Cilene Rodrigues sent in a letter of protest about the Everett and
Gibson review, which had said that her work did not exhibit ``high
scientific standards.'' The editor (Andries Coetzee) initially resisted
the idea of publishing it (and told Everett and Gibson that it would
not be published without their having right of reply), but he was eventually
persuaded to print it, and it appeared in \textit{Language} 96.2
(2020), 221--223, without a reply. A short editorial clarification
concerning one sentence in the Everett and Gibson review was also printed.
Thus \textit{Recursion Across Domains} ended up being the subject of
four different items in the pages of \textit{Language} when the usual
maximum for any book is one.

\subsection{Recent literature overviews}

The work NP\&R have done to damage Everett's reputation has been ample
to color the general impression a newcomer to the dispute will pick up.
The superbly detailed survey of Amazonian languages by \citet{Aikhenvald12}
takes the line of treating the issues as unfit for discussion, declaring
that ``there is neither consistency nor plausibility to the quasi-analytical
statements which have been made concerning this language [Pirah{\~a}], or
its culture, during the past fifteen years. I refrain from quoting these
sources'' (p.\,411, n.\,91). She thus avoids any discussion of the
polemics of the post-2005 literature. In fact she cites nothing on
Pirah{\~a} dated later than 1986.

Janet Chernela, an anthropologist specializing in Amazonia, recently
tried to survey the whole dispute in an article for \textit{Annual
Review of Anthropology} \citep{Chernela23}. She seems to think she
has provided a balanced summary, but her treatment of the relevant
literature is hopelessly skewed against Everett. She never even
mentions the existence of \textit{Handbook of Amazonian Languages},
and hence never refers to \citet{Everett86HAL}, unquestionably the
most important descriptive document in the whole dispute. She cites
\citet{NevPesRod09a} without ever mentioning that it was followed by
a detailed response \citep{Everett09} in the same issue of
\textit{Language}, nor the rebuttal to that by \citet{NevPesRod09b},
nor the final rejoinder to that by \citet{Everett13}. She very
briefly mentions the incompetently uncritical review article by
\citet{Hornstein19}, but seems unaware of the vastly more expert
critical one by \citet{EverGibs19}.

Admittedly, reading all of the post-2005 work just cited would be
an exhausting business -- anyone who doesn't come out of reading it
feeling dazed and confused just hasn't been paying attention. But
the skewing of Chernela's coverage is quite extraordinary. It is
possible that she fell victim of a major downside to accessing
literature online: anyone who had \textit{Language} 85 no.\,2
in their hands could not fail to see that \citet{NevPesRod09a}
is immediately followed by Everett's 37-page response, but if
Chernela simply heard about the former and downloaded a PDF of it
she might well have had no idea the latter existed.

However, she has less excuse in the matter of the two reviews.
She cites \citet{Hornstein19} in connection with Chomsky's claim
that ``variation between languages -- while possibly interesting
for other purposes -- is irrelevant to the nature of the FLN'' (p.\,140).
But its first page carried an editor's footnote explaining that
``This issue of \textit{Language} contains two review articles
focusing on the volume \textit{Recursion Across Domains},''
and adding: ``Since the topic of this volume (recursion) is one of
central interest (and some controversy) in current linguistic theory,
we thought it important to publish reviews from scholars who will
bring differing perspectives to the topic,'' and so on.  Those
differing perspectives do not come through in Chernela's account.

She makes some patently erroneous and unfounded claims, like that
NP\&R ``reanalyzed data collected among the Pirah{\~a} by Everett's
predecessors''(p.\,140). NP\&R did nothing of the sort, and do not
try to represent themselves as having done it. Steven Sheldon, whose
residence among the Pirah{\~a} antedated Everett's, did produce some
transcribed texts, which are utilized by \citet{FutrellEtAl16},
but NP\&R appear not to have known about them. NP\&R
(\citeyear{NevPesRod09a}:391) do cite a table of six pronoun forms
from a paper by Sheldon, but the paper \citep{Sheldon88} appeared two
years after Everett's main descriptive work on the language was in
print.

In another inexplicable piece of invention, Chernela asserts
that ``Much of Everett's field methodologies involved structured
interviews using a recorder'' (p.\,143), and she asserts that his
work ``flies in the face of Boasian anthropology'' because it fails
to ``interpret cultures and languages on the basis of each society's
own logic and values rather than through a universal yardstick'' and
``understand language as a social phenomenon in which meanings cannot
be understood apart from context.'' But Everett's work involved
interacting more closely with the community than any other outsider
has ever done or was ever competent to do. He lived in the community
and participated in its life for eight years. His children became
fluent in the language and often played with Pirah{\~a} children
all day.  He constantly strived in his work to ``interpret cultures
and languages on the basis of each society's own logic and values.''
Throughout \citet{Everett12} it is clear that language is being seen
as intimately linked to culture, and Boas is copiously discussed in
\citet{Everett16}.\footnote{\label{chernela}
   Chernela mentions the existence of both these books (p.\,144), but
   only in passing, and she misstates the title of the first.}
Like NP\&R, Chernela never met Everett or even emailed him.
She seems to have decided up front that he was to be her representative
of the typical desk linguist asking elicitation questions, not
the sensitive anthropological investigator attuned to culture, values,
and meaning.

The general pall of negativity that has been cast over Everett's work
may be responsible for some of Chernela's bias. Like NP\&R, she worked
without any contact with Everett or anyone else who had ever lived with
the Pirah{\~a} and learned their language. It was an anthropologist,
Bambi Schieffelin, who suggested to Chernela that she might write the
article, and neither of the two people thanked in her acknowledgment
note for reading the paper in draft (p.\,146) is a linguist. She does
no linguistic analysis; she simply browsed some of the recent
literature and came away with the broadly negative view of Everett's
work that NP\&R were intent on establishing as the default.

The end result is not too surprising given the intellectual climate
that the campaign of hostilities created. Linguists should be ashamed of
this ghastly parody of science, with its rumors of racism substituting
for scientific discussion, and career sabotage replacing rational
criticism. But what makes things worse is that it was under-informed
from the start. To see why Everett in the early 1980s was trying to
provide evidence of subordination in Pirah{\~a}, we need to look at
certain events predating all of his descriptive work, but the
digression is a relevant one.

\section{Overlooked prehistory}\label{prehistory}

In 1975, Daniel Everett was 24 and had just completed a Diploma
in Foreign Missions from the Moody Bible Institute in Chicago. He and
his wife were making plans to enter service as missionaries and bible
translators for the Summer Institute of Linguistics (SIL) in South
America.

Four thousand miles away, I was a 30-year-old lecturer in linguistics,
completing my first year at University College London. I had spent
1973--74 at King's College, Cambridge, learning typology from Ed
Keenan and Bernard Comrie, and spent the summer of 1974 at the LSA
Linguistic Institute at U Mass Amherst learning from Chomsky, Halle,
Keyser, Perlmutter, and Postal.

In 1976, barely done with writing my PhD dissertation on rule
interaction in classical transformational grammar, I was asked if
I would take on the supervision of a prospective PhD student: a
54-year-old SIL missionary named Desmond Cyril Derbyshire. He had had
no college degree; before he became a missionary he had been a
chartered accountant in Durham, England. I'm not sure whether my
senior colleagues believed the work of a middle-aged missionary would
amount to much, but fortunately for me they allowed him to enroll,
and I agreed to be his de facto advisor (de facto because the
university did not allow someone of my lowly rank to be a doctoral
supervisor). He turned out to be perhaps the finest scholar I ever
worked with.

\subsection{Discovering Amazonian languages}
\label{sec-discovering-amazonian}

By the time I met Derbyshire he had done nearly 20 years of work on
a Cariban language I had never heard of, spoken on a northern tributary
of the Amazon. In a lecture on constituent-order typology I presented
arguments (set out a in then-forthcoming article, \citealt{Pullum77})
that there was no convincing evidence for any language in the world
having an object-initial basic constituent order (OVS or OSV). The
only surface orders for the major constituents of the clause permitted
by universal grammar seemed to be SOV as in Hindi, SVO as in English,
VSO as in Irish, and VOS as in Malagasy. Derbyshire raised a hand
from the back row and reported that he had been working on a language
that he believed strongly preferred OVS as the order in transitive
clauses.

The language was Hixkaryana. We arranged to meet after class so that
I could learn something about its clausal syntax. Derbyshire had actually
published a preliminary study of it back in 1961, when I was in high
school \citep{Derbyshire61}, and it included a remark (using the
terminology of Kenneth Pike's largely forgotten tagmemics framework)
that ``the goal always precedes, and the actor usually follows, the
predicate tagmeme.'' In post-Greenberg terms, that meant OVS. But
there had been no discussion of this language in the subsequent
literature.

I gave Derbyshire some ideas on how he might confirm that he really
was dealing with an OVS language: there was the possibility that
(for example) Hixkaryana was just an SOV language in which the subject
was occasionally shifted to clause-final position in special discourse
contexts. There were substantial stocks of data to consult: a
collection of texts transcribed from native speakers and published
in Brazil ten years before; a Hixkaryana version of the entire New
Testament, checked throughout by native speaker consultants, in press
in Brasilia; and plentiful supplies of other data collected during
Derbyshire's twenty years of fieldwork, including a remarkable diary
privately composed by a native speaker who had learned to write the
language.

Text from all sources supported Derbyshire to the hilt. My belief
that universal grammar precluded object-initial basic constituent
orders was inescapably wrong. Hixkaryana was a rather rigid OVS
language: always OV, with auxiliary after the lexical verb, and
the subject clause-initial only infrequently, when specifically
focused or contrasted with something else (see
\citealt{Derbyshire85}:74).

Derbyshire and I began work on publicizing what appeared to be the then
new and surprising fact that there was definitely at least one clear
case of an OVS language. I worked with Derbyshire on preparing a squib
for publication in \textit{Linguistic Inquiry} \citep{Derbyshire77}.
And I suggested to him that his doctoral work might permit him to
also undertake a monograph for the Lingua Descriptive Series (LDS)
that was being planned by Bernard Comrie and Norval Smith.

The LDS monographs were required to adhere to a format carefully designed
to facilitate comparative research. The instructions for contributors
were published as a special issue of \textit{Lingua} (vol.~42, no.~1)
as the \textit{Lingua Descriptive Series Questionnaire}
(\citealt{ComrSmit77}, henceforth \textit{LDSQ}). It set out a systematic
section-numbering scheme for organizing descriptions in the series.

I showed Derbyshire my copy of \textit{LDSQ} as soon as I received it, and
he not only took up the task of writing an LDS monograph, but worked
efficiently enough to produce the inaugural one \citep{Derbyshire79},
a superb description which would have amply justified the award of a
PhD -- but in fact he also produced a distinct work to offer as his
PhD dissertation under the title \textit{Hixkaryana Syntax}, which
presented the description somewhat differently and added a second
part on typology and discourse syntax plus eleven appendices on
phonology and morphology (it was published later as
\citealt{Derbyshire85}).

The significance of \textit{LDSQ} to this story becomes clear in
the light of what its detailed instructions said about subordinate
clauses. It specified that Section 1.1.2 of the description was to
be headed ``Subordination.'' Subsection 1.1.2.1 was to state whether
there are ``any general markers of subordination, e.g.\ word order,
particles (in what position?), verb modification, etc.,'' and 1.1.2.2
was to cover ``Noun clauses'' -- the full finite subordinate clauses
that Jespersen calls content clauses. Section 1.1.2.2.3 was to deal
with declarative content clauses (``indirect statements''), 1.1.2.2.4
was to treat interrogative ones (``indirect questions''), and so on.
This had more significance than we then realized.

Derbyshire made some further visits to Brazil and began learning more about
what other SIL linguists had found. We began to pick up reports of
other OVS languages, plus one or two that seemed to be OSV. I~obtained
a grant from the UK Social Science Research Council to support Derbyshire's
work, not only on the syntax of Hixkaryana but also on these other
reported languages. I learned a lot about the history, geography,
ecology, and demography of Amazonia, and the appalling
treatment of its indigenous inhabitants, and together
Derbyshire and I prepared a paper entitled ``Object initial languages'' giving
brief accounts of a dozen object-initial languages (it was later
published in \textit{IJAL} as \citealt{DerbPull81}). This led to our
planning what became the four-volume \textit{Handbook of Amazonian
Languages} (\textit{HAL}).

The relevance of \textit{HAL} is, of course, that around 1983 or 1984
Derbyshire commissioned a chapter for it from the young Daniel Everett.
His grammatical overview of Pirah{\~a} became the longest chapter in
the first volume (\citealt{DerbPull86}, henceforth \textit{HAL\,1}).

Everett was by this time a PhD graduate of the Universidade Estadual
de Campinas in Brazil (the first linguistics PhD in the country),
with a dissertation on Pirah{\~a} grammar and syntactic theory.
Derbyshire was aware that Pirah{\~a} was a genetically isolated and
notoriously difficult language on which SIL had tried to make headway
for a quarter of a century. Two previous missionary linguists had
worked on it: Arlo Heinrichs, who did the difficult work of
establishing initial contact with the Pirah{\~a} and worked with them
from 1959 to 1967, publishing a preliminary view of the phonemes of
the language \citep{Heinrichs64}, and Steven Neil Sheldon, who worked
on the language from 1967 to 1976 and knows it fairly well. But Everett
and his then wife Keren were the first SIL members who learned to
speak and understand the language fluently. Everett's translation of
the \textit{Gospel of Mark} \citep{Everett86Mark} was the first piece
of bible translation SIL had ever achieved for the language.

To guide Everett and the other contributors of the grammatical sketches
in \textit{HAL}, Derbyshire and I had produced an analytical table
of contents, much briefer than the questionnaire for the LDS but
inspired by it. We reproduced it in \textit{HAL\,1}, pp.\,31--32.
And (the crucial point) Section~14 was to be headed ``Subordinate
clau\-ses.'' Everett had in fact already seen \textit{LDSQ} as soon as
it appeared, and was already assuming that he had to say things about
subordinate clauses.

It should not be too surprising, then, if Everett diligently strove
to find and exemplify subordinate clauses, looking for all the usual
grammatical furniture that speakers of European languages and
syntacticians at MIT would expect sentences to exhibit. NP\&R represent
it as suspicious that he would say in \citeyear{Everett83} and
\citeyear{Everett86HAL} that there were subordinate clauses and then
say in 2005 that there weren't. But he was effectively being directed
to say something about subordinate clauses by both of the two sets of
instructions he was using as guidance.

Looking back now, what surprises me is that Derbyshire and I did not rethink
our guidance, and change the question to ``Are subordinate clauses found
in the language?''; by the early 1980s we knew what Hixkaryana had
taught us about the topic of subordinate clauses. Derbyshire followed
\textit{LDSQ}'s directions closely, so linguists do not have to wonder
about what the subordinate clauses are like in any language with an
LDS monograph; they can just turn to Section 1.1.2 and find out. Here
is what Derbyshire says about Hixkaryana (p.21):

\begin{quote}
\small
\underline{\underline{\texttt{1.1.2. Subordination}}}\\[0.5ex]
\texttt{Subordination is restricted to nonfinite verbal forms,
specifically derived nominals (or, pseudo-nominals that function
as adverbials – see 1.1.2.2.6).}
\end{quote}

\noindent
Hixkaryana, then, had no content clauses at all. And turning to
Section 1.1.2.3, ``Adjective clauses (relative clauses)'' -- I'll
use the latter, more modern term -- we find that in Section
1.1.2.3.1 the marking of relative clauses was to be described;
in 1.1.2.3.2 the description should say whether there is a
distinction between restrictive and non-restrictive relative clauses;
and other subsections ask about their word order, etc. Here is
the relevant passage:

\begin{quote}
\small
\underline{\underline{\texttt{1.1.2.3. Adjective clauses
(relative clauses)}}}\\[0.5ex]
\texttt{There is no construction of the adjective clause (relative
clause) type. There are various means used to obtain the same effect
as such a clause: simple nominalization; placing NPs together in a
paratactic relationship, with intonational break; descriptive
sentence, usually involving an equative clause (see 1.2.1.1.4);
or some combination of these means.}
\end{quote}

\noindent
So relative clauses did not exist in Hixkaryana either.

\textit{LDSQ} also requires that 1.1.2.4 should cover ``adverb clauses,''
i.e.  clauses functioning as modifiers of location, manner, purpose,
cause, condition, result, or degree (1.1.2.4.2.1 -- 1.1.2.4.2.7).
On these, Derbyshire says:

\begin{quote}
\small
\underline{\underline{\texttt{1.1.2.4. Adverb clauses}}}\\[0.5ex]
\texttt{The nearest equivalent to adverb clauses is what I have
called adverb pseudo-clauses, for the same reason that I use
the term ``pseudo-clause'' in connection with nominal constructions
(see 1.1.2.2.6). These adverb pseudo-clauses are either (i)
postpositional phrases with a derived nominal as head of
the phrase, or (ii) constructions whose nuclear element is a
pseudo-nominal, without a postposition...}
\end{quote}

\noindent
Thus Hixkaryana also lacks finite clauses serving adjunct function;
it uses noun phrases (NPs) or phrases headed by adpositions
(postpositional ones, henceforth PPs).

One other relevant thing Derbyshire reports (Section 1.3, p.\,45) is that
``There are no formal means in the language for expressing coordination at
either the sentence or phrase level.'' The English coordinators \data{and},
\data{but}, and \data{or} have no direct equivalents.

To summarize, everything one can immediately think of that might be used
as the basis of an argument that sentences could be of arbitrary length
in Hixkaryana is ruled out. Hixkaryana could have been mentioned among
the languages I discussed in Section~\ref{intro} for which the possibility
of an infinite sentence inventory had been questioned in the literature
long before 2005.

\subsection{Everett's 1986 grammatical sketch}

Everett's description of Pirah{\~a} (\citeyear{Everett86HAL}), a revised
English version of the descriptive part of his PhD dissertation occupying
125 pages of \textit{HAL\,1}, is considerably more than a sketch. It
gives Section 14 (p.\,262) a longer introduction than other descriptions
in \textit{HAL}, postponing exemplification for the more detailed
subsections that followed. He mentioned topics like nominalization,
parataxis, and the expression of temporal and conditional adjuncts,
and but mostly commented on the complex verb morphology of the
language, which allows for new verbs to be formed by including more
than one verb root in a single word. Everett calls this ``verb
incorporation,'' mentioning the phenomenon known in relational grammar
as clause union, but what he calls verb incorporation lacks two
defining features of clause union: the amalgamated verb roots are
invariably understood with the same predicand, and (significantly)
he mentions that evidence of ``underlying bisententiality'' is absent

Everett states unequivocally that ``There is no preclausal complementizer
such as English \data{that} in Pirah{\~a}'' (p.\,262). In the early 1980s
it was of course very natural to look for a ``complementizer'': Everett
was strongly interested in government-binding theory (his dissertation
title includes the words ``and the theory of syntax''), and he wanted to
show how transformational grammar would apply to Pirah{\~a}. But there
was no COMP node to be found, because there were no finite complement
clauses for them to introduce. This means the familiar right-branching
nested English constructions that we invariably exhibit to undergraduates
in our syntax classes (\data{A~knows that B~said that C~thinks that $P$})
cannot be paralleled in a single Pirah{\~a} sentence.

Having noticed this, Everett voiced his suspicions to Noam Chomsky in
conversation. Directly after receiving his PhD, before \textit{HAL~1}
was published, he received a fellowship enabling him to spend a year
(1984--85) as a visiting scholar at MIT, where he had a conversation
that he describes as follows (\citealt{Everett07}:12, fn.\,7):
\begin{quote}
I talked to Chomsky about my idea that there seemed to be very
little evidence for embedding of any kind in Pirahã, apart from
these \mbox{\data{-sai}} examples which I was beginning to question.
We discussed it briefly and Noam gave me some ideas for further
testing the idea. Mark Baker, writing his PhD under Noam at
the time, mentioned to me one day as we were having lunch that Noam
was really intrigued by the idea that a language might not have
embedd[ing] (Mark said something like ``You really got Noam's
attention with what you told him about Pirah{\~a}'' \ldots).
\end{quote}
Chomsky, then, had heard about the apparent lack of embedding in
Pirah{\~a} from Everett himself, twenty years before the \textit{CA}
paper, and was quite interested.

Everett adds: ``I had a growing suspicion that my 1982 analysis was
wrong, based \ldots\ on artificially and exclusively elicited data''
(I return later to the highly significant issue of data elicitation),
but he says he ``did not take the time to work out an analysis with
no hypotaxis at all until 2004, when working at the Max Planck
Institute in Leipzig.''

\section{Subordination and nominalization}

NP\&R were well aware that there were Amazonian languages that seemed
to use nominalizations to do the work that English would do with
subordinate clauses. They make this relevant point:
\begin{quote}
As is well known, it is quite common for embedded clauses to look
more ``nominal'' than their main-clause counterparts, due to a partial
or complete suppression of tense, aspect, or agreement distinctions
found in the verbs of main clauses. Koptjevskaja-Tamm (1993) adopts
from Stassen 1985 the term \textsc{deranked} (vs.\ \textsc{balanced})
for reduced embedded clauses of this sort. Koptjevskaja-Tamm offers
many examples of languages that (either exclusively or quite generally)
use deranked constructions with nominal properties for complement-clause
embedding.
(\citealt{NevPesRod09a}:370)
\end{quote}
They cite languages like Adyghe, Ancient Greek, Classical Latin, Inuktitut,
Quechua, and Turkish as illustrating such ``deranking,'' and add that
``deranked embedded clauses appear to be common among Amazonian languages,''
citing \citet{Derbyshire87} and several descriptions from \textit{HAL},
among them Wai Wai, Macushi, and the \textit{HAL\,1} chapter on Apalai
\citep{Koehn86}.

What they don't mention is that they are just repeating this point
from Everett (\citeyear{Everett05}:629). It is Everett who cited
Koptjevskaja Tamm's book. And that book is about nominalizations, not
subordinate clauses. If we ``rank'' constituents by reference to main
clause features such as tense, nominalizations could be regarded
intuitively as ``deranked'' compared to content clauses. But
nominalizations are NPs, not clauses. \data{We were unaware that the
enemy had destroyed the city} has a subordinate clause in it, but
\data{We were unaware of the enemy's destruction of the city} does
not. After the publication of \citet{Chomsky70}, generative
grammarians ceased even trying to derive nominalizations
transformationally from clauses.

What's more, linguists still do not know how to draw a clear line between
embedded clauses and nominalizations. It is clear even for English.
There are several constructions that can (at least approximately)
express the semantic content of a clause in a less assertion-like
or prominent way.
Some express the downgraded material in a clause-like constituent
that lacks certain main clause properties such as tense or agreement;
but others, like Hixkaryana, have only very rough semantic parallels
to clause structures, exhibiting both the structure and the distribution
of NPs.  Consider the following English expressions related to the
declarative main clause \data{I~ate~it}:

\ea \ea \data{that I ate it} \\\relax
        [finite content clause]
    \ex \data{for me} \data{to eat it} \\\relax
         [infinitival clause]
    \ex \data{me eating it} \\\relax
         [``\textsc{acc}-\textit{ing}'' construction]
    \ex \data{my eating it} \\\relax
        [``\textsc{poss}-\textit{ing}'' construction]
    \ex \data{my eating of it} \\\relax
        [event nominalization, genitive determiner NP as agent]
    \ex \data{the eating of it} \\\relax
        [event nominalization with definite article]
    \z
\z
\noindent
Uncontroversially, (1a) is a transitive content clause, and most
modern linguists would call (1b) a transitive clause too.
And (1f) is certainly a simple definite NP. But in between there are
other constructions. The trouble starts with (1c). Linguists differ
radically on where clauses stop and NPs begin.
The morphology of the head in (1c) and (1d) is no help:
the \textit{-ing} verb form is called the ``gerund-participle'' in
\textit{The Cambridge Grammar of the English Language}
(\citealt{HuddPull02}) because no verb in English distinguishes the form
used in the progressive aspect (\data{I am eating it}) from the form used
in (1c) and (1d). Morphology therefore does not help draw the line
between clauses and NPs (after all, many words ending in
\mbox{\itshape-ing}, though derived from verb roots, do not belong
to verb lexemes at all).\footnote{\label{gerundsubject}
   An unhelpful irrelevance, which I will ignore, is that many
   prescriptive usage authorities insist that (1c) is a deprecated
   form that should be corrected to (1d). I take this view to be
   untenable; the more scholarly usage manuals reject it, noting the
   free variation between them found throughout English literature.}

The generative literature on these constructions has considered
arguments based on a wide range of phenomena; \citet{Pullum91}
gives a systematic survey of the data. Calling (1d) an NP accounts
nicely for the way it can be the object of a preposition, as in
\data{She disapproved of my eating it}. Jackendoff
(\citeyear{Jackendoff77}:222--223) accordingly takes that view; so
does \citet{Pullum91}; and so does \citet{Blevins05}, despite having
criticisms of \citet{Pullum91} and citing others who disagree with it.

\citet{Kiparsky17}, however, carefully argues for treating (1d) as
a clause -- but a clause with the unusual property of needing (in
Chomskyan terms) to be assigned case, as NPs are. That is essentially
what \citet{Stowell81} also advocated. It agrees with Jackendoff,
Pullum, and Blevins that (1d) has the external syntax of an NP, but
differs by assigning it the root node label that clauses have.

Neither Pullum nor Kiparsky is very clear on the status of (1c), the
so-called ``\textsc{acc}-\textit{ing}'' construction. Blevins argues
firmly that it too is an NP. However, Rodney Huddleston convinced me,
a decade after I wrote \citet{Pullum91}, that it is a clause, and also
that the ``\textsc{acc}-\textit{ing}'' and ``\textsc{poss}-\textit{ing}''
constructions are too similar in both external and internal syntax
to make it plausible that one is a clause and the other is not.
So \textit{The Cambridge Grammar} treats both (1c) and (1d) as
non-finite subordinate clauses differing only in the superficial
case-marking of the subject. My earlier view disagrees with my later
view, and I am still not entirely sure which is right. (I was
lucky enough not to face an inquisition by NP\&R accusing me of
trying to dishonestly cover up my earlier view.)

There is much more generative literature on ``\textsc{acc}-\textit{ing}''
and ``\textsc{poss}-\textit{ing}'' constructions than I can discuss here,
but the bottom line is that six decades after the earliest generative
studies of English nominalization and subordination, there is still
no sign of broad agreement on where to draw the line between NP and
clause constituents.  And if linguists are not clear where we should
draw the line between clauses and nominalizations in English, we can
hardly be confident about answering similar questions in vastly
less-studied languages. For Nevins and colleagues to claim they know
exactly where to draw the line between clauses and NPs for Pirah{\~a}
is absurd hubris.

\subsection{A few Pirah{\~a} examples}

NP\&R spend 50 pages of \textit{Language} trawling through Everett's
work looking for dishonesty. They blow plenty of smoke but come up
with essentially nothing definitive. I'll discuss just three examples
that might appear to be of interest because their English translations
contain non-finite subordinate clauses. They can be found in
\citet{Everett86HAL} Section 14.2.1, headed ``Infinitives, participials
and gerundives,'' pp.\,262--263 (just the terms that might be used if
the section were describing English).\footnote{\label{transcription}
   In citing Pirah{\~a} I'll follow Everett's transcription, except
   that his orthographic `\data{x}' for the glottal stop consonant is
   singularly hard for a linguist to get used to, so I replace it with the
   IPA glottal stop symbol `{\textglotstop}' in transcribed examples.}

\ea \ea \gll \data{K{\'o}{\textglotstop}oi} \data{so{\textglotstop}{\'o}{\'a}}
        \data{{\textglotstop}ib{\'\i}ibiha{\'\i}}  \data{tiob{\'a}hai}
        \data{bi{\'\i}o}  \data{kai-sai} \\
        K{\'o}{\textglotstop}oi already order.\textsc{prox}.\textsc{relcert} 
           child grass do\,[+\data{sai}]\\
    \ex \gll \data{hi} \data{ob{\'a}a{\textglotstop}{\'a}{\'\i}} 
             \data{kaha{\'\i}} \data{kai-sai} \\
             3rd  see/know.\textsc{intens}  arrow make\,[+\data{sai}]\\
   \z
\z

\noindent
For (2a) and (2b) Everett gives English translations containing
infinitival subordinate clauses. His free translation of (2a) is
`K{\'o}{\textglotstop}oi already ordered the child to cut the grass'
(where `\textsc{relcert}' is an epistemic mood suffix signaling a
report of something relatively certain). His translation of (2b)
is `He really knows how to make arrows.' NP\&R seize upon these as
examples of the subordinate clauses that Everett is supposedly
now trying to conceal. But Everett actually took both to be
nominalizations correspond to the subordinate clauses in English
(an echo of the way Derbyshire had found nominalizations doing the
work that English does with subordinate clauses). Both have the
verb stem \data{kai}, which is like French \data{faire} in covering
the meanings of both `do' and `make'. The constituents at issue are
\data{bi\'{\i}o~kai-sai} (grass-doing) and \data{kaha\'\i~kai-sai}
(arrow-doing).

In the 1980s Everett thought \mbox{\data{-sai}} was a nominalizer,
glossing it `\textsc{nomlzr}', and he continues to gloss it as
`nominalizer' in the \textit{CA} article (where it is misprinted
several times as `\textsc{nominative}' owing to careless proofreading).
This could mean that the examples might have been better translated
as `K{\'o}{\textglotstop}oi already assigned the child the grass-cutting'
and `He really knows arrow manufacture.' NP\&R, of course, have no idea
whether the NP analysis is correct, or whether we are looking at
subjectless non-finite VPs.

The results of their poring over Everett's work cannot be construed as
adequate support for the claim they want to make -- that Pirah{\~a}
has clause embedding of the sort familiar from the Indo-European
languages.

A few pages later Everett gives (in his (290) on p.\,278) example
\REF{ex:pullum:3},
which might look more promising as a case of a subordinate clause.

\ea\label{ex:pullum:3}
\gll \data{hi} \data{ti} \data{{\textglotstop}api-sai}  
           \data{{\textglotstop}ogi-hiab-a} \\
         3rd 1st go\,[+\data{sai}] want.not.\textsc{remote}\\
\z
\noindent
It consists of a 3rd-person pronoun, a 1st-person pronoun, a
verb meaning `go' with \mbox{\data{sai}} suffixed, a verb stem that
means `want', the negative suffix \data{hiab}, and the remote aspect
suffix \data{a} (on which see \citealt{Everett86HAL}:293--94). In
his early work, up to 1986, Everett thought it might best be translated
as `He doesn't want me to go.' NP\&R seize upon it as a highly significant
case of his having cited a sentence with a subordinate clause in
object position preceding a matrix verb of desiring (see their (23)
on p.\,375). It surely could not be plausibly treated as two successive
main clauses in paratactic relationship.

But \REF{ex:pullum:3} is problematic in a way that Nevins et al.\ were
unaware of -- and here they fell victims to their policy of avoiding
all contact with Everett. Looking back at the origin of sentence
\REF{ex:pullum:3}, Everett recalls that he constructed it himself,
and asked speakers whether it was acceptable -- a use of the
problematic ``can-you-say'' question.

Everett was never able to make much use of questions put to speakers
in his language learning.  ``How-do-you-say'' questions
(\citealt{Samarin67}:114, Ch~6; \citealt{SakeEver12},
{\textsection}6.4) were ruled out because he had no contact language in
which to ask them.\footnote{\label{voegelin}
   Fastidious field linguists shun them anyway, even when a contact language
   is available. Bloomfield never used them at all, according to
   \citealt{Voegelin60}:204.}
Hardly any Pirah{\~a} men (and none of the women) have even the crudest
smattering of Portuguese (again, see \citealt{Sakel12}); no one raised
as a native speaker in the Pirah{\~a} community seems ever to have
subsequently become fluent in Portuguese. Everett does mention that
early on he would sometimes be able to point to something and ask
``How do you say that?'' (\citealt{Everett08}:20) -- presumably
to elicit a noun; but that won't do for most concepts.

Later on, when he had attained a basic grasp of the language, he relied
a lot on ``perambulatory elicitation'' (\citealt{Everett86HAL}:200), which
means walking around the village chatting to people. But that still
cannot be called upon to elicit some key form that will help resolve
some puzzle about syntactic possibilities. When his conversational
abilities had improved enough, therefore, Everett sometimes used
``can-you-say'' questions. These have the advantage of being usable in
a fully monolingual situation, given only enough command of the target
language to express the question ``Can you say $S$?'' and pronounce the
conjectured candidate utterance $S$. So it becomes possible, at least
potentially, to check hypotheses about what is grammatical. But of
course you don't know what you're going to get.

This mode of proceeding calls for great caution, especially when
working with linguistically unsophisticated speakers (which will be
most speakers of most languages in the world, of course). ``Can you say''
questions presuppose that the consultant will understand that the $S$
is being mentioned, not used, and that the linguist is not asking
for permission to say something, or asking about physical possibility,
but rather wants a judgment of concerning grammatically correctness
in isolation from context. What Everett discovered in later years was
that the Pirah{\~a} had regularly been saying ``Yes'' to his occasional
``Can-you-say-$S$?'' questions, just to humor him, even if the $S$ was
decidedly unidiomatic.

Everett was caught out by exactly this behavior in another case. Early
in his study of Pirah{\~a} he assumed it obviously should be possible
to have more than one attributive modifier in the structure of a
Pirah{\~a} NP, just as in English. In example (268) of Everett
(\citeyear{Everett86HAL}:273) he cited \REF{ex:pullum:4} as the largest NP he had
in his corpus (and I give his 1986 glosses):

\ea\label{ex:pullum:4}
\gll \data{kabog{\'a}ohoi}  \data{bi{\'\i}si}  \data{ho{\'\i}hio} 
     \data{{\textglotstop}ita{\'\i}{\textglotstop}i} \\
     barrel   red   two   heavy \\
     \glt `two heavy red barrels'
\z

\noindent
The two modifiers might suggest modifiers can be stacked in NP. But
he had made several errors with \REF{ex:pullum:4}. The example wasn't really
in his corpus in any strict sense. He expressed unease even when citing
it, acknowledging that the example ``is rather artificial'' and ``was
not taken from textual material but rather was separately elicited.''
He later became convinced that the example is ungrammatical. Just
as he discovered that \data{bi{\'\i}si} (based on \data{bi{\'\i}}
`blood') means `bloodlike' rather than `red', and \data{ho{\'\i}hio}
doesn't mean exactly 2 but rather `a~couple' or `a~bit' (in a vague
sense that implies roughly 2 or 3 with count nouns), he also learned
that it was another case of informants who said things were fine as
a way of being tolerant of his imperfect grasp of their language:
they would nearly always assent to his ``can you say'' questions. When
he finally persuaded a speaker to give him the straight truth on
whether \REF{ex:pullum:4} was acceptable, he was told: ``Pirah{\~a} don't
say that.  You can say that. You are not Pirah{\~a}''
(\citealt{Everett09}:422).

The same sort of thing seems to have happened with \REF{ex:pullum:3}.
Since Everett never recorded anything like it in spontaneous use, he
recently decided to seek a second opinion on it from Keren Madora (the
only outsider who has lived with the Pirah{\~a} longer than Everett, and
the only other outsider who is truly fluent in Pirah{\~a}).
Formerly married to Everett, today she still lives very near the
Pirah{\~a} area and is in regular contact with speakers. Her opinion
(email, Madora to Everett, 10 January 2023) was that he is correct,
\REF{ex:pullum:3} is ungrammatical. Pirah{\~a} speakers never
spontaneously say anything like \REF{ex:pullum:3}.

Highly relevant information concerning the suffix \mbox{\data{-sai}}
was published in 2010 but was not available in 1986 or 2009. New empirical
evidence indicates that \mbox{\data{-sai}} is not a nominalizer at all.
Two of the only linguists outside of SIL who have worked directly
with Pirah{\~a} speakers in a context where they could get reliable
translations, Jeanette Sakel and Eugenie Stapert, constructed some
test sentences by concatenating two Pirah{\~a} clauses translatable
as `it's raining' (\data{piiboibai}) and `I don't go' (\data{ti
kah{\'a}pihiaba}), intended to suggest the meaning `If it's raining,
I don't go', and suffixing \mbox{\data{-sai}} to either the first
clause or the second. They then asked nine speakers (seven women, two
men) to simply repeat back what they'd said. They found that the
informants' responses might have \mbox{\data{-sai}} on the first
clause, or the second, or both, or neither, regardless of which input
sentence they were given.

Their conclusion (see \citealt{SakeStap10}:5--6) is that
\mbox{\data{-sai}} ``does not appear to be a marker of subordination,
as originally claimed by Everett (1986)'' (and they mean that it is
not a marker of nominalization either). Everett agrees, and now
believes it may be an optional marker for sentences conveying
discourse-old information. Its random placement in sentence repetitions
would be as expected if its old-information signaling role only made
sense in a discourse context: speakers charged with repeating two
sentences with no context apparently recalled vaguely that there was
a \mbox{\data{-sai}} in there somewhere, but didn't necessarily
remember where.

What does it mean for sentences like (2a) and (2b), if \mbox{\data{-sai}}
might not be either a nominalizer or a subordination marker after all?
I'm not sure. And I don't think anyone really is. But when looking at
attested Pirah{\~a} examples, with their short clauses and unclear
syntactic linkages, it is definitely useful to recall the perceptive
remarks of \citet{Liberman06} on Language Log, published before either
the Nevins boycott move or the first draft of NP\&R's paper, about
sentences in conversational English as recorded by novelists with a
good ear for colloquial speech. Liberman gives examples from Elmore
Leonard. One character is quoted as saying things like `We get to a
phone, we're out of the country before morning.' In the context it
is clear that the intended meaning is conditional. One can imagine
such a speaker saying, `It's raining, I don't go.' Everett cites very
similar examples of what he then thought were conditional clauses.
For example (\citealt{Everett86HAL}:265, ex.~(241)):

\medskip\noindent
\begin{tabular}[t]{lllllll}
(5) &\multicolumn{6}{l}{\data{Pai{\'o} hi ab{\'o}paisa{\'\i}
            ti {\textglotstop}i{\'\i} o{\'a}bo{\'\i}ha{\'\i}.}} \\
    &\data{Pai\'o}&\data{hi}&\data{ab-\'op-ai-sai}&\data{ti}&
          \data{{\textglotstop}i\'\i}&\data{o\'a-bo\'\i-ha\'\i} \\
    &(name)&3sg&turn-go-\textsc{atelic}-\textsc{cond}
                        &I &thing&buy-come-\textsc{near-certain}\\[0.5ex]
    &\multicolumn{6}{l}{`Pai\'o comes back, I'm gonna buy something.'}
\end{tabular}

\medskip
I am not in any way suggesting that everything is now resolved and
the picture is clear. Far from it. We have no truly reliable principles
to use in order to decide whether some Pirah{\~a} construction is
more analogous to \textit{if he returns} or \textit{him returning}
or \textit{his returning} or \textit{his return}. All sorts of
unclarities remain. Everett acknowledges having made errors in both
elicitation and analysis; in 1986 he thought \mbox{\data{-sai}} was
a morpheme forming subordinated constituents of some kind, probably
nominalizations that played the role subordinate clauses would play
in European languages, but after the convincing work of Sakel and
Stapert he no longer thinks that. It has been definitely confirmed
that \mbox{\data{-sai}} sometimes appears on what in English might
be a subordinate clause but also sometimes appears on what in English
would be a main clause.

In 1986 Everett also thought there were two \mbox{\data{-sai}}
morphemes, differing in tone, but subsequent F0 measurements by Miguel
Oliveira have revealed no statistically significant tonal difference
(a rough set of slides presenting the results was made available as
\citealt{OlivEver10}). Everett now thinks there is just one
\mbox{\data{-sai}}.

Given the present state of our knowledge, we certainlt cannot say that
NP\&R have refuted Everett's thesis about Pirah{\~a} sentences never
exhibiting clause embedding. One might perhaps argue that the case
is still open, but not that NP\&R examined the matter and settled it
-- which is what far too many linguists (Chomsky included) have been
lazy enough to assume. Simply citing \citet{NevPesRod09a} without
getting into any of the details is not sufficient. Those who are truly
intent on trying to support the ungracious claim that Everett lied
are going to have to start learning Pirah{\~a}.

\subsection{The crucial issue of embedding depth}

There is a vital point about nominalizations that NP\&R either failed
to notice or chose not to mention. What we really need to know, if we
are to address the only issue that makes this discussion sensible, is
whether a Pirah{\~a} nominalization (or non-finite clause or whatever)
can be embedded inside another, and the result inside another, and so
on, to arbitrary depths. NP\&R struggle to find even a single case of
a fully clear subordinate clause in Everett's early work (and they
never venture to propose a structure for even a single sentence), but
they certainly never even touch on the matter of showing embedding that
can be reiterated to arbitrary depth. Nothing they say suggests that
subordination in Pirah{\~a} (if it has any) can give rise to sentences
of arbitrary length. And that is what any serious notion of ``recursion''
has to be about.

In Standard English, after more than a thousand years of literacy
(which \citealt{ONeil77}, \citealt{Givon79}, \citealt{Mithun84}, and
\citealt{Kalmar85} suggest might be a crucial consideration) now has
fairly rich nominalization resources: even a clause like
\data{A~knows} [\data{that B~said} [\data{that C~thinks}
[\data{that D~predicts} [\data{it~will rain}]\,]\,]\,] can be paired with
a cumbersome NP analog like \data{A's~knowledge of B's statement about
C's opinion concerning D's prediction of impending rain} with roughly
the same content.\footnote{\label{cartoon}
   The reader might like to consider whether one could construct a
   nominalization that exactly captures the content of the husband's
   thought in Bruce Eric Kaplan's well-known \textit{New Yorker}
   cartoon (26 October 1998), where a man earnestly assures his wife:
   ``Of course I~care about how you imagined I~thought you perceived
   I~wanted you to feel.''}
But are such multiple embeddings of NPs constructible in every language?
I have never been able to see a way in which the nominalization resources
of languages like Hixkaryana, Apalai, or Pirah{\~a} could be used to
replicate any such internally ramified NP constructions. The most that
NP\&R have to suggest is that in one or two Pirah{\~a} examples there
may be depth-1 subordination of a non-finite secondary predication, but
they really only have what look like adsentential modifying phrases
appended to a clause. They cannot exhibit Pirah{\~a} evidence supporting
the claims of so many linguists that iterated embedding in human
languages is always allowed to unbounded depth. That is the claim
Everett was challenging.

\section{Hallucinated PP self-embedding}\label{ppsection}

The work presented in \textit{Recursion Across Domains} \citep{AmMaNeRo18}
is of astonishingly low quality, replete with glaring mistakes. The review
by \citet{EverGibs19} provides a selection of the evidence, concentrating
most on Pirah{\~a}, on which the authors had worked together in the field.
For the second review that \textit{Language} commissioned, the editors
certainly found the right man for the job: \citet{Hornstein19} faithfully
repeated Chomsky's theoretical position on ``recursion,'' elaborating
the rhetorical escape-hatch arguments (see Section~\ref{intro} above),
and then proceeded to uncritically endorse all data-oriented contributions
in the book regardless of their merits. Thus he reported that by using a
truth-value judgment experiment Uli \citet{Sauerland18} had managed to
``provide pretty dispositive evidence that Pirah{\~a} allows sentential
embedding under `say'\,'' (p.\,796). In truth Sauerland's statistical
analysis has vitiating flaws, and when his experiment is run on English
speakers it does not produce the results that would be needed to
support his claims anyway (see the analysis by
\citealt{EverGibs19}:781--784, who took the trouble to review the use
of statistics and test his experimental design on English speakers,
and the more detailed critique by Gibson in this volume).

I will not attempt a general survey of the material in
\textit{Recursion Across Domains} here, but I will just address a
particularly incompetent chapter about Pirah{\~a} PPs. Neither of
the \textit{Language} reviews mentioned the stunning error, and
presumably none of the referees for the book noticed it either.

The chapter by Filomena Sandalo, Cilene Rodrigues, Tom Roeper, Luiz Amaral,
Marcus Maia, and Glauber Romling da Silva (\citeyear{SandaloEtAl18})
claims that Pirah{\~a} syntax allows PPs to be embedded inside other PPs,
and reports experiments purportedly showing that native speakers have no
difficulty in processing and interpreting such phrases. The authors assume
(as is clear from their (15) on p.\,285) that the English phrase
\data{the coin on the paper on the chair on the board} has a
right-branching structure with a single NP constituent containing
all the PPs as modifiers of successively embedded NPs:
\data{chair on the board},
\data{paper on the chair on the board}, and so on.

They claim that Pirah{\~a} has precisely analogous phrases, with two
differences. First, Pirah{\~a} lacks determinatives such as the English
definite and indefinite articles; accordingly, it makes sense to ignore
articles in the English structure shown below -- it simplifies the tree
structure considerably.  And second, Pirah{\~a} PPs are postpositional.
The right-branching structure for English diagrammed (without articles)
in (6a) is claimed to have an analogous left-branching structure in
Pirah{\~a} with the terminal string `\data{tabo apo tiapapati apo
kapiiga apo gigohoi}'. (Sandalo et al.\ mistranscribe all of these
words, but I set that aside that for now.) Taking into account ``the
fact that Pirah{\~a} is a head-final language,'' they assume that an
English structure in (6a) -- where I omit determiners to save space
-- has an exact analog in Pirah{\~a}, which they depict
(in their (19) on p.\,287) as shown in (6b).

\noindent
(6) a. PP modifiers of NP in English\hspace*{3ex}
    b. Sandalo et al.'s Pirah{\~a} PP structure

\nopagebreak[4]

\noindent
\hspace*{1.5em}\scalebox{0.6}{\includegraphics{figures/pullum_figure1.pdf}}
\hspace*{2em}\scalebox{0.55}{\includegraphics{figures/pullum_figure2.pdf}}

\noindent
Sandalo et al.\ have overlooked a crucial syntactic fact. Pirah{\~a} is
\textsc{not} a uniformly head-final language. As Everett noted forty
years ago, in the noun phrase ``modifiers follow, while possessors normally
precede, the phrase head'' \citep[272]{Everett86HAL}. He lays out
the sequence of elements in the NP as follows
(p.\,273):\footnote{\label{nonumerals}
   See also Everett (\citeyear{Everett83}:132ff). Pirah{\~a} has no true
   numerals in the sense of names for the natural numbers, but presumably
   its vague quantity-related items like \data{b{\'a}agiso} or
   \data{{\textglotstop}a{\'\i}b{\'a}} `many',
   \data{{\textglotstop}ogi{\'\i}} `a lot', and
   \data{{\textglotstop}o{\'\i}hi} `few' take that slot in the NP.}

\medskip\noindent
(7)\quad
%%% GKP: Everett's full-size caps won't fit the page width:
%%% (POSSESSOR) + (PRO.CLITIC) + N + (MODIFIER) + (NUMERAL) + (DETERMINER)
%%% So I use small caps instead:
(\textsc{possessor}) + (\textsc{pro.clitic}) + N + (\textsc{modifier})
+ (\textsc{numeral}) + (\textsc{determiner})

\medskip\noindent
The vital point is that modifiers follow the head in NPs. So if there
were noun-modifying postpositional PPs embedded in NPs within other
such PPs, the result would be nothing like the fictive left-branching
tree in (6b). In fact there's a good reason that languages with nouns
postmodified by PPs don't allow iteration of the construction: it
yields center-embedding of the sort that poses major difficulties for
human sentence processing -- the kind seen in English center-embedded
sentences like \textsuperscript{??}\data{The children the women the
soldiers left saved protested}.

The purported phrase Sandalo et al.\ are trying to diagram would
actually come out as in \figref{fig:pullum:3}, where I correct the transcriptions
and word identification as well as the structure.

\begin{figure}
\includegraphics[width=\textwidth]{figures/pullum_figure3.pdf}
\caption{Expected structure if Pirah{\~a} had nesting of PP modifiers in NPs}
\label{fig:pullum:3}
\end{figure}

No one has ever suggested that PPs like in \figref{fig:pullum:3} are
encountered in Pirah{\~a} discourse, and no such structures were
presented to Sandalo et al.'s hapless
informant.\footnote{\label{ppstarlabels}
   Later they give a second similar structure for NPs containing PP
   modifiers which is best ignored. Their (21) on p.\,287 has nodes
   labeled ``PP*'' dominating other nodes with that label. On p.\,284
   they say they are using ``notation adapted from traditional Kleene*
   system'' [sic], but Stephen Kleene's star notation symbolizes a
   unary operation mapping a set of strings to its reflexive and
   transitive closure under concatenation. It makes absolutely no
   sense in a node label.}

It is difficult to guess what must have gone on in their experimentation
(they stress that it is to be regarded only as a pilot study). They claim
to have found that a native speaker named Iao{\'a} understood their
pronunciation of the purely fictional phrase (6b). Given that the word
they write as \textit{tiapapati} seems to be the imperative verb
\data{t{\'\i}apapa{\'a}ti}, meaning `sit down'
(\citealt{EverGibs19}:786-87), Iao{\'a} would have heard them as
saying something that meant roughly `Sit on the board. On top. On
the paper. Money.' The corrected string is given in (9):

\medskip\noindent
\begin{tabular}[t]{lccccccc}
(9)&\data{t{\'a}bo}&
    \data{{\textglotstop}apo{\'o}}&
    \data{t{\'\i}apap}&
    \data{{\textglotstop}apo{\'o}}&
    \data{kapiiga}&
    \data{{\textglotstop}apo{\'o}}&
    \data{gi{\'\i}go-ho{\'\i}} \\
   &board&on&chair&on&paper&on&money
\end{tabular}

\medskip\noindent
The most likely guess at how Iao{\'a} or any native speaker would
have parsed this would be as a list of successive PPs and a final NP,
as in (10).

\medskip\noindent
(10) Most likely native-speaker parse of (9)
\nopagebreak[4]

\quad\scalebox{0.6}{\includegraphics{figures/pullum_figure4.pdf}}

\medskip\noindent
Convinced that they had identified nested PPs in Pirah{\~a}, Sandalo
et al.\ (\citeyear{SandaloEtAl18}:289--292) proceeded to construct some
test sentences paired with pictures of alligators on mats on rocks on
beaches, and claim to have used them to produce evidence for
interpretation of nested PPs. They claim a picture of an alligator on
a mat on a beach was reliably distinguished from a picture of an
alligator on a mat beside another alligator on a beach. Further
discussion of this experiment is not really feasible; their account is
too ill-informed and confused, replete with botched transcriptions,
mistaken glosses, misidentified words (\data{tahoasi} is glossed as
`mat' when it actually means `beach' -- the word for `mat' is
\data{paah{\'o}{\'\i}s{\'\i}}), and so on.

In another experiment they tried to get Iao{\'a} to play a ``game''
involving coins being put on a paper that was on a chair on a board,
or on a paper on a chair, or on a paper on a board. They note (p.\,294)
that where they supplied a string like ``gigohoi kapiiga apo tiapapati
apo tabo apo'' (intended to be \data{gi{\'\i}go-ho{\'\i} kapiiga
{\textglotstop}apo{\'o} t{\'\i}apap {\textglotstop}apo{\'o} t{\'a}bo
{\textglotstop}apo{\'o}}, glossed `coin paper-on chair-on board-on'),
when Iao{\'a} repeated the string ``he switched the order of the PPs in
the sentence,'' yielding what they wrongly transcribe as `tabo apo
tiapapati apo kapiiga apo gigohoi' (`board-on chair-on paper-on coin').
This was a sign of something gone terribly wrong: Iao{\'a} was unable
to come anywhere near repeating what they thought was a single NP in
his language. But in an almost unbelievable fit of wishful thinking
(hope springs eternal in the human breast), they interpret this as
``spontaneous evidence'' in favor of their hypothesis!
It seems more likely that Iao{\'a} scarcely knew what was going on,
but took their attempted PPs to be independent phrases, not
successively embedded modifiers in an NP, and repeated them back in
LIFO (last in, first out) order.

There is also a very simple semantic observation that may play a
role in interpreting the events that they take as vindication of their
hallucinated PP embedding claims. We normally take the `on' relation
between medium-sized physical objects to be transitive. Any coin on
a piece of paper on a chair is also a coin on a chair. Any alligator
on a mat on a beach is an alligator on a beach.

The most plausible conclusion from Sandalo et al.'s bungled experiments
is that Iao{\'a} parsed the fictive PPs individually, and then (with the
sharp general intelligence Everett has always noted among the Pirah{\~a})
simply guessed what the linguists wanted him to do.

\section{Sentence-length extensibility more generally}

As promised earlier, I have avoided the impenetrable thickets of confusion
found where linguists use the words ``recursive'' and ``recursion''; I have
focused instead on the clearer issue of syntactic devices that can in
principle allow the construction of sentences of arbitrary length.

The issue does not have the fundamental importance that some have seen
in it. Linguistic creativity is not tied to any claim about an infinitude
of sentences, since human linguistic creativity (as Everett has often
stressed) resides mainly at the discourse level.
Nor is it tied to the ability to grasp concepts.
Absence of propositional attitude verbs in a language, for example,
does not entail speakers' inability to engage in metacognition.
Everett deftly illustrates how a complex proposition with a logical
form like [\data{if} [\data{P and~Q}] ] \data{then~R} does not need
to be expressed in one sentence when he titled a conference paper:
``You drink. You drive. You go to jail. Where's recursion?''
\citep{Everett10}.

Everett's opponents seem to assume that linguistic life with only
simple main clauses would hardly be worth living.
But there is no reason to regard a language lacking unbounded sentence
extensibility devices as less useful or expressive than a language.
\citet{Kornai14} argues that the information-carrying complexity of
a finite language can actually be greater than that of an infinite one.

One way of stressing the difference between finite and infinite languages,
often touched on in undergraduate textbooks, depends on pointing out that
for a finite language the grammar could be given in the form of a simple
list of sentences. But that was never a very sensible point to harp on.
From the complexity of verbs alone (\citealt{Everett86HAL}:288--301)
it is apparent that the set of Pirah{\~a} sentences would be way too vast
even to be compiled, stored, or accessed by either a brain or a currently
imaginable computer, let alone to be of real online use either cognitively
or computationally. The grammatical complexity of Pirah{\~a} would still
pose the usual problems for the theory of language acquisition: inducing
generalizations from exposure to data would have to be involved, not just
memorizing complete utterances. As Gibson (this volume) argues, what's
important is compression of information (Kolmogorov complexity), not
infinitude.

Whether the set of all sentences in a language is finite or not is
in any case inherently difficult to settle, for a number of reasons,
and would remain so even if all of Everett's specific claims about
Pirah{\~a} syntax are accepted.

First, the lexicon has to be stipulatively fixed at some finite number
$N$ of words, though we have no clue about what $N$ might be because
new words (e.g.\ personal names) are being coined all the time, and
the interaction of agglutinative word formation and lexicalization
in languages like Turkish or Inuktitut makes it implausible that there
is any such $N$ at all.

Second, the notion ``sentence'' needs a clear definition; syntacticians
casually assume it is a well-understood primitive term, but it is not
easily defined at all. Separating a passage of spoken language into
sentences in a way that a different linguist would replicate is very
difficult, and beset with problems raised by false starts,
parenthetical interruptions, direct quotations, appositional
expansions, rhetorical repetitions, whatever semicolons represent in
writing, and asyndeton (coordination without coordinator words, as
in Dickens's \data{It was the best of times, it was the worst of
times, it was the age of wisdom, it was the age of
foolishness\ldots}').

Third, with regard to hypotaxis (subordination), \citet{PawlSyde00}
argue that it hardly occurs at all in spontaneous speech, even in
English, once we set aside a limited number of high-frequency partially
customizable schemata like \data{I~think~\blank}\, or \data{It depends
whether~\blank}\,, and similar formulas. This would presumably be all
the more true for languages spoken in cultures where no one writes
or reads. A few folk tales or epic poems might have a broadly fixed
(or even faithfully memorized) traditional form, but most language
use will be informal chatting, and Pawley and Syder claim that
spur-of-the-moment construction of hypotactic sentences will be rare
to nonexistent.

There are other phenomena that could introduce difficulties: NP
apposition, roughly definable as adjacent iterated NPs with the same
reference and syntactic function (\citealt{Karlsson10} cites an
attested five-NP example in Swedish); intensificatory or iconic
repetition of attributive adjectives (\data{a~big, big, big problem})
or adverbs (\data{I~really, really mean it}) or VPs \data{They hit
me and hit me and hit~me\ldots}) or NPs (\data{cows, cows, \ldots\
cows, as far as you could see}). Such possibilities are seldom noted
in reference grammars. Only study of large corpora of texts will tell
us whether such iterable sentence-lengthening constructions are found
in the syntax of an exclusively oral language like Pirah{\~a}.

How might we even estimate the likelihood that Pirah{\~a} truly has no
unbounded syntactic resources for sentence lengthening? A beautiful
and oddly neglected paper by \citet{WidmerEtAl17} addreses this
question. Widmer and colleagues suggest some additional methods that
could be employed to figure out the probability of a language lacking
such resources. They identify five ways in which NPs in Indo-European
languages can be lengthened by embedding other NPs inside them: stacked
genitive determiners, adjectivization-derived modifiers, modifiers
with head marking, adpositional modifiers, and simple noun
juxtaposition (I assume apposition is to be included under the latter
heading). They show that Indo-European languages have repeatedly
developed such devices and also lost them through syntactic change
over the past few thousand years.

Through a clever calculation they then assess how likely an Indo-European
language is to end up at a given time with at least one such device in
its NP syntax, concluding that it is very high indeed: they estimate
that with probability $\sim$0.98, any Indo-European language, at any
given point in its history, will have at least one grammatical device
for arbitrarily expanding NPs. As an explanatory conjecture, they suggest
that for some reason the human processing capacity finds it helpful for
there to be some such mechanism provided by the grammar.

However, they add (p.\,822): ``With regard to sentence-level syntax,
it remains an open question whether syntactic recursion or simple
conjunction is preferred.'' To settle it, ``a larger sample of data
would be needed.'' We cannot know what the answer is, or how likely
it is that any arbitrary language in the world (not just in the
Indo-European family) would have some kind of iterable
sentence-lengthening syntactic device available at all times in its
history. But suppose the probability of languages having such features
were as high as $\sim$0.99. It would still be expected, given the
7,000 languages attested in the world today, that there might be 70
languages or more in which such devices are absent. The literature
on ancient languages and languages of preliterate cultures has thrown
up quite a few candidates, as discussed in Section~\ref{intro}.
Pirah{\~a} just happens to be the clearest case -- and the one that
kicked the hornets' nest politically.

\section{Conclusions}

No one should claim, in the present state of our knowledge, that we
have a good understanding of the syntax of Pirah{\~a} (or for that
matter any other language, even Standard English). The corpus study
of Pirah{\~a} syntax by \citet{FutrellEtAl16} is a sterling effort
at utilizing what materials we have (specifically, parsing texts
collected by Steven Sheldon in an effort to find evidence of
subordination), but in many ways it just underlines how woefully
unclear things are. Much more work has to be done.

That work will not be accomplished without collaborations that involve
people who (i) have no advance commitment to particular results or
empirical claims and (ii) are prepared to spend time paying close
attention to everyday usage in the Pirah{\~a} speech community. That
will mean extended residence in Pirah{\~a} villages, and consultation
with people who have substantial experience with the language.

Such people exist. Steven Sheldon worked on the language from
1967 to 1976, and knows it well. Caleb Everett, Kristene Diggins,
and Shannon Russell all learned to speak and understand the language
when living in Pirah{\~a} villages as children, and their parents
Daniel Everett and Keren Madora are outsiders with unprecedented
fluency. Madora has studied the language in depth since 1977 and
still lives close to the Pirah{\~a} villages; Everett spent a total
of about eight and a half years with the Pirah{\~a} between 1977
and 2006, and made various visits thereafter, becoming fully fluent
in the language. He translated the \textit{Gospel of Mark} into it
\citep{Everett86Mark}. Yet NP\&R decided to work without having a
single conversation with any of these people.

This represents a sadly missed opportunity. If linguists like NP\&R
had applied their analytical theoretical abilities to the available
data in a collaborative spirit, drawing on the knowledge of active
speakers of the language (particularly Everett himself), new linguistic
insights might have been gained. That chance has been lost, probably
forever. They have wrecked their credibility by making it so obvious
that from the start they aimed simply to bring Everett into disrepute.
All that linguistics ended up getting out of their work was an
uninformed retrospective document review. They have divided linguists
into two irreconcilable warring camps, and made the entire discipline
of linguistics look, as it did to Tom Bartlett, like a snakepit of
hostility.

Like any scientists, linguists have a duty to maintain ethical
standards and intellectual open-mindedness -- even when someone is
claiming Chomsky was wrong about something, or when the popular
press tries to fluff up a science story into something earth-shaking
or theory-trouncing and publishes absurd overstatements.

Certainly it was ridiculous hyperbole for
\textit{New Scientist} (18 March 2006)
to call Everett's account of Pirah{\~a} ``the final nail in the
coffin for Noam Chomsky's hugely influential theory of universal
grammar.'' If we're honest we'll admit that Chomsky does not have
enough of a detailed theory of universal grammar to constitute a full
coffinload. Nor do his opponents have solid enough empirical accounts
of language acquisition to nail down the lid of such a casket anyway.

It was similarly absurd for the
\textit{Chicago Tribune} (10 June 2007)
to suggest that Everett's work is analogous to a high-school physics
teacher finding ``a hole in the theory of relativity''; but we all know
that sort of thing often happens when popular news media try to cover
science. Providing better and clearer hype-free accounts of our work
to science journalists will be an enduring burden, but one that we
all have to shoulder. Calmly, and with some understanding of the
fragile and difficult business of popular journalism.

I can well imagine how irksome it has been for Chomsky to see overblown
hype about a putatively theory-shaking discovery in the jungle repeated
in scores of news sources. But that doesn't justify the petty spite of
his ``charlatan'' remark to \textit{Folha de S.~Paulo} in February 2009,
or his assertion that ``Daniel Everett's contributions are basically
nothing'' in a 2021 video interview.\footnote{\label{basicallynothing}
   \url{https://www.youtube.com/watch?v=UBla-h36ywA}}

Over the past four decades, Everett can be fairly said to have done
more for Amazonian linguistics than any other linguist now living.
His detailed descriptions of Pirah{\~a} and Wari' are lasting
contributions, as is his energetic promotion and encouragement of
descriptive work on other Brazilian languages. His basic claim about
Pirah{\~a} syntax not permitting unbounded sentence length is very
probably true. He did not deserve the years of hot-tempered public
allegations and insults (or the worse incidents of insult, hate mail,
and shouting in his face that he does not publicly report). A sector
of our field seems to have lost its moral compass over this issue.

It speaks well of Everett that never in all the years since 2005 has
he responded to his tormentors with insults or abuse: he argues points
of fact, but he refrains from accusing his enemies of scientific
misconduct, devious motives, or self-interested mendacity. For that,
and much more, we should salute him.

And as regards the validity of the accusations hurled at him by his
many opponents, none of them familiar with the lives and spoken
language of the Pirah{\~a}, I quote in conclusion the opinion of a
young Brazilian anthropologist writing recently about Pirah{\~a}
culture \citep{Felizes23}\addpages:
\begin{quote}
A rela{\c{c}}{\~a}o de Daniel e Karen Everett com os Pirah{\~a} {\'e}
algo que perdura at{\'e} aos dias atuais. Durante mais de quarenta
anos de convívio – permanente ou espor{\'a}dico – conquistaram a
reputa{\c{c}}{\~a}o de grandes amigos, de saberem bem a l{\'\i}ngua, de
serem exímios contadores de histórias e de se tornarem importantes
aliados, a quem os Pirah{\~a} geralmente recorrem para resolver
potenciais conflitos ou aprender coisas sobre o mundo dos brancos.

\noindent
[Daniel and Keren Everett's relationship with the Pirah{\~a} is something
that has endured to the present day. During more than forty years of
coexistence -- permanent or sporadic -- they gained the reputation of
being great friends, of knowing the language well, of being excellent
storytellers and of becoming important allies, to whom the Pirah{\~a}
often turn to resolve potential conflicts or learn things about the
white world.]
\end{quote}
That is the view formed by an independent third party with a personal
commitment to studying the life of the Pirah{\~a}, some who has spent time
in Pirah{\~a} villages, made the acquaintance of Keren Madora [formerly
Everett], and witnessed the consequences of the Everetts' 46 years of
friendship with the Pirah{\~a} at first hand.


\section*{Acknowledgments}
I have known Daniel Everett and his work since 1983, and gratefully
acknowledge his openness and candor during our discussions of this topic.
During four decades of friendship we have agreed on many things and
disagreed on many others. In writing this paper I have confirmed
claims about specific people's actions and statements from suitable
sources or personal reminiscences as far as that was possible.
I thank many friends who have spent time reading drafts, answering
my questions, helping me to verify facts, saving me from errors,
and making useful suggestions. Among them are
Judith Aissen,
Ash Asudeh,
Peter Austin,
Jim Blevins,
Bernard Comrie,
Peter Culicover,
Hope Dawson,
Lise Dobrin,
Ted Gibson,
John Goldsmith,
Randy Allen Harris,
Lloyd Humberstone,
Brian Joseph,
John Joseph,
Ed Keenan,
Bob Ladd,
Pim Levelt,
Bob Levine,
Noah Ley,
Joan Maling,
John McWhorter,
Philip Miller,
Stefan M{\"u}ller,
Georgia Morgan,
David Nash,
Johanna Nichols,
Steven Piantadosi,
Steven Pinker,
Jerry Sadock,
Jeanette Sakel,
Rich Thomason,
Sally Thomason,
Tom Wasow,
Rebecca Wheeler,
Melinda Wood, and
Annie Zaenen.
Correspondence with David Pesetsky and Cilene Rodrigues helped me to
improve the accuracy of certain claims in section~\ref{war}.
Errors may remain, and they are solely mine.
Presentations of some of the content of this paper were made during
2023 at MIT, George Mason University, the Max-Planck Institute
for Psycholinguistics in Nijmegen, and the 2024 meeting of the
North American Association for the History of the Language Sciences.

%\sloppy


\printbibliography[heading=subbibliography,notkeyword=this]

\end{document}
