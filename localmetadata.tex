\title{From fieldwork to linguistic theory}
\subtitle{A tribute to Dan Everett}
\BackBody{Dan Everett is a renowned linguist with an unparalleled breadth of contributions, ranging
  from fieldwork to linguistic theory, including phonology, morphology, syntax, semantics,
  sociolinguistics, psycholinguistics, historical linguistics, philosophy of language, and
  philosophy of linguistics. Born on the U.S.-Mexican border, Daniel Everett faced much adversity
  growing up and was sent as a missionary to convert the Pirahã in the Amazonian jungle, a group of
  people who speak a language that no outsider had been able to become proficient in. Although no Pirahã
  person was successfully converted, Everett successfully learned and studied Pirahã, as well as
  multiple other languages in the Americas. Ever steadfast in pursuing data-driven language science,
  Everett debunked generativist claims about syntactic recursion, for which he was repeatedly
  attacked. In addition to conducting fieldwork with many understudied languages and revolutionizing
  linguistics, Everett has published multiple works for the general public: \emph{Don’t sleep, there
    are snakes}, \emph{Language: The cultural tool}, and \emph{How language began}. This book is a collection of
  15 articles that are related to Everett’s work over the years, released after a tribute event for
  Dan Everett that was held at MIT on June 8th, 2023.}

\typesetter{Moshe Poliak, Stefan Müller}

\author{Edward Gibson and Moshe Poliak}

\ISBNdigital{978-3-96110-473-4}
\ISBNhardcover{978-3-98554-102-7}
\BookDOI{10.5281/zenodo.11351540}
% \proofreader{}
% \lsCoverTitleSizes{51.5pt}{17pt}% Font setting for the title page

\renewcommand{\lsSeries}{eotms}
\renewcommand{\lsSeriesNumber}{}
\renewcommand{\lsID}{434}
